\documentclass[edeposit,fullpage]{Classes/uiucthesis2009}
\usepackage{booktabs}
\usepackage{multirow} \usepackage{mathtools} \usepackage{tabularx}
\usepackage{bm} \usepackage{setspace} \usepackage{amsmath} \usepackage{hyperref}
\usepackage{pbox,multirow} \usepackage[thinlines]{easytable} \usepackage{lineno}
\usepackage{bigstrut} \usepackage{enumitem}
\usepackage[titletoc]{appendix}
\usepackage{cite}
\usepackage{changepage}
\usepackage{breqn}
\usepackage{subcaption}
\usepackage{caption}
\usepackage{graphicx}
\usepackage{mathrsfs}
\usepackage{multicol}
\usepackage{slashed}
\usepackage[numbers,sort&compress]{natbib}

\singlespacing
%Newcommands --> define here your alias 
\newcommand{\gvc}{GeV/$c$} \newcommand{\gvcs}{(GeV/$c$)$^2$}
\newcommand{\gvcw}{GeV/$c^2$} \newcommand{\mvcw}{MeV/$c^2$}
\newcommand{\mvc}{MeV/$c$} \newcommand{\diff}{\text{d}}
\newcommand{\siv}{$f_{1,T}^{\, q\perp}(x,\mathbf{k_T})$}
\newcommand{\xn}{$x_N$} \newcommand{\xpi}{$x_\pi$}
\newcommand{\xf}{$x_F$}
\newcommand{\qt}{$q_T$} \newcommand{\Mmumu}{$M_{\mu\mu}$}
\newcommand{\atrack}{$\alpha_T$}
\newcommand{\adet}{$\alpha_D$}
\newcommand{\jp}{J/$\Psi$}

% ******************************** Front Matter ********************************
\begin{document}

\title{Transverse Spin Effects From Pions Impinged on a Transversely Polarized
  Proton Target}

\author{Robert Shannon Heitz}
\department{Physics}
%\schools{B.A., University of Columbia, 1981\\ A.M., University of
%  Illinois at Urbana-Champaign, 1986}
\phdthesis
\advisor{Matthias Grosse Perdekamp and Caroline Riedl}
\degreeyear{2019}
\committee{Professor Liang Yang, Chair\\
  Professor Matthias Grosse Perdekamp, Co-Director of Research\\
  Research Assistant Professor Caroline Riedl, Co-Director of Research\\
  Professor John Stack\\
  Professor Alexey Bezryadin}
  
\maketitle

\frontmatter

%% Create an abstract that can also be used for the ProQuest abstract.
%% Note that ProQuest truncates their abstracts at 350 words.
\begin{abstract}
The transverse momentum structure of the nucleon generalizes the 1-dimensional
parton distribution functions (PDFs) to the a 3-dimensional description.  The
extended distributions are called transverse momentum dependent (TMD) parton
distribution functions.  TMDs functions include transverse quark and gluon
momentum to parameterize the non-perturbative description of a nucleon.  Eight
TMDs parameterize a nucleon at leading order and the Sivers TMD is unique in
that it is spin-dependent and theorized to change signs between the Drell-Yan
process and Semi-Inclusive Deep Inelastic Scattering (SIDIS).

The COMPASS spectrometer is unique in that it has the ability to study the
products from a polarized target and the beam can be modified to measure the
SIDIS and Drell-Yan processes.  The COMPASS collaboration collected transverse
spin-dependent data to measure the Sivers function from SIDIS and in 2016
COMPASS published the results of a Sivers asymmetry amplitude from the SIDIS
process.  In 2015 COMPASS collected data to study the Drell-Yan process from a
transversely polarized proton target and a 190~{\gvcw} $\pi^-$ beam.  Therefore
COMPASS has the unique ability conclude on the non-universality of the Sivers
TMD between the SIDIS and Drell-Yan processes.

This thesis presents analysis of the 2015 COMPASS Drell-Yan data taking.  The
analysis focuses on the Sivers TMD but also provides results on the tranvsersity
and pretzelosity TMD functions which are expected to be universal.  The Sivers
results presented in this thesis are consistent with a sign flip between
Drell-Yan and SIDIS.
\end{abstract}

%% Create a dedication in italics with no heading, centered vertically
%% on the page.
\begin{dedication}
To my parents for always supporting me.
\end{dedication}

\chapter*{Acknowledgments}
I would like to thank my co-advisors Caroline Riedl and Matthias Grosse
Perdekamp for giving me the opportunity to learn about new and interesting
physical phenomena and as well for giving me the opportunity to learn about
different cultures.  I would like to thank Vincent Andrieux for many interesting
discussions both about physics and life in France.

I would like to think the funding agencies for make this research possible.
This work was supported by the Nation Science Foundation through the NSF PRAC
award \#1713684 and the NSF grant \#1506416.  This research was also supported
by the Blue Waters sustained-petascale computing project, which is supported by
the National Science Foundation (awards OCI-0725070 and ACI-1238993) and the
state of Illinois. Blue Waters is a joint effort of the University of Illinois
at Urbana-Champaign and its National Center for Supercomputing Applications.

Lastly I would like to acknowledge my cat Fox Trot for being both a source
stress and an anxiety reliever.
\begin{figure}[h!t]
  \centering
  \includegraphics[width=0.6\textwidth,trim=2cm 4cm 2cm 4cm,clip]{fox}
  \caption{Fox Trot}
  \label{fig::fox}
\end{figure}


% *********************** Adding TOC and List of Figures ***********************
\tableofcontents
\listoftables
\listoffigures


%\clearpage
\mainmatter
\renewcommand{\thepage}{\arabic{page}}
% ******************************** Main Matter *********************************
\ifpdf
\graphicspath{ {Chapters/Introduction/Figs/} }
\section{Theoretical Motivation}
\subsection{Proton Structure}
%\begin{frame}[label=current]

\begin{frame}
  \frametitle{Proton Structure}

  \begin{equation*}
     \sigma
  \end{equation*}
  
  \begin{columns}
    \column{0.4\textwidth}
    \begin{figure}
      \centering
      \includegraphics[width=0.7\textwidth, trim=4cm 9cm 4cm 9cm, clip]
      {DISxSect}
    \end{figure}
    \column{0.6\textwidth}
    \begin{figure}
      \centering
      \includegraphics[width=0.7\textwidth, trim=7cm 10cm 7cm 10cm, clip]{Proton}
    \end{figure}
  \end{columns}

  \begin{itemize}
  \item Protons are made up of:
    \begin{itemize}
    \item up/up/down valence quarks (describe proton quantum numbers)
    \item sea quarks and gluons
    \end{itemize}
  \end{itemize}
  
\end{frame}


\begin{frame}
  \frametitle{Longitudinal Proton Spin Structure}

  \begin{columns}
    \column{0.47\textwidth}
    \setlength\abovecaptionskip{-5pt}
    \setlength{\belowcaptionskip}{-1pt}
    \begin{figure}
      \vspace*{-0.3cm}
      \centering
      \includegraphics[width=\textwidth]{Helicity}
      \caption{Proton helicity distribution}
    \end{figure}
    \column{0.47\textwidth}
    \begin{dmath*}
      \Delta q = q^+ - q^- \\ \text{aligned quarks - anti-aligned quarks}
    \end{dmath*}
  \end{columns}
  
  \begin{itemize}
  \item Helicity distributions determined from $l^{\uparrow}+p^{\uparrow}
    \rightarrow l' + \pi + X$
  \item Integration gives the spin contribution from each quark flavor
  \end{itemize}
\end{frame}


\begin{frame}
  \frametitle{Transverse Proton Spin Structure}

  \begin{itemize}
  \item In a quark collinear approximation the quark transverse momentum should be small
    \begin{itemize}
    \item [] $\Rightarrow$ Analyzing power, A$_N$ = $\frac{1}{P}\frac{\sigma_{Left}^{\pi} - \sigma_{Right}^{\pi}}{\sigma_{Left}^{\pi} + \sigma_{Right}^{\pi}} \sim 10^{-4}$
    \end{itemize}
  \item E704 ($p^{\uparrow}+p\rightarrow \pi+X$) found much greater asymmetries   \end{itemize}

  \begin{figure}
    \centering
    \includegraphics[width=\textwidth]{AN}
  \end{figure}
  \begin{itemize}
  \item A$_N$ is found large independent of the center of momentum energy
  \end{itemize}
\end{frame}


\begin{frame}
  \frametitle{Sivers Effect}

  \begin{itemize}
  \item One possible way to explain large single spin asymmetries
  \item Gives a correlation between proton transverse spin and transverse momentum of a parton
  \end{itemize}

  \setlength\abovecaptionskip{-5pt}
  \setlength{\belowcaptionskip}{-10pt}
  \begin{figure}
    \vspace*{-0.3cm}
    \centering
    \includegraphics[width=0.7\textwidth]{SiversEffect}
    \caption{Lattice calculations of the quark transverse momentum in a polarized proton.}
  \end{figure}

  \begin{columns}
    \column{0.8\textwidth}
    \begin{itemize}
    \item Surprising result from theory: Sivers function is non-universal
    \item Expected to flip signs between
      \begin{itemize}
      \item [] $l+p^{\uparrow} \rightarrow l'+\pi+X$ and $h+p^{\uparrow} \rightarrow l+\bar{l}+X$
      \end{itemize}
    \end{itemize}
    \column{0.2\textwidth}
    \pause
    \includegraphics[width=\textwidth]{NSFmilestone}
  \end{columns}
\end{frame}


\begin{frame}
  \frametitle{Sivers Measurement}

  \begin{itemize}
  \item COMPASS and HERMES measured a non-zero Sivers amplitude from
    semi-inclusive deep inelastic scattering (SIDIS)
    \begin{itemize}
      \item [] $l+p^{\uparrow} \rightarrow l'+h+X$
    \end{itemize}
  \end{itemize}

  \begin{figure}
    \centering
    \includegraphics[width=0.95\textwidth]{SIDIS_siv}
    \caption{Sivers Amplitude related to the Sivers function}
  \end{figure}
\end{frame}



\chapter{Theoretical and Experimental Overview} \label{ch::theory_exp}
\ifpdf
\graphicspath{{Chapters/Theory/Figs/Raster/}{Chapters/Theory/Figs/PDF/}{Chapters/Theory/Figs/}}
\else \graphicspath{{Chapters/Theory/Figs/Vector/}{Chapters/Theory/Figs/}}
\fi

\section{Deep Inelastic Scattering}
To understand the structure of the nucleon it is useful to first introduce the
original process which described the nucleon as having a sub-structure.  This
process is the Deep Inelastic Scattering (DIS) process where a lepton impinges
on a nucleon denoted as

\begin{equation}
l(\ell) + N(P) \rightarrow l(\ell') + X(P_X),
\end{equation}
\noindent
where $l$ denotes a lepton, $N$ denotes a nucleon, $X$ represents all products
not detected and $\ell$, $\ell'$, $P$ and $P_X$ are the four momentum for their
respective lepton or nucleon.  This process is an electromagnetic reaction where
a the lepton is scattered via virtual photon exchange with the nucleon.  The
leading order Feynman diagram for this reaction is shown in
Fig.~\ref{fig::DIS_LO}.

\begin{figure}[h!t]
  \centering
  \includegraphics[width=0.5\textwidth, trim=3cm 9cm 3cm 9cm, clip]{DIS_LO}
  \caption{The leading order Feynman diagram for deep inelastic scattering}
  \label{fig::DIS_LO}
\end{figure}

DIS is traditionally studied with a high energy lepton beam and a fixed nuclear
target.  The initial state kinematics are described by

\begin{equation}
  s = (\ell+P)^2 \quad \mathrm{or} \quad E,
\end{equation}
\noindent
where $s$ is the center of mass energy and $E$ is the energy of the lepton beam.
The detected reaction kinematics in the lab frame are described by

\begin{multicols}{2}
  \noindent
  \begin{equation}
    \label{equ::DIS_Q2}
    Q^2 = -q^2 = -(\ell - \ell')^2 \approx EE'(1-\cos\theta )
  \end{equation}
  \begin{equation}
      \label{equ::BjorkenX}
      x = \frac{Q^2}{2P \cdot q} = \frac{Q^2}{2M\nu}
  \end{equation}
\end{multicols}

\begin{multicols}{2}
  \noindent
  \begin{equation}
    \nu = E - E'
  \end{equation} 
  \begin{equation}
    \label{equ::inelasticity}
    y = \frac{P \cdot q}{P \cdot \ell} = \frac{E - E'}{E} = \frac{\nu}{E}
  \end{equation} 
\end{multicols}

\begin{multicols}{2}
  \noindent
  \begin{equation}
    \label{equ::DIS_W}
    W^2 = (P+q)^2
  \end{equation}
\end{multicols}

\noindent
where $q$ is the virtual photon four momentum, $E'$ is the scattered leptons
energy, $x$ is Bjorken x, $\nu$ is the change in energy of the scattered lepton,
$y$ is the inelasticity and $W^2$ is invariant mass of hadron final state.  In
the last relation from Eq.~\ref{equ::DIS_Q2}, $\theta$ is the scattering angle
of the lepton with respect to the beam and the approximation is only true when
the lepton mass is assumed to be zero.  In Eq.~\ref{equ::BjorkenX}, $M$ is the
nucleon mass.  In the parton model, section~\ref{sec::parton_model}, $x$ has the
interpretation as being the momentum fraction of the struck parton with respect
to its parent hadron and therefore $x$ ranges between 0 and 1.  The
inelasticity, $y$, measure the proportional lepton energy reduction and
therefore takes on a value between 0 and 1.

The process is called deep if $Q^2 >> M^2$ and inelastic if $y < 1$.  For
practical purposes in experiments, the deep inelastic criteria corresponds to a
$Q^2 > 1~GeV$ and $W^2 > M^2$.  As can be seen in
Eq.~[\ref{equ::DIS_Q2}-\ref{equ::DIS_W}], not all the variables are independent.
DIS is described by two independent variables usually given by ($x$, $Q^2$) or
($x$, $y$).  For reference, in the limit as $y \to \; 1$ the process becomes
elastic scattering and can then be described by only one independent variable.

The cross-section for DIS is defined as~\cite{Barone:2001sp}
\begin{equation}
  \label{equ::DIS_xsection}
  \mathrm{d}\sigma =
  \frac{1}{4P\cdot \ell}\frac{e^4}{Q^4} L_{\mu\nu}W^{\mu\nu}
  2\pi\frac{\mathrm{d}^3\ell'}{(2\pi)^32E'}
\end{equation}
\noindent
where $L_{\mu\nu}$ is the leptonic tensor and $W^{\mu\nu}$ is the hadronic
tensor.  The leptonic tensor describes free leptons and can therefore be
calculated in perturbation theory.  It can be decomposed into a systematic
spin-independent tensor and an anti-symmetric spin-dependent tensor.  Summing
over all the possible spins of the lepton beam, the leptonic tensor is

\begin{equation}
  L_{\mu\nu} = 2\Big(\ell_{\mu}\ell'_{\nu} + \ell_{\nu}\ell'_{\mu} -
  g_{\mu\nu}\ell \cdot \ell' \Big) +
  2m\epsilon_{\mu\nu\rho\sigma}s^{\rho}q^{\sigma}
\end{equation}
\noindent
where $m$ is the lepton mass and $s^{\rho}$ is the spin four vector of the
lepton.

Generically the hadronic tensor is defined as
\begin{equation}
  W^{\mu\nu} = \frac{1}{2\pi}
  \int \mathrm{d}^4\xi e^{iq \cdot \xi}
  \langle PS | J^{\mu}(\xi)J^{\nu}(0) | PS \rangle
\end{equation}
\noindent
where $J$ is an electromagnetic current and $|PS \rangle$ represents the nucleon
with momentum $P$ and spin $S$.  The hadronic tensor describes a hadron bound
together by quantum chromo-dynamics (QCD).  As of yet there is no known
technique for calculating the hadronic tensor in a perturbation theory or
otherwise.  Instead the hadronic tensor can be written in the most general
Lorentz invariant form using structure functions to parameterize the
non-perturbative nature of the tensor.  With the use of these structure
functions, the differential DIS cross-section can be written
\begin{equation}
  \label{equ::DIS_diffxsection}
  \frac{\mathrm{d}\sigma}{\mathrm{d}x\mathrm{d}y} =
  \frac{8\pi\alpha^2ME}{Q^4}
  \Big\{
  xy^2F_1(x, Q^2) + \Big(1-y\Big)\frac{F_2(x, Q^2)}{x}
  + c_1(y, \frac{Q^2}{\nu}) g_1(x, Q^2) + c_2(y, \frac{Q^2}{\nu}) g_2(x, Q^2)
  \Big \}
\end{equation}
\noindent
where $\alpha$ is the electromagnetic coupling constant; $F_1$, $F_2$, $g_1$,
$g_2$ are structure functions; and $c_1$ and $c_2$ are functions which depend on
the polarization of the target.  The SLAC collaboration measured the structure
functions, $F_1$ and $F_2$, and found mild variations as a function
$Q^2$~\cite{Bloom:1969kc,Breidenbach:1969kd}.  This phenomenon now known as
Bjorken scaling lead to the theory of the parton model~\cite{Bjorken:1969ja}.
Fig.~\ref{fig::F2} shows the $F_2$ structure function which is approximately
constant as a function of $Q^2$.

\begin{figure}[h!t]
  \centering
  \includegraphics[width=0.5\textwidth, trim=3cm 4cm 3cm 4cm, clip]{F2}
  \caption{The $F_2$ structure function measured by several experiments.  Note
    that the data is shifted up by a factor $2^{i_x}$ to see the $x$ dependence.
    Image taken from~\cite{Tanabashi:2018oca}}
  \label{fig::F2}
\end{figure}


\section{The Parton Model} \label{sec::parton_model}
The parton model is described in what is called an infinite momentum frame where
the nucleon is moving which large momentum.  In the parton model the nucleon, in
high energy scattering processes, is considered to be composed of point like
constituent mass-less particles called partons.  At high energy scattering the
QCD strong force binding the partons becomes asymptotic small and therefore the
partons appear to be free.  The cross-section in DIS can then be described as a
lepton scattering incoherently off a free parton in the nucleon.  In the parton
model the hadron tensor for scattering off a quark can be written
as~\cite{Barone:2001sp}

\begin{dmath}
  W^{\mu\nu} = \frac{1}{2\pi} \sum_q e_q^2 \sum_X
  \int \frac{\mathrm{d}^3 P_X}{(2\pi)^32E_X}
  \int \frac{\mathrm{d}^4k}{(2\pi)^4}
  \int \frac{\mathrm{d}^4k'}{(2\pi)^4} \delta(k^{\prime2}) \times
       [\bar{u}(k')\gamma^{\mu}\langle X | u(k) | PS \rangle]*
       [\bar{u}(k')\gamma^{\nu}\langle X | u(k) | PS \rangle]
       \times (2\pi)^4\delta^4(P-k-P_X)(2\pi)^4\delta^4(k+q-k'),
\end{dmath}
\noindent
where $e_q$ is the electric charge of quark flavor $q$; and $u$ and $\bar{u}$
are free Dirac spinors.  This hadronic tensor can be simplified by introducing
the quark-quark correlation matrix as

\begin{equation}
  \Theta_{ij}(k, P, S) =
  \sum_X \int \frac{\mathrm{d}^3 P_X}{(2\pi)^32E_X}(2\pi)^4\delta^4(P-k-P_X)
  \times \langle PS | \phi_j(0) | X \rangle \langle X | \phi_i(0) | PS \rangle,
\end{equation}
\noindent
where $\phi(\xi) = e^{-ip \cdot \xi}u(p)$ is a quark field.  Using the
quark-quark correlation matrix, the hadronic tensor can be written as

\begin{equation}
  W^{\mu\nu} = \sum_q e_q^2 \int \frac{\mathrm{d}^4k}{(2\pi)^4}
  \int \frac{\mathrm{d}^4k'}{(2\pi)^4} \delta(k^{\prime2})
  (2\pi)^4\delta^4(k+q-k')\times \mathrm{Tr}
  [ \Theta \gamma^{\mu}\slashed{k'}\gamma^{\nu} ].
\end{equation}
\noindent
In the cases of unpolarized or longitudinally polarized DIS the lead order
contributing terms from the quark-quark correlator
are~\cite{Mulders:1995dh,Boer:1997nt,Bacchetta:2006tn}

\begin{equation}
  \label{equ::simpleQQcorr}
  \Theta = \frac{1}{2}
  \Big(
  f_1(x)\slashed{P} +
  g_{1L}(x)\lambda\gamma_5\slashed{P}
  \Big)
\end{equation}
\noindent
where $\lambda$ is the longitudinal polarization of the hadron.  The hadronic
tensor simplifies to a symmetric contribution and an anti-symmetric
contribution~\cite{Barone:2001sp}

\begin{equation}
  \label{equ::simpleHadronTensor}
  W^{\mathrm{symmetric}}_{\mu\nu} = \frac{1}{P\cdot q} \sum_q e_q^2
  \Big( (k_{\mu}+q_{\mu})P_{\nu} + (k_{\nu}+q_{\nu})P_{\mu}-g_{\mu\nu}
  \Big) f_1^q(x),
\end{equation}
\begin{equation}
  W^{\mathrm{anti-symmetric}}_{\mu\nu} =
  \lambda\epsilon_{\mu\nu\rho\sigma}(k_{\nu}+q_{\nu})P^{\rho}\sum_q e^2_q
  g^q_{1L}(x),
\end{equation}
\noindent
where in Eq.~\ref{equ::simpleQQcorr} $f_1$ and $g_1$ are two parton distribution
functions (PDFs).  $f_1$ is interpreted as the quark number density and $g_{1L}$
is interpreted as the total quark helicity distribution in a hadron.  $f_1$ is
then refers to the density of unpolarized quarks in a hadron and $g_{1L}$ refers
to the density of quarks longitudinally polarized in the same longitudinal
direction as the hadron.  To make this explicit, $f_1$ and $g_{1L}$ can be
written

\begin{multicols}{2}
  \noindent
  \begin{equation}
    f_1 = f_1^{+} + f_1^{-},
  \end{equation}
  \begin{equation}
    g_{1L} = f_1^{+} - f_1^{-},
  \end{equation}
\end{multicols}
\noindent
where + and - denote the helicity.  To be clear the parton distribution $g_{1L}$
is not the same as the structure function $g_1$.

The unpolarized quark number density, $f_1$, has been extracted from global
analysis of several experiments~\cite{Rojo_2015}.  Fig.~\ref{fig::NNPDF_10GeV}
shows the current $xf_1$ values and confidence intervals for different quarks
and gluons specifically in the proton.

\begin{figure}[h!t]
  \centering \includegraphics[width=0.5\textwidth,trim=6cm 9cm 6cm 8cm,
    clip]{NNPDF_10GeV}
  \caption{The unpolarized parton distribution functions times the momentum
    fraction.  The different color correspond to different quarks or gluons.
    Image taken from~\cite{Tanabashi:2018oca}}
  \label{fig::NNPDF_10GeV}
\end{figure}

The longitudinal spin structure, $g_{1L}$ has also been measured at SMC, HERMES,
and COMPASS~\cite{Adeva:1997is,PhysRevLett.92.012005,Savin:2011zz}.  The global
analysis fit is shown in Fig.~\ref{fig::Proton_g1L} using the parameterizations
from NNPDF2014, AAC2008, DSSV2008 and
LSS2010~\cite{Harland-Lang:2016yfn,Abt:2016vjh,Nocera:2014gqa,Hirai:2008aj}.

\begin{figure}[h!t]
  \centering \includegraphics[width=0.5\textwidth,trim=1cm 8cm 1cm 8cm,
    clip]{Proton_g1L}
  \caption{The longitudinally polarized parton distribution functions times the
    momentum fraction for the u-quark (top) and the d-quark (bottom).  Image
    taken from~\cite{Tanabashi:2018oca}}
  \label{fig::Proton_g1L}
\end{figure}

In the parton model the structure function $F_1$ and $F_2$ are related to each
other and to the unpolarized quark number as
\begin{equation}
  F_2(x) = 2xF_1(x) = \sum_q e_q^2x\Big(f^q_1 + f^{\bar{q}}_1 \Big)
\end{equation}
which is known as the Callan-Gross relation~\cite{PhysRevLett.22.156}.  As well
the structure function $g_1$ is related to the helicity distribution, $g_{1L}$,
as
\begin{equation}
  g_1(x) = \frac{1}{2} \sum_q e^2_q g_{1L}(x).
\end{equation}


\section{Transverse Momentum Dependence}
The transverse momentum of the partons is integrated over when measuring the DIS
process.  This is because only the scattered lepton is measured and any
transverse parton motion cannot be measured.  The Drell-Yan
process~\ref{sec::DY} and the SIDIS process~\ref{sec::SIDIS} however are
sensitive to the internal transverse momentum of the partons.  When including
the transverse parton momentum, the most generic leading order quark-quark
correlator can be written~\cite{Mulders:1995dh,Boer:1997nt,Bacchetta:2006tn}

\begin{dmath}
  \label{equ::GeneralQQcorrelator}
  \Theta = \frac{1}{2}\Big[ f_1(x,k_{\bot})\slashed{P} +
    \frac{1}{M}h_1^{\bot}(x,k_{\bot})\sigma^{\mu\nu}k_{\mu}P_{\nu} +
    g_{1L}(x,k_{\bot})\lambda\gamma_5\slashed{P} +
    \frac{1}{M}g_{1T}(x,k_{\bot})\gamma_5\slashed{P}(k_{\bot} \cdot S_{\bot}) +
    \frac{1}{M}h_{1L}(x,k_{\bot})\lambda
    i\sigma_{\mu\nu}\gamma_5P^{\mu}k_{\bot}^{\nu} +
    h_1(x,k_{\bot})i\sigma_{\mu\nu}\gamma_5P^{\mu}S_{\bot}^{\nu} +
    \frac{1}{M^2}h_{1T}^{\bot}(x,k_{\bot})i\sigma_{\mu\nu}\gamma_5P^{\mu} \Big(
    k_{\bot} \cdot S_{\bot}k_{\bot}^{\nu} - \frac{1}{2}k_{\bot}^2S_{\bot}^{\nu}
    \Big ) +
    \frac{1}{M}f_{1T}^{\bot}(x,k_{\bot})\epsilon^{\mu\nu\rho\sigma}\gamma_{\mu}P_{\nu}k_{\rho}S_{\sigma}
    \Big],
\end{dmath}
\noindent
where $k_{\bot}$ denotes the transverse parton momentum and $S_{\bot}$ denotes
the transverse hadron spin.  Eq.~\ref{equ::GeneralQQcorrelator} includes eight
transverse momentum dependent (TMD) PDFs which are functions of $x$ and
$k_{\bot}$.  The notation used to depict the TMDs functions is the so-called
Amsterdam notation.  The letters represent the different quark polarizations
where $f, g, h$ stand for unpolarized, longitudinally polarized and transversely
polarized respectively.  The subscript 1 denotes leading order and the
subscripts $T$ and $L$ denote a transversely polarized hadron and a
longitudinally polarized hadron respectively.  Finally the superscript $\bot$
denotes that the distribution is an odd function of $k_{\bot}$ and therefore is
zero when integrated over the parton transverse momentum.  Fig.~\ref{fig::TMDs}
organizes the TMDs by nucleon and quark polarizations and gives a visual of each
TMD's interpretation.

\begin{figure}[h!t]
  \centering
  \includegraphics[width=0.6\textwidth, trim=2cm 2cm 2cm 2cm,clip]{TMDs}
  \caption{The eight TMDs needed to describe a spin 1/2 nucleon at leading
    order.  The columns represent the different nucleon polarization and the
    rows represent the different quark polarizations.  The individual figures
    give a visual of the TMD's interpretation.}
  \label{fig::TMDs}
\end{figure}

\subsection{Sivers Distribution}
The Sivers TMD was first purposed to explain large nucleon spin-dependent
asymmetries~\cite{Sivers}.  The interpretation of the Sivers TMD, {\siv}, is
that it gives a correlation between transverse spin of the parent hadron and
transverse momentum of parton.  When viewing the hadron in the direction of it's
momentum, if {\siv} is positive then it is expected that there are more partons
with momentum going left than going right.  Intuitively a non-zero {\siv} would
then imply that the bound quarks carry orbital angular momentum.  As of yet
however, there is no theoretical link between orbital angular momentum and the
Sivers function.

The Sivers function is odd under time reversal.  As a result of this it was
originally believe to be a forbidden correlation.  However it was shown that the
Sivers function could be non-zero from gluon exchange during the initial state
in the Drell-Yan process and during the final state in
SIDIS~\cite{Brodsky:2002cx,Brodsky:2002rv}.  Surprising it was shown that a
non-zeron Sivers function is expected to have opposite sign in SIDIS and
Drell-Yan~\cite{collins_2002}.  That is

\begin{equation}
  f_{1T}^{\bot} |_{Drell-Yan} = - f_{1T}^{\bot} |_{SIDIS}.
\end{equation}

give sivers SIDIS results from HERMES/COMPASS.  Give interpretation

\section{Drell-Yan} \label{sec::DY}

\section{Semi-Inclusive Deep Inelastic Scattering} \label{sec::SIDIS}

\begin{dmath}
  W_{\mu\nu} =
  \Big( \frac{q_{\mu}q_{\nu}}{q^2}-g_{\mu\nu} \Big) \frac{F_1(x, Q^2)}{M} +
  \Big(P_{\mu} - \frac{P \cdot q}{q^2}q_{\mu} \Big)
  \Big(P_{\nu} - \frac{P \cdot q}{q^2}q_{\nu} \Big) \frac{F_2(x, Q^2)}{M^2\nu} +
  i\epsilon_{\mu\nu\rho\sigma}\frac{q^{\rho}}{P \cdot q}
  \Big\{ S^{\sigma}g_1(x, Q^2) +
  \Big( S^{\sigma} - \frac{S\cdot q}{P\cdot q}P^{\sigma}\Big)g_2(x,Q^2) \Big\},
\end{dmath}
\noindent
where $F_1$, $F_2$, $g_1$ and $g_2$ are structure functions which are determined
from experiments.


%Chapter 1 
\chapter{The COMPASS Experiment} 
\label{Chap::compass}
\ifpdf
\graphicspath{{Chapters/COMPASS/Figs/Raster/}{Chapters/COMPASS/Figs/PDF/}{Chapters/COMPASS/Figs/}}
\else \graphicspath{{Chapters/COMPASS/Figs/Vector/}{Chapters/COMPASS/Figs/}} \fi

The COmmon Muon Proton Apparatus for Structure and Spectroscopy (COMPASS)
experiment is a fixed target experiment at located on the French side at CERN.
COMPASS started taking data in 2002 in the same hall as earlier Euopean Muon
Collaboration (EMC), New Muon Collaboration (NMC) and Spin Muon Collaboration
(SMC) experiments.  COMPASS has studied hadron structure through (SI)DIS,
Drell-Yan and Primakoff reactions and has done hadron spectroscopy measurements.
\par

The COMPASS spectrometer is a two-stage spectrometer.  The two stages are in a
series and each stage contains various tracking detectors and as well at the end
of each stage there is a muon wall filter for distinguishing between muons and
other particles.  Both stages also contain an electromagnetic and hadron
calorimeter.  Each stage is centerend around a strong spectrometer magnet used
for determining particle momentum.  The first stage downstream of the target is
the large angle spectrometer (LAS) and it is centered around the SM1 magnet
which has an integrated field of 1~Tm.  This stage detects tracks with larger
polar scattering angles roughly between 26 mrad and 160 mrad.  The second stage
is the small angle spectrometer (SAS) and it detects particle tracks having a
scattering angle between roughly 8 mrad and 45 mrad.  This stage is centered
around the SM2 magnetic which has an integrated field of 4.4~Tm.  A graphic of
the 2015 setup is shown in Fig~\ref{fig::compassSpec}.\par

This chapter gives an overview of the COMPASS data taking setup with specific
interest on the 2015 setup from which the data in this thesis was produced from.
For a more thorough review of the spectrometer see
reference~\cite{compassSpec}.  This chapter is roughly organized by how the
data taking occurs.

\section{The Beam}

\section{The Target}

\section{Tracking Detectors}

\section{Particle Identification}

\section{Trigger}

\section{Data Acquisition}

\section{Data Production}

\section{2015 Drell-Yan Setup}


This section of the spectrometer covers high
$Q^2$ and high $x_b$.  The target-pointing trigger system in LAS consists of two
hodoscopes, one down stream of the last tracker in LAS and the other just after
an iron muon filter in front of the second spectrometer magnet.

The trigger in SAS is
similar to the LAS trigger but for muons of a lower $Q^2$ value.  The tracking
detectors used in both spectrometers are gaseous detectors (drift chambers,
straws, multi-wire proportional detectors), micro-megas, and gas electron
multiplier (GEM) detectors.  The two stages are separated by two spectrometer
magnets: SM1, with a magnetic field integral of 1~Tm and SM2, with a magnetic
field integral of 4.4~Tm~\cite{compassSpec}.

Both SM1 and SM2 have magnetic
fields in the vertical direction meaning charged particles are deflected in the
x-z plane and can therefore have their momentum determined.  For beam
reconstruction there is a beam telescope upstream of the target consisting of
eight planes of scifi detectors.  To account for the transverse magnetic field
in the polarized target a chicane magnet system was added in the beam line.
This meant that the beam entered at a slight angle in the beam telescope but
exited the target going straight. \par

\section{The Beam old}
The Super Proton Synchrotron (SPS) is the second largest accelerator
at CERN.  It has a circumference of almost 7 km and it can accelerate
protons up to an energy of 450~GeV.  The SPS extracts beam to the
Large Hadron Collier and as well sends beam to various experiments in
the North Area at CERN.  COMPASS is one of these North Area
experiments, receiving beam tangent from the SPS on the M2 beam line.
In 2015 the SPS was delivering to COMPASS around $100 \times 10^{11}$
protons over a 4.9 second spill length and a spill was sent
approximately twice every 32 seconds.  The proton beam extracted from
the SPS then would collide with a primary 500~mm long, beryllium
target and from there a secondary hadron beam consisting of 97\% $\pi
^-$, 2\% $\bar{p}$ and 1\% $K^-$ was captured with magneto-optics into
a beamline leading to the COMPASS spectrometer.  The flux of the
secondary hadron beam averaged $0.6 \times 10^8
\frac{\mathrm{hadrons}}{\mathrm{sec}}$ and its momentum was
190~$\frac{\mathrm{GeV}}{\mathrm{c}}$~\cite{compassDYpaper}.



\section{Experimental Setup}

%\begin{figure}
%  \centering \includegraphics[width=0.3\textwidth]{Absorber}
%  \caption{The absorber at COMPASS was made of Alumina with a tungsten
%    plug and it was placed just downstream of the target cells.}
%  \label{fig::Absorber}
%\end{figure}





%\begin{figure}
%  \centering
%  %\includegraphics[width=\textwidth]{CompassSpectrometer}
%  \includegraphics[width=0.85\textwidth]{CompassSpectrometer}
%  \caption{The COMPASS spectrometer.}
%  \label{fig::COMPASS_spec}
%\end{figure}



The target material used was ammonia ($\mathrm{NH}_3$), where the proton from
the hydrogen nucleus was the polarizable nucleon.  The average polarization
throughout the data taking was 0.73 and a 0.6~T dipole magnet was used to
maintain target polarization.  With this dipole field and polarization, the spin
relaxation time for the target was approximately 1000 hours.  The target was
separated into two 55~cm cells of radius 2~cm which were in turn separated by
20~cm and oppositely polarized, as shown in figure~\ref{fig::Absorber}.  In 2015,
COMPASS took nine data periods labeled W07-W15.  Each data period lasted two
weeks and the spin orientation of the targets was reversed after the first week
of every period to reduce systematic effects arising from different geometric
acceptances and luminosities of up and downstream target cells.
%Most notably the spin
%orientation was reversed to have equal luminosity and to force equal
%geometrical acceptance between the two spin states. \par

%\begin{figure}[h]
%  \centering
%  \includegraphics[width=0.45\textwidth]{Absorber}
%  \caption{The absorber at COMPASS was made of Alumina with a tungsten
%    plug and it was placed just downstream of the target cells.}
%  \label{fig::Absorber}
%\end{figure}

For the 2015 Drell-Yan data taking a hadron absorber,
see figure~\ref{fig::Absorber}, was placed just downstream of the target
cells.  This was done to reduce the amount of hadrons and electrons
detected in the spectrometer and therefore ensured a cleaner di-muon
sample.  The absorber material was mostly alumina (Al$_2$O$_3$) and
concrete and the absorber corresponded to approximately 7.5
interactions lengths of material.  Inside the absorber was an aluminum
target followed by a tungsten plug, each of radius 2.5 cm, which acted
as a beam dump.  The aluminum target and tungsten plug served the
double purposes as absorbers and also as unpolarized nuclear targets.
In addition a thin $^6\mathrm{Li}$ absorber was added just downstream of the
primary absorber to absorb thermal neutrons produced in the primary
absorber.  This $^6\mathrm{Li}$ absorber was proposed to improve the
performance of the first tracking detector downstream of the
target. \par

\subsection{DAQ and Reconstruction}
The data acquisition system (DAQ) was recording events at a rate of
approximately 30~kHz with a dead time of 10\%.  In 2015, COMPASS
recorded approximately 750~terabytes of raw data from the nine,
two-week periods.  Raw data refers solely to individual detector
timing and wire or strip positions and does not correspond to physics
observables of interest.  From this raw information the CORAL
reconstruction software at COMPASS is able to determine the trajectory
and momentum of charged particles going through the COMPASS
spectrometer.

%This reconstruction stage reduces the data volume by approximately a factor of 10.

\chapter{Drift Chamber 05} \label{ch::DC05}
\ifpdf
\graphicspath{{Chapters/DC5/Figs/Raster/}{Chapters/DC5/Figs/PDF/}{Chapters/DC5/Figs/}}
\else \graphicspath{{Chapters/DC5/Figs/Vector/}{Chapters/DC5/Figs/}}
\fi

Drift Chamber 05 (DC05) is a large-area planar drift chamber 05.  It was
constructed in 2014 and 2015 at the University of Illinois and Old Dominion
University and was then shipped to CERN for final assemble.  DC5 was installed
to the large angle spectrometer of COMPASS during the spring of 2015.

The DC5 detector was an import tracking detector, successfully collected data
from 2015 through 2018 and will continue to be an important component for track
reconstruction in future measurements.  The author of this thesis helped with
the construction at Illinois and the assembly at CERN and as well for performing
calibrations and maintaining DC5.

\section{Motivation for Drift Chamber 05}
Simulations of the COMPASS spectrometer for Drell-Yan measurements determined
that 96\% of all events include a track in the large angle
spectrometer~\cite{proposal}.  For this reason the Drell-Yan trigger system was
setup only to record events with at least one track in LAS.  As DC5 was
installed in LAS it is therefore very important in track reconstruction for
Drell-Yan measurements.  Additional simulations with the Drell-Yan setup and
this corresponding COMPASS Drell-Yan trigger showed that the global
reconstruction efficiency drops below 30\% without DC5 and half of another large
area tracker in LAS~\cite{quintans_rec_march12}.  That is to say the
spectrometer reconstruction efficiency is very poor without DC05 and half of an
unstable detector.  Fig.~\ref{fig::specRecEff} shows the nominal global
reconstruction efficiency and the worst case scenario for Drell-Yan
measurements.  For these reasons it was very important to have DC05 installed
and working reliably.

\begin{figure}[h!t]
  \centering
  \begin{subfigure}{.5\textwidth}
    \centering \includegraphics[width=\linewidth, trim=5cm 8cm 5cm 8cm,
      clip]{standardRec}
    \caption{Global reconstruction efficiency with all detectors working.  This
      image was taken from~\cite{quintans_rec_march28}}
    \label{fig::standardRec}
  \end{subfigure}%
  \begin{subfigure}{.5\textwidth}
    \centering
    \includegraphics[width=\linewidth, trim=5cm 8cm 5cm 8cm, clip]{withLARec}
    \caption{Global reconstruction efficiency without DC05 and half of another
      LAS large area tracker.  This image was taken
      from~\cite{quintans_rec_march28}}
    \label{fig::withLARec}
  \end{subfigure}
  \label{fig::specRecEff}
\end{figure}

\section{Preparation for DC05}
To prepare for constructing DC05, the detector response was simulated using
Garfield~\cite{garfield}.

In preparation for the construction of DC05 the detector was simulated and two
prototypes were built.  The simulations were performed using
Garfield~\cite{garfield} and the purpose of these simulations was to determine
an operating threshold capable of achieving a 200 $\mu$m position resolution.
With this position resolution goal in mind, Garfield simulated the electric
potential field in each drift cell and as well the arrival times for ionized
electrons as a function of the number of primary ionized electrons for
detection.  The variance of the electron arrival times was then used to
determine the position resolution as a function of the number of primary
electrons detected.  The results of the simulation are shown in figure
\ref{fig:PosRes}.  The conclusion from the simulations was that the threshold
should be tuned to detect the amplification of the 5th primary electron
corresponding to approximately a 4 fC threshold.  This was accordingly one of
the main design goals for the front-end electronics. \par

%\begin{figure}
%  \centering
%  \includegraphics[width=0.5\textwidth]{GarPosResolution.png}
%  \caption{}{Primary electron number detected versus the position resolution.
%    The detector should be sensitive to detect the fifth primary electron to
%    achieve the desired position resolution. }
%  \label{fig:PosRes}%
%\end{figure}

In addition two prototypes were constructed for training and testing.  To start
a clean room was built at the Nuclear Physics Lab (NPL) at UIUC and this is
where the two prototypes were built as well as where much of the actual detector
was built.  The first prototype was named prototype A and consisted of 1 plane,
eight sense wires, 9 field wires and was built to a length of 50 cm.  Protoype B
was the second detector constructed and consisted of two planes with 16 sense
wires per plane and a length of 163 cm.  Prototype A had the opportunity to be
tested with beam at DESY and was shown to achieve 200 $\mu$m
resolution~\cite{choi}.  Both of these prototypes were built using similar
materials and construction techniques as the full size detector.  In particular
these two prototypes were the needed experience for working with sense wires
having a diameter of 20 $\mu$m.
\section{Design}

%\begin{figure}
%  \centeringq \includegraphics[width=0.5\textwidth]{DC05_full.png}
%  \caption{}{The completed DC05 being craned into the COMPASS large area
%    spectrometer.}
%  \label{fig:DC05}%
%\end{figure}

The design of DC05, figure~\ref{fig:DC05}, was based off a previous large-area
tracker at COMPASS.  DC05 has an active area of 249x209cm$^2$ and consists of
eight planes with a total of 2304 sense wires and 2312 field wires.  The eight
planes of DC05 correspond to four views, where each view measures a coordinate.
The coordinates measured from DC05 are the horizontal, vertical and $\pm$
10$^{\circ}$ with respect to the horizontal.  The horizontal and vertical
coordinates consist of 2x256 sense wires and the offset to horizontal
coordinates each consist of 2x320 wires for increased acceptance.  Each plane
was made from a G-10 frame and five frames stacked together constituting a view.
The whole detector was closed in with two precision, stainless steel stiffening
frames, which were assembled with aluminized mylar as a gas window. \par The
views of DC05 consisted of three cathode layers and two anode layers.  The
cathodes layers were made from carbon paint sprayed on a 25 $\mu$m thin mylar
layer.  There were two single-layer cathodes layers and one layer with carbon on
two sides within each view.  Additionally a 30 cm circular so-called beam killer
was added to the cathodes to control the efficiency in the central part of the
detector.  The cathodes were nominally set to -1675 V and the beam killer
voltage was set to -900 V for zero efficiency in the high flux central region.
The voltage on the beam killer can however be raised above the amplification
threshold if the beam flux is reduced and it is desirable to study the central
region.  \par The anode layers were made from alternating 20 $\mu$m gold-plated
tungsten sense wires and 100 $\mu$m gold-plated copper beryllium field wires, as
shown in figure~\ref{fig:driftcell}.  The field wires were also placed at -1675
V and the sense wires were at 0 V.  The gas used was made of: 45\% argon, for
amplification; 45\% ethane, for quenching; and 10\% CF$_4$, to reduce aging
effects.  All these properties corresponded to a gain of approximately 10$^4$.

%\begin{figure}
%  \centering
%  \includegraphics[width=0.5\textwidth]{DriftCell.png}
%  \caption{}{The drift cell dimensions of one view in DC05.}
%  \label{fig:driftcell}%
%\end{figure}

\section{Construction}
The construction of DC05 was carried out as precisely as possible starting with
the precision from the stainless steel stiffening frames.  The stiffening frames
where cut with the highest relative accuracy by cutting the two frames on top of
each other to a precision of 50 $\mu$m everywhere in their plane.  The precision
from the stiffening frames was then transferred to the anode and cathode frames
through 40 positioning pins.  The G-10 frames were milled from strips at the NPL
using a precision milling machine.  Each four strips were then epoxied together
on top of one of the stiffening frames.  \par The cathodes had mylar stretched
and epoxied to them using a custom built stretching machine at CERN.  An
external company then spray painted carbon on them to make a resistance of
approximately 30 $k\Omega /m$.  All the sense and field wires were hand soldered
and sequently verified for position using a microscope.  It was estimated the
position placement of each sense wire was at least as good as half the diameter
of a sense wire or precise to 10 $\mu$m.  The final assemble was done at CERN.
This consisted of stacking each of the 21 G-10 frames on top of the stiffening
frame and attaching copper electronic shielding all along the exterior of the
detector to reduce electronic noise. \par There were various tests performed
throughout the construction process for quality assurance before the final
installation.  The starting tests were measuring thickness and position of
important cuts on the G-10 strips using a micrometer.  G-10 strip thicknesses
were iteratively milled until they reached better than 50 $\mu$m in thickness
accuracy.  The thickness deviation of the whole detector including the stainless
steel stiffening frames was better than 750 $\mu$m.  The mechanical tension of
the sense wires was tested for stability by ensuring the voltage difference
between sense and field wires could reach as high as 2400 V in air.  In addition
the wire tension was cross-checked by determining the resonance frequency with
which the wires vibrated.  The resonance frequency was determined by placing the
wires in a constant magnetic field and varying a sinusoidal current across each
wire till the wires vibrated maximally.  The leakage current between sense and
field wires was verified to be less than 100 nA at nominal voltage in air.
Finally amplification tests were first performed using a strontium-90 source and
verifying the counts per electronics board increased below the radioactive
source.

\section{2015 Performance}
The overall performance of DC05 was checked using the COMPASS reconstruction
software CORAL.  In all cases the view of study was excluded from the
reconstruction algorithm and the individual hit information for the view of
interest was saved to get an unbiased measurement.  The efficiency was found to
be between 85\% and 90\% depending on the plane.  Using the so called RT
relation, figure~\ref{fig:RT}, the location of a track within a drift cell can
be most accurately determined.  This RT relation needs to be tuned as a
calibration to minimize the track residuals.  The RT relation also varies
depending on the beam type, intensity and the trigger type.  \par The double
layer residual was used to determine the position resolution.  The double layer
residual is the radial difference between the expected x/y positions of the two
planes in a view.  This double layer residual is independent of the track
resolution and only depends on the variance addition of the two individual
planes.  For the 2015 Drell-Yan physics taking the resolution achieved was
approximately 430 $\mu$m. This was determined by fitting the double residual
with a Gaussian to extract the variance and assumed equal variance per plane in
a view.

%\begin{figure}
%  \centering \includegraphics[width=0.5\textwidth]{RT_DC05Y2.png}
%  \caption{}{Time versus position relation, or RT relation, after calibrating.
%    The red fit shows the calibration determined.}
%  \label{fig:RT}%
%\end{figure}

\section{Conclusion}
The large-area DC05 was constructed at Illinois, Old Dominion University and
CERN.  It was built to upgrade the spectrometer performance at COMPASS and over
its two year construction period, involved the colaboration from students,
technicians and professors from five continents.  DC05 is a stable detector that
has been in use for two years and will continue to collect useful data for
future physics data taking at COMPASS.

\ifpdf
\graphicspath{ {Chapters/Alignment/Figs/} }
\section{COMPASS Alignment}

\begin{frame}%[label=current]
  \frametitle{Alignment Motivation $\qquad \quad \color{red} \mathscr{RH} \text{
      contributions to analysis}$}

  \begin{itemize}
  \item COMPASS includes over 350 detector planes which need to be aligned
  \item Reconstructing particle tracks is not possible without
    alignment
  %\item Accomplished by minimizing a $\Chi^2$ function
  \item First check of spectrometer performance
  \end{itemize}

  \begin{columns}
    \column{0.48\textwidth}
    \begin{figure}
      \centering
      \includegraphics[width=\textwidth]{GMalign}
      \caption{Alignment effects, GEM type detector.}
    \end{figure}
    \column{0.5\textwidth}
    \begin{figure}
      \centering
      \includegraphics[width=\textwidth]{STalign}
      \caption{Calibration effects for straw type detector.}
    \end{figure}
  \end{columns}
\end{frame}


%\subsection{Alignment Data}
\begin{frame}
  \frametitle{Alignment Data}

  \begin{itemize}
  \item Low intensity alignment runs taken each data period
  \item $\mu^-$ beam used to have straight tracks
  \item Trigger system modified and detector centers turned on to
    maximally illuminate detectors
  \end{itemize}

  \begin{figure}
    \centering
    \includegraphics[width=\textwidth]{AlignmentRun}
    \caption{Detector data monitoring plots.}
  \end{figure}
\end{frame}


%\subsection{Alignment Procedure}
\begin{frame}%[label=current]
  \frametitle{Alignment Procedure Checks $\color{red} \mathscr{RH} \text{ contributions to analysis}$}
  
  \begin{itemize}
  \item Alignment performed by minimizing $\chi^2$ function of all
    track residuals
  \item Detectors aligned in 4 parameters:
    \begin{itemize}
    \item x position, y position, angle, wire/strip pitch
    \end{itemize}
    
  \item Alignment procedure iterated several times for best residuals
  \item 3 residual types checked to ensure alignment converged
  \end{itemize}

  \begin{figure}
    \centering
    \includegraphics[width=\textwidth]{AlignmentChecks}
    %\caption{}
  \end{figure}
\end{frame}

\chapter{Analysis of High Mass Drell-Yan Transverse Spin Phenomena}
\label{ch::hmanalysis}
\ifpdf
\graphicspath{{Chapters/HMAnalysis/Figs/}}
\fi

This chapter goes over the analysis techniques and results from the 2015
transversely polarized Drell-Yan data taking.  The chapter begins by describing
the data collection setup and the event selection criteria followed by the
analysis techniques used to determine asymmetry amplitudes.  The analysis
techniques described are: the standard transverse spin-dependent asymmetry (TSA)
analysis, Sec~\ref{sec::standTSA}, the double ratio analysis,
Sec~\ref{sec::doubleratio}, the $q_T$ weighted asymmetry analysis,
Sec~\ref{sec::qtweighted}, and finally the left-right asymmetry analysis,
Sec~\ref{sec::leftrightasym}.  All of these analyses are related in that they
measure TMD effects from the Drell-Yan process.  For this reason the event
selection and kinematical asymmetry binning described in the opening sections
will be the same for all analyses in this chapter unless stated otherwise.

\section{Data Sample} \label{sec::datasample}

\subsection{Data Collection} \label{sec::datacollection}

The data sample is from the 2015 COMPASS Drell-Yan measurement.  In this
measurement a 190~GeV/c $\pi^-$ beam impinged on a transversely polarized NH$_3$
target and two oppositely charged muons were detected in the spectrometer.
Fig.~\ref{fig::dy_2015_targ_setup} gives a visual of the basic setup and
chapter~\ref{ch::compass} goes over the spectrometer setup and beam in more
details.  The COMPASS spectrometer began taking commissioning data in April of
2015.  The data collected for this analysis, after the commissioning phase, is
from July 8 through November 12 of 2015.  After November 12, the SPS stopped
providing beam to COMPASS.  The data is split into 9 periods lasting
approximated 2 weeks each and labeled W07-W15.  During each data period the
spectrometer conditions were frozen so no detector changes could affect the
spectrometer acceptance.  Sec~\ref{sec::polTarget} describes the polarized
target in more details.  In summaray, the NH$_3$ target was split into two
oppositely polarized cells separated by 20~cm with one cell polarized vertically
up and one cell polarized vertically down in the lab frame.  Each data period is
split into two sub-periods to reduce systematic effects of acceptance and
luminosity dependencies.  The polarization of both cells was flipped between
sub-periods.  A summary of the analysis data taking from each period is shown in
Table~\ref{tab::datataking}.

\begin{figure}[h!t]
  \centering \includegraphics[width=0.6\textwidth, trim=4cm 7cm 4cm 7cm,
    clip]{dy_2015_targ_setup}
  \caption{Basic pictorial setup of the target region in 2015 COMPASS Drell-Yan
    data collection.}
  \label{fig::dy_2015_targ_setup}
\end{figure}


\begin{table}[h!t]
  \centering
  \begin{tabular}{ |c|c|c|c|c|c| }
    \hline Period& Sub-period& Polarization& First-Last run& Begin date& End
    date \\ \hline
    
    \multirow{2}{2em}{W07}& one& $\downarrow \uparrow$& 259363 - 259677& July
    9& July 15 \\ & two& $\uparrow \downarrow$& 259744 - 260016& July 16& July
    22 \\ \hline

    \multirow{2}{2em}{W08}& one& $\uparrow \downarrow$& 260074 - 260264& July
    23& July 29 \\ & two& $\downarrow \uparrow$& 260317 - 260565& July 29&
    August 5 \\ \hline

    \multirow{2}{2em}{W09}& one& $\downarrow \uparrow$& 260627 - 260852&
    August 5& August 12 \\ & two& $\uparrow \downarrow$& 260895 - 261496&
    August 12& August 26 \\ \hline

    \multirow{2}{2em}{W10}& one& $\uparrow \downarrow$& 261515 - 261761&
    August 26& September 1 \\ & two& $\downarrow \uparrow$& 261970 - 262221&
    September 4& September 9 \\ \hline

    \multirow{2}{2em}{W11}& one& $\downarrow \uparrow$& 262370 - 262772&
    September 11& September 22 \\ & two& $\uparrow \downarrow$& 262831 -
    263090& September 23& September 30 \\ \hline

    \multirow{2}{2em}{W12}& one& $\uparrow \downarrow$& 263143 - 263347&
    September 30& October 7 \\ & two& $\downarrow \uparrow$& 263386 - 263603&
    October 8& October 14 \\ \hline

    \multirow{2}{2em}{W13}& one& $\downarrow \uparrow$& 263655 - 263853&
    October 15& October 21 \\ & two& $\uparrow \downarrow$& 263926 - 264134&
    October 22& October 28 \\ \hline

    \multirow{2}{2em}{W14}& one& $\uparrow \downarrow$& 264170 - 264330&
    October 28& November 2 \\ & two& $\downarrow \uparrow$& 264429 - 264562&
    November 4& November 8 \\ \hline

    \multirow{2}{2em}{W15}& one& $\downarrow \uparrow$& 264619 - 264672&
    November 9& November 11 \\ & two& $\uparrow \downarrow$& 264736 - 264857&
    November 12& November 16 \\ \hline
    
  \end{tabular}
  \caption{COMPASS 2015 data taking periods}
  \label{tab::datataking}
\end{table}

\subsection{Stability Tests} \label{sec::stability}
To ensure the data analyzed were recorded during stable beam and spectrometer
conditions, stability of the analysis data was performed on a spill-by-spill and
run-by-run basis.  The data was recorded in runs with a maximum of 200 spills
per run.  One spill can have several thousand events.

\subsubsection{Bad Spill Analysis} To determine if a given spill is deemed
unstable several macro variables were averaged over per spill and compared to
neighboring spill averages.  These macro variables were chosen specifically to
be sensitive to the general stability conditions of the data collection and are
listed in the following itemized list.  The analysis criteria for an bad
spill events was two oppositely charged muons.  A muon is defined as having
crossed 15 radiation lengths of material.

\begin{itemize}
\item number of beam particles divided by the number of events
\item number of beam particles divided by the number of primary vertices
\item number of hits per beam track divided by the number of beam particles
\item number of primary vertices divided by the number of events
\item number of outgoing tracks divided by the number of events
\item number of outgoing particles from a primary vertex divided by the number
  of primary vertices
\item number of outgoing particle from primary vertex divided by the number of
  events
\item number of outgoing particles from primary vertex divided by the number of
  events
\item number of hits from outgoing particles divided by the number outgoing
  particles
\item number of $\mu^+$ tracks divided by the number of events
\item number of $\mu^+$ tracks from primary vertex divided by the number of
  events
\item number of $\mu^-$ tracks divided by the number of events
\item number of $\mu^-$ tracks from primary vertex divided by the number of
  events
\item $\sum \chi^2$ of outgoing particles divided by the number of outgoing
  particles
\item $\sum \chi^2$ of all vertices divided by the number of all vertices in an
  event
\item Trigger rates (LASxLAS, OTxLAS, LASxMT)
\end{itemize}

If the data collected was stable during a spill the average values from the
macro variables in the above itemized list are expected to be constant from one
spill to the next.  To determine if a spill was recorded in unstable conditions
the spill of interest is compared with its neighboring 2500 spills occurring
before and after in time.  If the spill of interest is over a specified sigma
deviation from any of the neighboring spills too many times, the spill is marked
as a bad spill.  If a spill fails this bad spill criteria for any of the macro
variables in the above itemized list, the spill is deemed bad and not included
in the analysis.  The criteria for the sigma distance and number of times a
spill crosses this distance to be deemed a bad are different for each data
taking period.  In addition to checking the nearest neighbor spills, an entire
run is marked bad if the run has less than 10 spills or greater than 70\% bad
spills.  Table~\ref{tab::badspillpercent} describes the bad spill impact on each
period. \par

\subsubsection{Bad Run Analysis}
The stability of the spectrometer is also verified by a run-by-run check in
parallel to the spill-by-spill checks.  The run-by-run analysis compares
kinematic distributions and the average of these distributions per run to the
kinematic distributions and averages from the other runs in a given period.  The
distributions tested are: x$_{\mathrm{N}}$, x$_{\pi}$, x$_{\mathrm{F}}$,
q$_{\mathrm{T}}$, M$_{\mu\mu}$, P$_{\mu^+}$, P$_{\mu^-}$, P$_{\gamma}$,
P$_{\pi^-}$, and vertex x, y and z positions.  The quantities in the run-by-run
analysis are expected to influence the asymmetries measured, however their
distributions and averages are not expected to have spin-influenced effects from
the limited statistics in just a single run.

The distributions are compared with an unbinned-Kolmogorov test and the
averages over a distribution are compared based on their deviations from each
other.  The unbinned-Kolmogorov test is made between all the runs in a given
period.  A run is marked bad if it is incompatible with most of the runs in a
period.  Additionally, the mean for each distribution in a run is is compared
with the average from a given period.  When an average kinematical variables
from a run is more than five standard deviations from the average within a
period, the run is rejected.  The results of the bad spill rejection after
having already applied the bad spill rejection are shown in
Table~\ref{tab::badspillpercent}.

\begin{table}[h!t]
  \centering
  \begin{tabular}{ |c|c|c| }
    \hline \textbf{Period}& \textbf{Bad Spill Rejection}&
    \textbf{Bad Spill and Bad Run Rejection} \\ \hline \hline
    
    W07& 11.79\%& 17.94\%\\ \hline
    W08& 18.00\%& 21.19\%\\ \hline
    W09& 14.76\%& 17.11\%\\ \hline
    W10& 15.88\%& 17.80\%\\ \hline
    W11& 22.49\%& 26.14\%\\ \hline
    W12& 12.71\%& 13.79\%\\ \hline
    W13& 22.32\%& 22.73\%\\ \hline
    W14& 8.91\%& 10.70\%\\ \hline
    W15& 3.94\%& 3.94\%\\ \hline

  \end{tabular}
  \caption{Stability analysis rejection percentages}
  \label{tab::badspillpercent}
\end{table}

\subsection{Event Selection} \label{sec::dy_eventselection}
The cuts in the event selection were chosen to ensure the event consisted of two
oppositely charged muons resulting from a pion collision in the transversely
polarized target.  The event selection was initially filtered from miniDSTs to
$\mu$DSTs where only events with at least two muons detected were kept in the
$\mu$DSTs.  The cuts used in this analysis are described in the following
enumerated list, where the event selection is performed on the $\mu$DSTs and the
events used are from the slot1 reconstruction.  A summary of the number of
events remaining after the last cuts is shown in Table~\ref{tab::EventTable}.

\begin{enumerate}
\item Two oppositely charged particles from a common best primary vertex having
  an invariant mass between 4.3~{\gvcw} and 8.5~{\gvcw}.  A primary vertex is
  defined as any vertex with an associated beam particle.  In case of multiple
  common primary vertices the best primary vertex is determined by CORAL tagging
  the vertex as best primary (PHAST method PaVertex::IsBestPrimary()).  In the
  case that CORAL did not tag any of the common vertices as the best primary the
  vertex with the smallest spatial $\chi^2$ value is used as the best primary
  vertex.  The mass range of 4.3~{\gvcw} through 8.5~{\gvcw} is deemed the high
  mass range.  Fig.~\ref{fig::DY_InvariantMass} shows the mass invariant mass
  distribution and the background components.  The mass range between 4.3 and
  8.5~{\gvcw} corresponds to over 96~\% Drell-Yan events.
\item A dimuon trigger fired.  A dimuon trigger firing means there are at least
  two particles in coincidence in this event. The dimuon triggers used were a
  coincidence between two particles in the large angle spectrometer, LAS-LAS
  trigger, or a particle in the large angle spectrometer and a particle in the
  Outer hodoscope, LAS-Outer trigger.  The triggering process is further
  described in Sec~\ref{sec::trigger}.  The LAS-Middle trigger was used as a
  veto on beam decay muons.  A beam decay muon results from the decay of a beam
  pion, kaon or anti-proton into a muon depicted as $\pi^- \rightarrow \mu^- +
  \bar{\nu}_{\mu^-}$, $K^- \rightarrow \mu^- + \bar{\nu}_{\mu^-}$, or $\bar{p}
  \rightarrow \mu^- + \bar{\nu}_{\mu^-}$ respectively.  A beam decay muon can
  then be in coincidence with a positive muon from another decay or strong
  reaction in the target resulting in an unwanted background process.  The
  LAS-Middle trigger was used as a veto because this trigger was found to have
  many events resulting from a beam pion decaying to a muon.
\item Both particles are muons.  A muon was defined as having crossed 30
  radiation lengths of material between the particles first and last measured
  points.  This criteria has been previously been determined to be effective at
  distinguishing between muons and hadrons.  In the data production no
  detectors were used from upstream of the hadron absorber so the absorber is
  not included in the determination of material crossed.
\item The first measured point for both particles occurs before 300~cm and the
  last measured point occurs after 1500~cm.  This cut ensures both particles
  have positions upstream of the first spectrometer magnet and downstream of the
  first muon filter.
\item The timing of both muons is defined.  This checks that the time relative
  to the trigger time is determined for both muons so further timing cuts can be
  performed.
\item Both muons are in time within 5~nanoseconds.  This track time for each
  muon is defined relative to the trigger time as in the previous cut.  This cut
  rejects uncorrelated muons.
\item The muon track's spacial reduced $\chi^2$ are individually less than 10.
  This cut ensures track quality.
\item A validation that each muon crossed the trigger it was associated as
  having triggered.  This trigger validation cut was performed by extrapolating
  (PHAST Method PaTrack::Extrapolate()) each muon track back to the two
  hodoscopes it fired and determining if the muon crossed the geometric
  acceptance of both of these hodoscopes.
\item The event does not occur in the bad spill or run list.  Many tests were
  performed to test the basic stability of the spectrometer and beam as
  described in Sec~\ref{sec::stability}.  The spills placed on the bad spill
  list were deemed to occur during unstable data taking conditions.
\item The Drell-Yan kinematics are physical.  That is $0 < x_{\pi} \;
  x_{\mathrm{N}} < 1$ and $-1 < x_{\mathrm{F}} < 1$.
\item The transverse momentum of the virtual photon, $q_T$, is between 0.4 and
  5.0 GeV/c.  The lower limit ensures the azimuthal angular resolution is
  sufficient and the upper cut further ensures the kinematic distributions are
  physically possible and not badly reconstructed events.
\item The vertex originated within the z-positions of the transversely polarized
  target cells defined by the target group.  -294.5~cm$<$ Z$_{\mathrm{vertex}}$
  $<$-239.3~cm for the upstream target and -219.5~cm$<$ Z$_{\mathrm{vertex}}$
  $<$-164.3~cm for the downstream target.
\item The vertex is within the radius of the polarized target measured to be
  1.9~cm.
\end{enumerate}

\begin{table}[h!t]
  \begin{adjustwidth}{-2cm}{}
    \begin{tabular}{ |c|c|c|c|c|c|c|c|c|c|c|c| }
      \hline \textbf{Cuts}& \textbf{W07}& \textbf{W08}& \textbf{W09}&
      \textbf{W10}& \textbf{W11}& \textbf{W12}& \textbf{W13}& \textbf{W14}&
      \textbf{W15} & \textbf{WAll} & \textbf{Remaining} \\ \hline

      \multirow{2}{13em}{High Mass $\mu^-\mu^+$ with a common best primary
        vertex}& 19410& 19184& 19654& 20707& 31371& 23563& 20561& 13154& 7697&
      175301& 100.00 \% \\ & & & & & & & & & & & \\ \hline
      
      Good Spills& 15947& 14899& 16217& 16895& 23041& 20184& 16026& 11796& 7422&
      142427& 81.70 \% \\ \hline

      0$<$ x$_{\pi}$ x$_N$ $<$1, -1$<$ x$_F$ $<$1& 15932& 14886& 16200& 16885&
      23022& 20171& 16013& 11794& 7414& 142317& 81.70 \% \\ \hline

      0.4$<$ q$_T$ $<$5(GeV/c)& 14342& 13385& 14609& 15239& 20667& 18101& 14365&
      10588& 6636& 127932& 60.75 \% \\ \hline

      Z Vertex within NH$_3$& 4256& 4024& 4330& 4552& 6369& 5503& 4411& 3130&
      2028& 38603& 15.05 \% \\ \hline

      Vertex Radius $<$ 1.9cm& 4175& 3950& 4257& 4474& 6252& 5414& 4334& 3078&
      1987& 37921& 12.21 \% \\ \hline
      
    \end{tabular}
    \caption{Numbers of selected di-muon events in this analysis of 2015 COMPASS
      data}
    \label{tab::EventTable}
  \end{adjustwidth}
\end{table}

\subsection{Binning}
The asymmetries are measured in bins of $x_N$, $x_{\pi}$, $x_F$,
$q_T$, and $M_{\mu\mu}$. $x_N$ and $x_{\pi}$ are the momentum
fractions of the target nucleon and beam pion respectively, $x_F$ =
$x_{\pi} - x_N$, $q_T$ is the transverse momentum of the virtual
photon and $M_{\mu\mu}$ is the invariant mass of the di-muon.  The binning was
determined by requiring equal statistical population in each kinematic bin.  In
addition, the asymmetries are determined in an integrated bin using all the
analysis data.  The binning limits of the analysis are summarized in
Table~\ref{tab::binning}.

\begin{table}[h!t]
  \centering
  \begin{tabular}{ |c|c|c|c|c| }
    \hline \textbf{Kinematics}& \textbf{Lowest limit}& \textbf{Upper limit bin
      1}& \textbf{Upper limit bin 2}& \textbf{Upper limit bin 3}\\ \hline
    
    $x_N$& 0.0& 0.13& 0.19& 1.0\\ \hline $x_{\pi}$& 0.0& 0.40& 0.56&
    1.0\\ \hline $x_F$& -1.0& 0.22& 0.41& 1.0\\ \hline $q_T$ (GeV/c)& 0.4& 0.86&
    1.36& 5.0\\ \hline $M_{\mu\mu}$ (GeV/c$^2$)& 4.3& 4.73& 5.50& 8.5 \\ \hline
    
  \end{tabular}
  \caption{Analysis binning limits}
  \label{tab::binning}
\end{table}

\subsection{Analysis Notation}
Table~\ref{tab::ANnotations} summarizes the general notations used in the
asymmetry analysis definitions and derivations used throughout this chapter.

\begin{table}[h!t]
  \centering
  \caption{Notations used for defining the asymmetry analysis}
  \label{tab::ANnotations}
  \begin{tabular}{ |c|c| }
    
    \hline \textbf{Notation}& \textbf{Description}
    \\ \hline 1(2)& target cell
    number. 1=upstream, 2=downstream \\ \hline
    $\uparrow(\downarrow)$ & target
    cell vertical polarization direction, up(down) \\ \hline
    $|S_T|$& fraction of
    polarized target nucleons \\ \hline
    $l(r)$ & virtual photon detected left(right) of spin \\ \hline
    $J(S)$ & spectrometer Jura(Saleve) side meaning west(east) side
    \\ \hline
    
  \end{tabular}
\end{table}

\section{Transverse Spin-Dependent Asymmetries} \label{sec::standTSA}
This section describes the standard TSA analysis, in Drell-Yan, for which the
results of 2015 data are published in reference~\cite{compassDYpaper}.  The main
motivation for this analysis was to conclude on the sign flip of the Sivers
function flip between the Drell-Yan and SIDIS processes using data from the same
experimental setup for both processes.  The results shown are those determined
by the COMPASS Drell-Yan analysis group.

The kinematical distributions shown for this analysis are the same as for the
remaining analyses in this chapter.  This results from the fact that all the
analyses in this chapter use the same event selection and cuts.  The only
exception to this is for the $q_T$-weighted analysis, Sec~\ref{sec::qtweighted},
which cannot cut on $q_T$ and therefore has a different $q_T$ distribution as is
explained in Sec~\ref{sec::high_qt}.

As was noted in the event selection~\ref{sec::dy_eventselection}, the data
considered are in the invariant mass range [4.3-8.5~{\gvcw}].
Fig.~\ref{fig::DY_InvariantMass} shows the invariant mass range from the 2015
COMPASS data.  All cuts except a cut on invariant mass are included in
Fig.~\ref{fig::DY_InvariantMass} and as well a fit to show the background
processes is included.

The fit is determined from Monte-Carlo data, described in
Table~\ref{tab::MCproduction}, and combinatorial background analysis.  The
Monte-Carlo data simulated all hard processes with a decay to two oppositely
charged muons and can be reconstructed in the COMPASS spectrometer.
Combinatorial background analysis estimates the background as $N_{combinatorial}
= 2\sqrt{N_{\mu^+\mu^+}N_{\mu^-\mu^-}}$.  As can be seen from the Monte-Carlo
curves in Fig.~\ref{fig::DY_InvariantMass}, there are two distinguishable
background peaks.  The lower mass peak at about 3~{\gvcw} corresponds to
J/$\Psi$ production and the higher mass peak at around 3.6~{\gvcw} corresponds
to $\Psi$' production.  All the analyses in this chapter use the mass range
between 4.3 and 8.5~{\gvcw} and as Fig.~\ref{fig::DY_InvariantMass} shows, the
Drell-Yan process dominates in this mass range.  The background percentage was
estimated to be below 4\% in this mass range.

\begin{figure}[h!t]
  \centering \includegraphics[width=0.6\textwidth,trim=1cm 7cm 1cm 7cm,
    clip]{DY_InvariantMass}
  \caption{The 2015 COMPASS invariant dimuon mass distribution and a fit to this
    data.  The data fit is from Monte-Carlo and combinatorial background
    analysis and is provided to show the background processes.  This image is
    taken from~\cite{compassDYpaper}.}
  \label{fig::DY_InvariantMass}
\end{figure}

Fig.~\ref{fig::DY_qT} shows the transverse virtual photon momentum, $q_T$,
distribution.  With the cut on $q_T$ between [0.4-5({\gvc})], the average $q_T$
is 1.2~{\gvc} while the average $M_{\mu\mu}$ is 5.3~{\gvcw}.  As stated in
chapter~\ref{ch::theory_exp}, the regime where TMD functions are the theoretical
model for parton distributions is when $q_T << M_{\mu\mu}$.  While the average
$q_T$ is less than the average $M_{\mu\mu}$, it is not excluded that the results
in this chapter are outside of the TMD regime.  Nevertheless all the results
presented in this chapter are determined assuming the TMD description is valid.

\begin{figure}[h!t]
  \centering
  \begin{subfigure}{0.45\textwidth}
    \centering \includegraphics[width=\textwidth,trim=2cm 7.8cm 2cm 7cm,
      clip]{DY_qT}
    \caption{The $q_T$ distribution where the shaded region shows the data used
      in the high mass analysis and the unshaded region shows the full
      distribution without a $q_T$ cut.  This image is taken
      from~\cite{compassDYpaper}.}
    \label{fig::DY_qT}
  \end{subfigure}
  \begin{subfigure}{.02\textwidth}
    \includegraphics[width=\linewidth]{tmp3}
    \label{fig::tmp3}%
  \end{subfigure}
    \begin{subfigure}{0.48\textwidth}
    \centering \includegraphics[width=\textwidth,trim=3.1cm 3.8cm 3.1cm
      3.8cm,clip]{DY_xPivxN}
    \caption{The 2-dimensional distribution of $x_{\pi}$ vs. $x_{N}$.  Both
      $x_{\pi}$ and $x_N$ are safely in their respective valence regions.  This
      image is taken from~\cite{compassDYpaper}.}
    \label{fig::DY_xPivxN}
    \end{subfigure}
    \caption{High mass Drell-Yan kinematic variables: $q_T$ and $x_N$
      vs. $x_\pi$.}
\end{figure}

The distribution of $x_{\pi}$ versus $x_N$ is shown in
Fig.~\ref{fig::DY_xPivxN}.  The Bjorken-x of the proton, $x_N$, is almost
exclusively above 0.1 and as well Bjorken-x for the pion, $x_{\pi}$ is in its
valence region.  For these reasons it is safe to say that the Drell-Yan reaction
studied in the following analyses is the result of the pion's anti-u-quark
annihilating with the proton's u-quark.

The results from TSA analysis are determined from an extended unbinned maximum
likelihood fit to the data.  The virtual photon depolarization values are
determined on an event by event basis unlike the other analyses in this chapter.
The published integrated results of the 2015 data for the leading order and
sub-leading order TSAs are shown in Fig.~\ref{fig::DY_intAsymAmps}.  The leading
order TSAs are non-zero with approximate significance of: 1 sigma for the
Sivers TSA, $A_T^{\sin(\phi_S)}$, 1.2 sigma for the pretzelosity TSA,
$A_T^{\sin(2\phi_{CS}+\phi_S)}$ and 2 sigma for the transversity TSA,
$A_T^{\sin(2\phi_{CS}-\phi_S)}$.

\begin{figure}[h!t]
  \centering \includegraphics[width=0.43\textwidth,trim=3cm 7cm 3cm 6cm,
    clip]{DY_intAsymAmps}
  \caption{The integrated TSAs with statistical error bars and systematic
    uncertainty bands.  $A_T^{\sin(\phi_S)}$, $A_T^{\sin(2\phi_{CS}+\phi_S)}$,
    and $A_T^{\sin(2\phi_{CS}-\phi_S)}$ are leading order TSAs and
    $A_T^{\sin(\phi_{CS}+\phi_S)}$ and $A_T^{\sin(\phi_{CS}-\phi_S)}$ are
    sub-leading order TSAs.}
  \label{fig::DY_intAsymAmps}
\end{figure}

The comparison of the Sivers TSA, $A_T^{\sin(\phi_S)}$, with the expected sign
flip is shown in Fig.~\ref{fig::DY_Siv_signFlip}.  The positive solid theory
curves show the expected Sivers TSA assuming the Sivers function flips sign
between Drell-Yan and SIDIS.  The main difference in these three theory curves
is the $Q^2$ evolution which is also the main uncertainty in each prediction.
As can be seen the Sivers TSA is compatible with the expected sign change.
However, the error bars on the Sivers asymmetry amplitude are too large to
conclusively distinguish between the three theory curves or even to definitively
conclude on the sign change between Drell-Yan and SIDIS.  That being said, the
amplitude $A_T^{\sin(\phi_S)}$ is 2 sigma away from being incompatible with a
sign flip.

\begin{figure}[h!t]
  \centering \includegraphics[width=0.6\textwidth, trim=0.5cm 7cm 0.5cm 7cm,
    clip]{DY_Siv_signFlip}
  \caption{The Sivers TSA along with theory curves for the expected sign change
    (sold curves) and without the sign change (opaque curves).  Theory curves
    and uncertainties are calculated using $Q^2$ evolution from
    DGLAP~\cite{Anselmino:2016uie}, TMD-1~\cite{Echevarria:2014xaa},
    TMD-2~\cite{Sun:2013hua}.  This image is taken from~\cite{compassDYpaper}.}
  \label{fig::DY_Siv_signFlip}
\end{figure}


\section{Double Ratio Analysis} \label{sec::doubleratio}
The double ratio method is used to determine spin-dependent asymmetry
amplitudes.  This means the asymmetry amplitudes $A^{\sin\phi_S}_T$,
$A^{\sin(2\phi+\phi_S)}_T$ and $A^{\sin(2\phi-\phi_S)}_T$ can be determined from
the 2015 transversely polarized Drell-Yan data.  The benefit of this method is
that the spectrometer acceptance does not affect the determination of the
asymmetry amplitudes.  The author of this thesis performed the analysis in this
section and found results consistent with those determined from TSA analysis.

\subsection{Asymmetry Extraction}
The double ratio is defined as

\begin{equation}
  R_D(\Phi) =
  \frac{N_1^{\uparrow}(\Phi)N_2^{\uparrow}(\Phi)}
       {N_1^{\downarrow}(\Phi)N_2^{\uparrow}(\Phi)},
\end{equation}
\noindent
where $N$ represents the counts, 1(2) is the upstream(downstream) target cell
and $\uparrow$($\downarrow$) denotes the transverse polarization direction.  The
number of counts, $N(\Phi)$, is defined as

\begin{equation}
  \label{equ::countsdef}
  N(\Phi) = L * \sigma(\Phi) * a(\Phi),
\end{equation}

\noindent
where $L$ is the luminosity, $\sigma$ is the cross-section and $a$ is the
spectrometer acceptance.  In Eq.~\ref{equ::countsdef} the acceptance is a
function of detector efficiencies and the spectrometer acceptance.  When
assuming the spin-dependent Drell-Yan cross-section,
Eq.~\ref{equ::DY_MostusefulXsect}, the number of counts, $N(\Phi)$, can be
written

\begin{equation}
  \label{equ::spindependentCounts}
  N(\Phi) = a(\Phi)L\sigma_U\Big(1 \pm D_{[\theta]}|S_T|A^w_T\sin(\Phi)\Big).
\end{equation}
\noindent
where +(-) is for target polarized up(down), $D_{[\theta]}$ is the virtual
photon depolarization factor and $|S_T|$ is the target polarization percentage.
The depolarization factor, $D_{[\theta]}$ is defined in
Eq.~\ref{equ::DYDepolarizationFactorDef}.  It can be thought of as the
probability for the virtual photon to decay and produce such an asymmetry
amplitude to that for a transversely polarized photon decay.  The target
polarization percentage, $|S_T|$ is defined as $fP$, where $f$ is the dilution
factor, Eq.~\ref{equ::dilution}, and $P$ is the target polarization percentage.
Therefore the double ratio can be written
\begin{align}
  R_D(\Phi) &= \frac{ a_1^{\uparrow}(\Phi)L_1^{\uparrow}\sigma_U\Big(1 +
    D_{[\theta]1}^{\uparrow}|S_{T1}^{\uparrow}|A^w_T\sin(\Phi)\Big)
    a_2^{\uparrow}(\Phi)L_2^{\uparrow}\sigma_U\Big(1 +
    D_{[\theta]2}^{\uparrow}|S_{T2}^{\uparrow}|A^w_T\sin(\Phi)\Big) } {
    a_1^{\downarrow}(\Phi)L_1^{\downarrow}\sigma_U\Big(1 -
    D_{[\theta]1}^{\downarrow}|S_{T1}^{\downarrow}|A^w_T\sin(\Phi)\Big)
    a_2^{\downarrow}(\Phi)L_2^{\downarrow}\sigma_U\Big(1 -
    D_{[\theta]2}^{\downarrow}|S_{T2}^{\downarrow}|A^w_T\sin(\Phi)\Big) }
  \\ \nonumber &= \Big(\frac{a_1^{\uparrow}(\Phi)a_2^{\uparrow}(\Phi)}
     {a_1^{\downarrow}(\Phi)a_2^{\downarrow}(\Phi)} \Big)
     \Big(\frac{L_1^{\uparrow}L_2^{\uparrow}}
         {L_1^{\downarrow}L_2^{\downarrow}}\Big)
         \frac{\Big(1+D_{[\theta]1}^{\uparrow}|S_{T1}^{\uparrow}|A^w_T\sin(\Phi)\Big)
           \Big(1+D_{[\theta]2}^{\uparrow}|S_{T2}^{\uparrow}|A^w_T\sin(\Phi)\Big)}
              {\Big(1-D_{[\theta]1}^{\downarrow}|S_{T1}^{\downarrow}|A^w_T\sin(\Phi)\Big)
                \Big(1-D_{[\theta]2}^{\downarrow}|S_{T2}^{\downarrow}|A^w_T\sin(\Phi)\Big)
              }.
\end{align}
\noindent
As is described in Sec~\ref{sec::datacollection}, the data is collected in two
week periods where the conditions of the spectrometer are frozen for each data
taking period.  For this reason the following reasonable acceptance assumption
is made
\begin{equation}
  \label{equ::a_resonable_assump}
  \frac{a_1^\uparrow(\Phi) a_2^\uparrow(\Phi)}
       {a_1^\downarrow(\Phi) a_2^\downarrow(\Phi)}
       = C.
\end{equation}
\noindent
where $C$ is a constant.  In addition
$L^{\downarrow(\uparrow)}_2$ = $rL^{\uparrow(\downarrow)}_1$ where $r$ is a
constant reduction factor and therefore the luminosity terms cancel out as
\begin{equation}
  \frac{L_1^{\uparrow}L_2^{\uparrow}}{L_1^{\downarrow}L_2^{\downarrow}}
  = \frac{L_1^{\uparrow}rL_1^{\downarrow}}{L_1^{\downarrow}rL_1^{\uparrow}}
  = 1.
\end{equation}
\noindent
Finally the asymmetry amplitudes and target polarizations are assumed to be
small so the double ratio can be simplified to

\begin{align}
  R_D(\Phi) &=
  C\frac{\Big(1+D_{[\theta]1}^{\uparrow}|S_{T1}^{\uparrow}|A^w_T\sin(\Phi)\Big)
    \Big(1+D_{[\theta]2}^{\uparrow}|S_{T2}^{\uparrow}|A^w_T\sin(\Phi)\Big)}
  {\Big(1-D_{[\theta]1}^{\downarrow}|S_{T1}^{\downarrow}|A^w_T\sin(\Phi)\Big)
    \Big(1-D_{[\theta]2}^{\downarrow}|S_{T2}^{\downarrow}|A^w_T\sin(\Phi)\Big)}
  \\ \nonumber &\approx
  C\frac{1+\Big[D_{[\theta]1}^{\uparrow}|S_{T1}^{\uparrow}|+D_{[\theta]2}^{\uparrow}|S_{T2}^{\uparrow}|\Big]
    A^w_T\sin(\Phi)}
  {1-\Big[D_{[\theta]1}^{\downarrow}|S_{T1}^{\downarrow}|+D_{[\theta]2}^{\downarrow}|S_{T2}^{\downarrow}|\Big]
    A^w_T\sin(\Phi)} \\ \nonumber &\approx
  C\Big(1+\Big[D_{[\theta]1}^{\uparrow}|S_{T1}^{\uparrow}|+D_{[\theta]2}^{\uparrow}|S_{T2}^{\uparrow}|\Big]
  A^w_T\sin(\Phi)\Big)\Big(1+\Big[D_{[\theta]1}^{\downarrow}|S_{T1}^{\downarrow}|+D_{[\theta]2}^{\downarrow}|S_{T2}^{\downarrow}|\Big]
  A^w_T\sin(\Phi)\Big) \\ \nonumber &\approx C\Big(1 +
  \Big[D_{[\theta]1}^{\uparrow}|S_{T1}^{\uparrow}|+D_{[\theta]2}^{\uparrow}|S_{T2}^{\uparrow}|+D_{[\theta]1}^{\downarrow}|S_{T1}^{\downarrow}|+D_{[\theta]2}^{\downarrow}|S_{T2}^{\downarrow}|\Big]A^w_T\sin(\Phi)\Big).
\end{align}
\noindent
Then making the assumption that the polarizations, $S_T$, and the virtual photon
depolarization factors, $D_{\Phi}$ are approximately constant throughout a data
period, the asymmetry amplitude of interest can be determined by fitting the
double ratio with the function
\begin{equation}
  \label{equ::dr_fit_formula}
  f(\Phi) = [p0](1+4[p1]\sin(\Phi),
\end{equation}
\noindent
where $[p0]$ and $[p1]$ are fit parameters and $[p1]$ represents the asymmetry
amplitude of interest.  The $[p1]$ parameter is later corrected for average
polarization and virtual photon depolarization factors.

The double ratio, $R_D$, is determined as a function of $\Phi$, where the angle
$\Phi$ depends on which asymmetry amplitude is being determined.  The assumption
made in the measured counts formula, Eq.~\ref{equ::spindependentCounts}, is that
all angles except the spin-dependent $\Phi$ angle are integrated over.  When
this is the true, all the Drell-Yan cross-section fourier components integrate
to zero except the constant term.  The following table,
Table~\ref{tab::ratio_phiAngles}, list which $\Phi$ angle is used to determine
which spin-dependent asymmetry amplitude.

\begin{table}[h!t]
  \centering
  \caption{Measured counts as a function of each $\Phi$ angle}
  \begin{tabular}{ |c|c|c|c| }
    \hline \textbf{Asymmetry Amplitude}& \textbf{Corresponding TMD Function}&
    \textbf{$\Phi$ Angle}& \textbf{$\Phi$ Range (radians)} \\ \hline
    
    $A^{\sin(\phi_S)}_T$& Sivers& $\phi_S$& [-$\pi$, $\pi$] \\ \hline

    $A^{\sin(2\phi-\phi_S)}_T$& Transversity& $2\phi-\phi_S$& [-3$\pi$, 3$\pi$]
    \\ \hline

    $A^{\sin(2\phi+\phi_S)}_T$& Pretzelosity& $2\phi+\phi_S$& [-3$\pi$, 3$\pi$]
    \\ \hline
  \end{tabular}
    \label{tab::ratio_phiAngles}
\end{table}

The variance of the double ratio, assuming Poisson counting statistics, is

\begin{equation}
  \sigma^2_{R_D} = R^2_D(\Phi)\Big(\frac{1}{N_1^\uparrow(\Phi)}
    + \frac{1}{N_2^\uparrow(\Phi)}
    + \frac{1}{N_1^\downarrow(\Phi)}
    +\frac{1}{N_2^\downarrow(\Phi)}
   \Big).
\end{equation}

\subsection{Results}\label{sec::doubleratio_results}
The results of the asymmetry amplitudes are determined in each of the nine
periods and then combined as a weighted average.  The asymmetries are calculated
this way to minimize the effects of acceptance changes between periods as the
spectrometer was kept stable within each period but had the options for detector
changes and repairs between periods.  As well this weighted average method
allows for future measurements to be combined as a weighted average with the
final overall results without the need to know individual period results.  This
resulting asymmetry amplitudes are determined from a weighted average as
\begin{equation}
  \label{equ::wAvg}
  A = \frac{
    \sum_{period}
  A_{period}\sigma^{-2}_{period}
  }{
    \sum_{period} \sigma^{-2}_{period}
    },
  \quad \delta A = \sqrt{\sum_{period}
  \frac{1}{\sigma^{-2}_{period}}}.
\end{equation}

For each period and each kinematical bin, the asymmetry is determined by fitting
the double ratio and with Eq.~\ref{equ::dr_fit_formula}.  The results of the fit
actually determines the quantity

\begin{equation}
  A^w_T \langle D_{[\theta]} \rangle \langle |S_T|\rangle.
\end{equation}
\noindent
The asymmetry amplitude is ultimately determined by dividing the fit results by
the average polarization and average virtual photon depolarization values per
period.

To determine the asymmetry amplitude, the double ratio is binned in eight bins
in $\Phi$.  Eight bins are chosen due to the low statistics from Drell-Yan data.
Fig.~\ref{fig::dr_example_trans} shows an example of the binned double ratio and
fit results.

\begin{figure}[h!t]
  \centering \includegraphics[width=0.6\textwidth,trim=0cm 1cm 0cm 1cm,
    clip]{dr_example_trans}
  \caption{An example double ratio and corresponding fit (red) to determine the
    amplitude $A_T^{\sin(2\phi-\phi_S)}$}
  \label{fig::dr_example_trans}
\end{figure}

\noindent
The results for all the spin-dependent asymmetry amplitudes are shown in
Fig.~\ref{fig::dr_final_results}.  As can be seen, the significance of the
integrated asymmetry amplitudes is: over 1 sigma above zero for the Sivers,
$A^{\sin\phi_S}_T$, over 3 sigma above zero for preztelosity,
$A^{\sin(2\phi+\phi_S)}_T$, and 3 sigma below zero for transversity,
$A^{\sin(2\phi-\phi_S)}_T$.

\begin{figure}[h!t]
  \centering \includegraphics[width=\textwidth,trim=0cm 2.5cm 0cm 2.5cm,
    clip]{dr_final_results}
  \caption{The results and statistical error bars for the transverse
    spin-dependent asymmetry amplitudes $A^{\sin\phi_S}_T$ (top),
    $A^{\sin(2\phi+\phi_S)}_T$ (middle) and $A^{\sin(2\phi-\phi_S)}_T$ (bottom)
    determined from the double ratio method.}
  \label{fig::dr_final_results}
\end{figure}


\section{$q_T$-Weighted Asymmetries} \label{sec::qtweighted}
The $q_T$ weighted asymmetries analysis is used to determine three asymmetry
amplitudes related to TMD functions.  This analysis determined the three
amplitudes: $A_T^{\sin(\phi_S) q_T/M_N}$, $A_T^{\sin(2\phi+\phi_S)
  q^3_T/(2M_{\pi}M_N^2)}$ and $A_T^{\sin(2\phi-\phi_S) q_T/M_{\pi}}$ which are
related to the Sivers, preztelosity and transversity TMD PDFs respectively.

The theoretical introduction and motivation for measuring $q_T$-weighted
asymmetries is provided in Sec~\ref{sec::qt_w_theory}.  The author of this
thesis was a cross checker for the $q_T$-weighted asymmetry results which is a
required step for any results to become public.  For the full details of the
$q_T$-weighted analysis see reference~\cite{janthesis}.

\subsection{Event Selection}
The results for this analysis were released prior to the slot1 reconstruction
production and therefore this analysis uses the t3 reconstruction.  For
$q_T$-weighted asymmetries the results depend on the full range of the $q_T$
distribution.  In the other analyses in this chapter however, a cut was placed
on high and low $q_T$ values to ensure better azimuthal angular resolution and
quality reconstructed events.  This cut cannot be applied for $q_T$-weighted
analysis because it will affect the weighting used to determine the asymmetry
amplitudes.  On the other hand the combinatorial background and badly
reconstructed events from the high $q_T$ phase space should be cut.  The next
section goes into the details and the remedy for a $q_T$ related cut. All of the
other cuts from Sec~\ref{sec::dy_eventselection} are the same except for this
$q_T$ cut. Figure~\ref{tab::qt_EventTable} provides the final cut order and the
remaining statistics after each cut for this $q_T$-weighted analysis.

\subsubsection{High $q_T$} \label{sec::high_qt}
The $q_T$ distribution without any $q_T$ cuts is shown in
Fig.~\ref{fig::qT_noCuts}.  As can be seen the $q_T$ distribution reaches values
much higher than the maximum 5~{\gvc} cut from the other analyses in this
chapter.  The most fundamental problem with this $q_T$ distribution is that some
of the events violate conservation of momentum.  A first remedy to the high
$q_T$ values then is to add a cut which demands momentum conservation.  This is
achieved by demanding that the momentum sum of the detected muons is physically
possible, $\ell^+ \; + \; \ell^- \; < \; 190$~GeV/c.  Note that this cut does
not take into account the momentum spread of the beam due to the fact that the
beam momentum spread is expected to be small.  Fig.~\ref{fig::qT_PconserveCut}
shows how this cut affects the $q_T$ distribution.  As can be seen, $q_T$ still
reaches values much higher than the 5~GeV/c cut from the other TMD analyses.
The remaining high $q_T$ events still have the potential to be poorly
reconstructed events or combinatorial background and for this reason an
additional cut was put on the individual muons transverse momentum such that
$\ell_T^{\pm} \; <$ 7~GeV/c.

\begin{figure}[h!t]
  \centering
  \begin{subfigure}{.46\textwidth}
    \centering \includegraphics[width=\linewidth, trim=6cm 8.7cm 6cm 8cm,
      clip]{qT_noCuts}
    \caption{$q_T$ distribution without cuts on $q_T$.  All other cuts expect
      the $q_T$ cut from Sec~\ref{sec::dy_eventselection} are applied.  This
      image is from~\cite{janthesis}}
    \label{fig::qT_noCuts}
  \end{subfigure}%
  \begin{subfigure}{.02\textwidth}
    \centering
    \includegraphics[width=\linewidth]{tmp2}
    \label{fig::tmp2}%
  \end{subfigure}
  \begin{subfigure}{.46\textwidth}
    \centering \includegraphics[width=\linewidth, trim=6cm 9cm 6cm 8cm,
      clip]{qT_PconserveCut}
    \caption{$q_T$ distribution after the momentum conservation cut is added,
      $\ell^+ \; + \; \ell^- \; < \; 190$ GeV/c.  All other cuts expect the
      $q_T$ cut from Sec~\ref{sec::dy_eventselection} are applied.  This image
      is from~\cite{janthesis}}
    \label{fig::qT_PconserveCut}
  \end{subfigure}
  \caption{$q_T$ distributions without and with momentum conservation cuts.}
\end{figure}

\begin{figure}[h!t]
  \centering
    \begin{tabular}{ |c|c|c| }
      \hline \textbf{Cuts}& \textbf{Events} & \textbf{\% Remaining} \\ \hline

      \multirow{2}{15em}{$\mu^+\mu^-$ from best primary vertex,
        4.3$<M_{\mu\mu}<8.5$ GeV/c$^2$}& 1,159,349& 100.00\\ & & \\ \hline
      
      \multirow{2}{14em}{Triggers: (2LAS or LASxOT) and not LASxMiddle}& 
      868,291& 74.89\\ & & \\ \hline
      
      Z$_{first} <$ 300 cm, Z$_{last} >$ 1500 cm& 784,379& 67.66\\ \hline

      $\Delta$t defined& 776,643& 66.99\\ \hline
      
      $|\Delta\mathrm{t}| <$ 5 ns& 337,081& 32.18\\ \hline

      $\chi^2_{track}$/ndf $<$ 10& 370,054& 31.92\\ \hline

      $\ell^+ + \ell^- < 190$ GeV/c& 219,304& 18.92\\ \hline

      $\ell_T^\pm < 7$ GeV/c& 219,014& 18.89\\ \hline

      Trigger Validation& 168,939& 14.57\\ \hline

      Good Spills& 137,812& 11.89\\ \hline

      0$<$ x$_{\pi}$ x$_N$ $<$1, -1$<$ x$_F$ $<$1& 137,802& 11.89  \\ \hline

      Z Vertex within NH$_3$& 42,646& 3.68\\ \hline

      Vertex Radius $<$ 1.9cm& 39,088& 3.37\\ \hline

    \end{tabular}
    \caption{Event selection statistics for $q_T$-weighed asymmetry analysis
      from all periods combined}
    \label{tab::qt_EventTable}
\end{figure}

\subsection{Binning}
As with the other analyses in this chapter, the asymmetry is determined in bins
of the Drell-Yan physical kinematic variables: $x_N$, $x_\pi$, $x_F$,
$M_{\mu\mu}$ and an overall integrated value.  No $q_T$ binning is used however,
because a full integration of the $q_T$ variable needs to be taken into account
to form the weighted asymmetry.

\subsection{Asymmetry Method}
The weighted asymmetry amplitudes $A_T^{\sin(\phi_S) q_T/M_N}$,
$A_T^{\sin(2\phi+\phi_S) q^3_T/(2M_{\pi}M_N^2)}$ and $A_T^{\sin(2\phi-\phi_S)
  q_T/M_{\pi}}$ are determined using a modified double ratio.  As with the
double ratio method from Sec~\ref{sec::doubleratio}, the modified double ratio
does not depend on the spectrometer acceptance.  The modified double ratio is
defined as

\begin{equation}
  \label{equ::modified_dr}
  R^W_{DM}(\Phi)=
  \frac{N^{\uparrow W}_{1}N^{\uparrow W}_{2}
    - N^{\downarrow W}_{1}N^{\downarrow W}_{2}}
       {\sqrt{(N^{\uparrow W}_{1}N^{\uparrow W}_{2}
         + N^{\downarrow W}_{1}N^{\downarrow W}_{2})
         (N^{\uparrow}_{1}N^{\uparrow}_{2}
         + N^{\downarrow}_{1}N^{\downarrow}_{2})}},
\end{equation}
\noindent
where similar notation is used from the previous analyses where
$\uparrow$($\downarrow$) is the transverse polarization direction, 1(2) denotes
the upstream(downstream) cell, $N^{W}$ is the weighted counts, $W$ is the weight
used and $N$ denotes the unweighted counts.  The angles $\Phi$, in the modified
double ratio, are the same used for the double ratio,
Table~\ref{tab::ratio_phiAngles}, and give access to asymmetry amplitudes
related to the same corresponding TMD functions.  Under the same reasonable
acceptance ratio assumption, Eq.~\ref{equ::a_resonable_assump}, from the double
ratio method the acceptance cancels out in the double ratio method.  Using this
assumption, the modified double ratio reduces to

\begin{equation}
  \label{equ::dr_fit_form}
  R^W_{DM}(\Phi) \approx 2 \tilde{D}_{\sin\Phi}\langle S_T \rangle
  A_T^{\sin(\Phi)W} \sin\Phi,
\end{equation}
\noindent
where $\tilde{D}_{\sin\Phi}$ is an integrated virtual photon depolarization
factor defined as

\begin{equation}
  \tilde{D}_{\sin\phi_S} = 1, \quad\quad \tilde{D}_{\sin(2\phi\pm\phi_S)} =
  \frac{\int a(\theta)\sin^2\theta d\cos\theta} {\int a(\theta)(1+\cos^2\theta)
    d\cos\theta} = \frac{1-\langle \cos^2\theta\rangle} {1+\langle
    \cos^2\theta\rangle}.
\end{equation}

The statistical error for the modified double ratio is

\begin{equation}
  \sigma^2_{R^W_{DM}} = \frac{\sum_{c,p} \sigma^2_{N_c^{pW}}
    4(N^{\uparrow}_1N^{\uparrow}_2)N^{\downarrow}_1N^{\downarrow}_2)^2}
        {\sum_{c,p} \sigma^2_{N_c^{p}}
          (N^{\uparrow}_1N^{\uparrow}_2 + N^{\downarrow}_1N^{\downarrow}_2)^4}
        \sum_{c,p}\frac{1}{N_c^p},
\end{equation}
\noindent
where $\sigma^2_{N_c^{pW}} = \sum (W^p_c)^2$ is the sum of event weights, $c$ is
cell 1 or cell 2 and $p$ is polarization $\uparrow$ or $\downarrow$.

The weighted asymmetry amplitudes are determined by forming the modified double
ratio in eight bins in the appropriate $\Phi$ angle and fitting this
distribution.  If an infinite number of bins were used and there was sufficient
data, the modified double ratio would be the function form of
Eq.~\ref{equ::dr_fit_form}.  Due to the limited statistics however, $R^W_{DM}$
must be binned in a finite number of bins.  Therefore to account for the fact
that ratio is determined in a finite number of $\Phi$ bins, the average value of
Eq.~\ref{equ::dr_fit_form} over the bin width is used as the fit distribution.
This means the functional fit is

\begin{equation}
  \langle R^W_{DM} \rangle = \frac{1}{\Delta\Phi}
  \int_{\Phi_i-\frac{\Delta\Phi}{2}}^{\Phi_i+\frac{\Delta\Phi}{2}}
  R^W_{DM}(\Phi') d\Phi' =
  \frac{2}{\Delta\Phi}\sin(\frac{\Delta\Phi}{2})R^W_{DM}(\Phi_i),
\end{equation}
\noindent
where $\Delta\Phi$ = $\frac{2\pi}{8}$ for eight bins in $\Phi$.
Fig.~\ref{fig::DRFitSiv} shows the double ratio as a function of $\Phi=\phi_S$
for period W07 in one bin of $x_N$.  One $R^W_{DM}$ is determined for each of
the 3 (number of bins) $\times$ 9 (number of periods) $\times$ 3 (number of
asymmetry amplitudes) = 81 modified double ratios.

\begin{figure}[h!t]
  \centering \includegraphics[width=0.6\textwidth, trim=0cm 2.5cm 0cm
    2.5cm,clip]{DRFitSiv}
  \caption{The double ratio as a function of $\phi_S$ used to determine the
    Sivers asymmetry amplitude.  This is for period W07 and the lowest bin in
    $x_N$.  The red line shows the fit.  The results of the fit are shown in the
    statistics box.}
  \label{fig::DRFitSiv}
\end{figure}

\subsection{Results}
As explained in Sec~\ref{sec::doubleratio_results}, the asymmetry amplitudes are
determined for each period and the final asymmetry is determined as a period
weighted average as in Eq.~\ref{equ::wAvg}.  For the same reason as the previous
analyses and explained in Sec~\ref{sec::doubleratio_results}, the polarization
and virtual photon depolarization factors from each period are used to correct
the asymmetry amplitude determined in each period.  The final results are shown
in Fig.~\ref{fig::wA_results} along with the results from the release values.
As can be seen the results agree with those results obtained for the release
which was a requirement before the results could be released to the
public~\cite{Matousek:2018qqd}.

\begin{figure}[h!t]
  \centering \includegraphics[width=\textwidth,trim=0.2cm 1.5cm 0.2cm 1.5cm,
    clip]{wA_results}
  \caption{The comparison of weighted asymmetry amplitude results from the
    released values from Jan Matousek (black) and the cross checker Robert Heitz
    (red).  From the top row down the asymmetry amplitudes are
    $A_T^{\sin(\phi_S) q_T/M_N}$, $A_T^{\sin(2\phi+\phi_S)
      q^3_T/(2M_{\pi}M_N^2)}$ and $A_T^{\sin(2\phi-\phi_S) q_T/M_{\pi}}$
    respectively.}
  \label{fig::wA_results}
\end{figure}


\section{Left-Right Asymmetries} \label{sec::leftrightasym}

This section goes over the analysis details for measuring the left-right
asymmetry from the transversely polarized Drell-Yan data.  A theoretical
introduction showing how the left-right asymmetry is related to the Sivers TMD
PDF and related past results for this asymmetry are given in
Sec~\ref{sec::lr_theory}.  In short the measured asymmetry can be defined as
\begin{equation}
  A_{lr} = \frac{1}{|S_T|}
  \frac{\sigma_l - \sigma_r}{\sigma_l +
    \sigma_r},
\end{equation}
\noindent
which when assuming the leading order Drell-Yan cross-section,
Eq.~\ref{equ::DY_MostusefulXsect}, is related to the Sivers asymmetry amplitude
as
\begin{equation}
  A_{lr} = \frac{2A_T^{\sin(\phi_S)}}{\pi}.
\end{equation}

There are many ways to determine the left-right asymmetry.  The relevant
techniques for the 2015 COMPASS setup are described and compared to ensure
confidence of the end results.  

\subsection{Asymmetry Extractions}
\subsubsection{Geometric Mean} \label{sec::GeoMean}
The most basic method to determine the left-right asymmetry is

\begin{equation}
  \label{equ::simpleAN}
  A_{lr,simple} = \frac{1}{|S_T|}
    \frac{N_l - N_r}{N_l + N_r},
\end{equation}

\noindent
where $N_l = \int_{\phi_S=0}^{\phi_S=\pi}N(\phi_S)d\phi_S$ denotes the counts
measured left and $N_r = \int_{\phi_S=\pi}^{\phi_S=2\pi}N(\phi_S)d\phi_S$
denotes the counts measured right.  Eq.~\ref{equ::simpleAN} can be used to
determine the left-right asymmetry per target cell.  An intuitive picture of
left and right defined in the target frame is shown in
Fig.~\ref{fig::leftright}.  This simple method to determine the left-right
asymmetry is intuitive and can be helpful for visualizing the forthcoming
methods to determine $A_{lr}$.  Left and right in all definitions in this
section are determined relative to the target spin direction as

\begin{equation}
  \label{equ::Defleftright}
  \begin{aligned}
    &\text{Left}: \hat{q}_T \cdot (\hat{S}_T \times \hat{P}_{\pi}) > 0 \\
    &\text{Right}: \hat{q}_T \cdot (\hat{S}_T \times \hat{P}_{\pi}) < 0, 
  \end{aligned}
\end{equation}

\noindent
where $\hat{q}_T$, $\hat{S}_T$ and $\hat{P}_{\pi}$ are unit vectors in the
target reference frame for the virtual photon transverse momentum, the target
spin and the beam pion momentum respectively.

\begin{figure}[h!t]
  \centering
  \includegraphics[width=0.6\textwidth, trim=7cm 6cm 7cm 7cm,clip]{leftright}
  \caption{The definition of the left plane (red) and right plane (green)
    defined from a target spin up configuration in the target frame}
  \label{fig::leftright}
\end{figure}

The simple definition of the left-right asymmetry, Eq.~\ref{equ::simpleAN}, is
unfortunately dependent on the spectrometer acceptance.  This is can be realized
from the fact that the definition of the detected counts,
Eq.~\ref{equ::countsdef}, depends on the spectrometer acceptance $a(\phi_S)$
which therefore means $A_{lr,simple}$ also depends on the spectrometer
acceptance.  This is a problem because the spectrometer acceptance can change
with time and space and therefore can be dependent on the physical kinematics
which produced the event.  Such dependencies can cause unphysical false
asymmetries in the measurement of $A_{lr}$ and must therefore be removed or must
be included as systematic effects. \par

Forming the geometric mean asymmetry is a way to determine the left-right
asymmetry without acceptance effects from the spectrometer.  The geometric mean
asymmetry is defined as
\begin{equation}
  \label{equ::ANgeomean}
  A_{lr,geo} =
  \frac{1}{|S_T|}
  \frac{\sqrt{N_l^{\uparrow}N_r^{\downarrow}}
    - \sqrt{N_r^{\uparrow}N_r^{\downarrow}}
  }{
    \sqrt{N_l^{\uparrow}N_l^{\downarrow}}
    + \sqrt{N_r^{\uparrow}N_r^{\downarrow}} },
\end{equation}

\noindent
which also defines a left-right asymmetry per target cell.  It is not difficult
to simplify the geometric mean asymmetry as
\begin{equation}
  \label{equ::ANgeomean_expand}
  A_{lr,geo} = \frac{1}{|S_T|}\frac{\kappa_{geo}
    \sqrt{\sigma_l\sigma_l} -
    \sqrt{\sigma_r\sigma_r}}{\kappa_{geo}
    \sqrt{\sigma_l\sigma_l} + \sqrt{\sigma_r\sigma_r}}
  = \frac{1}{|S_T|}\frac{\kappa_{geo}\sigma_l - \sigma_r}{
    \kappa_{geo}\sigma_l + \sigma_r},
\end{equation}

\noindent
where $\kappa_{geo}$ is a ratio of acceptances defined for the geometric mean as
\begin{equation}
  \kappa_{geo} =
  \frac{
    \sqrt{a^\uparrow_J a^\downarrow_S}
    }{
    \sqrt{a^\uparrow_S a^\downarrow_J}
  },
  \label{equ::accGeoMean}
\end{equation}

\noindent
where $J$ stands for the Jura spectrometer side and $S$ stands for Saleve
spectrometer side which are the west and east sides respectively.  The
assumption made for the notation in Eq.~\ref{equ::accGeoMean}, which will be
made throughout this section, is that the target is polarized exactly vertical
in the target frame.  If this assumption is violated, the Jura and Saleve
acceptances blend into each other and Eq.~\ref{equ::accGeoMean} is no longer the
correct notation for the acceptance ratio.  The assumption is violated when the
trajectories of the beam particle and the target polarization do not make a
right angle in the laboratory frame, in which case the target will no longer be
polarized vertically in the target frame.  However the target will be assumed to
be vertically polarized in the target frame strictly for ease of notation.

Relation~\ref{equ::ANgeomean_expand} is equal to $A_{lr}$ if $\kappa_{geo} = 1$.
However as stated previously, time effects can vary $\kappa_{geo}$ from
unity. These effects are estimated through false asymmetry analysis and included
in the systematic uncertainties described in Sec~\ref{sec::systematics}.
Equation~\ref{equ::ANgeomean} is therefore to a good approximation an acceptance
free method to determine $A_{lr}$.  It is also defined for the upstream and
downstream cells independently and therefore can be used as a consistency check
between the two target cells.

The statistical uncertainty of the geometry mean is
\begin{equation}
  \delta A_{lr,geo} = \frac{1}{|S_T|}
  \frac{
    \sqrt{
      N_{l}^{\uparrow}N_{l}^{\downarrow}
      N_{\mathrm{r}}^{\uparrow}N_{r}^{\downarrow}
    }
  }{
    \Big( \sqrt{N_{l}^{\uparrow}N_{l}^{\downarrow}} +
    \sqrt{N_{r}^{\uparrow}N_{r}^{\downarrow}} \Big)^2
  }
  \sqrt{
    \frac{1}{N_{l}^{\uparrow}} +
    \frac{1}{N_{l}^{\downarrow}} +
    \frac{1}{N_{r}^{\uparrow}} +
    \frac{1}{N_{r}^{\downarrow}}
  } \quad,
\end{equation}

\noindent
which reduces to $\frac{1}{|S_T|}\frac{1}{\sqrt{N}}$, where $N = N^\uparrow_l +
N^\downarrow_l + N^\uparrow_r + N^\downarrow_r$, in the case of equal statistics
per target cell meaning $N^\uparrow_l = N^\downarrow_l = N^\uparrow_r =
N^\downarrow_r = N/4$.

\subsubsection{Two-Target Geometric Mean} \label{sec::TwoTargGeoMean}
The previous geometric mean asymmetry determined an $A_{lr}$ per target cell.
As described in Sec~\ref{sec::datasample} however, COMPASS had two oppositely
polarized target cells in 2015.  It is desirable from a statistical point of
view and for comparison purposes to determine one $A_{lr}$ from the 2015 COMPASS
setup.  This can be accomplished by modifying the geometric mean to add both
target cells as follows:

\begin{equation}
  \label{equ::AN4TargGeomean}
  A_{lr,2Targ} =
  \frac{1}{|S_T|}
  \frac{ \sqrt[4]{ N_{1,l}^\uparrow N_{1, l}^\downarrow
      N_{2,l}^\uparrow N_{2, l}^\downarrow }
    - \sqrt[4]{ N_{1,r}^\uparrow N_{1,r}^\downarrow
      N_{2,r}^\uparrow N_{2,r}^\downarrow }
  }{
    \sqrt[4]{ N_{1,l}^\uparrow N_{1, l}^\downarrow
      N_{2,l}^\uparrow N_{2, l}^\downarrow }
    + \sqrt[4]{ N_{1,r}^\uparrow N_{1,r}^\downarrow
      N_{2,r}^\uparrow N_{2,r}^\downarrow } }.
\end{equation}

As in the basic geometric mean asymmetry, Sec~\ref{sec::GeoMean}, left and right
are determined relative to the spin direction of the target as in
Eq.~\ref{equ::Defleftright}.  Similarly to Eq.~\ref{equ::ANgeomean_expand}, the
two-target geometric mean asymmetry can be written as
\begin{equation}
  A_{lr,2Targ}= \frac{1}{|S_T|}
  \frac{
    \kappa_{2Targ} \sqrt[4]{\sigma_l\sigma_l\sigma_l\sigma_l} -
    \sqrt[4]{\sigma_r\sigma_r\sigma_r\sigma_r}
  }{
    \kappa_{2Targ} \sqrt[4]{\sigma_l\sigma_l\sigma_l\sigma_l} +
    \sqrt[4]{\sigma_r\sigma_r\sigma_r\sigma_r}
  }
  =
  \frac{1}{|S_T|}
  \frac{\kappa_{2Targ}\sigma_l - \sigma_r}{
    \kappa_{2Targ}\sigma_l + \sigma_r},
  \label{equ::AN2targAcceptCancel}
\end{equation},

\noindent
where now $\kappa_{2Targ}$ is the ratio of acceptances from all targets and
polarizations.  This inclusive acceptance ratio is defined as
\begin{equation}
  \label{equ::acc4TargGeoMean}
  \kappa_{2Targ} =
  \frac{
    \sqrt[4]{
      a^\uparrow_{1,J}
      a^\downarrow_{1,S}
      a^\uparrow_{2,J}
      a^\downarrow_{2,S}}
  }{
    \sqrt[4]{
      a^\uparrow_{1,S}
      a^\downarrow_{1,J}
      a^\uparrow_{2,S}
      a^\downarrow_{2,J}}
  }.
\end{equation}

\noindent
In this case the acceptance ratio is expected to vary less with time and
therefore be closer to unity than the normal geometric mean acceptance ratio,
Eq.~\ref{equ::accGeoMean}.  This is a consequence of having the different target
cells oppositely polarized.  Rewriting Eq.~\ref{equ::acc4TargGeoMean} with
sub-period superscripts instead of target polarization superscripts results in
the relation

\begin{equation}
  \label{equ::acc4TargGeoMean_subperiod}
  \kappa_{2Targ} =
  \frac{ \sqrt[4]{ a^{one}_{1,J} a^{two}_{1,S} a^{two}_{2,J} a^{one}_{2,S} }
  }{
    \sqrt[4]{ a^{one}_{1,S} a^{two}_{1,J} a^{two}_{2,S} a^{one}_{2,J} }
  }
\end{equation}

\noindent
where sub-period $one$ is with the upstream target polarized up and the
downstream target polarized down and vise versa for sub-period $two$.  From
Eq.~\ref{equ::acc4TargGeoMean_subperiod} it is more evident that the acceptance
ratio terms for sub-period $two$ are reciprocal to the terms for sub-period
$one$ and therefore the acceptance ratio is expected to be more stably close to
unity.

Finally the statistical uncertainty of the two target geometric mean is
\begin{equation}
  \delta A_{lr,2Targ} = \frac{1}{|S_T|}
  \frac{LR}{\Big( L+R \Big)^2}
  \sqrt{
    \sum_{c,p}
    \Big(
    \frac{1}{N_{c,l}^{p}}
    + \frac{1}{N_{c,r}^p}
    \Big)
  } \quad,
\end{equation}

\noindent
where $L$ can be thought of as the left counts and equals to
$\sqrt[4]{N_{1,l}^\uparrow N_{1,l}^\downarrow N_{2,l}^\uparrow
  N_{2,l}^\downarrow}$ and $R$ can be thought of as the right counts and equals
$\sqrt[4]{N_{1,r}^\uparrow N_{1,r}^\downarrow N_{2,r}^\uparrow
  N_{2,r}^\downarrow}$.  As with the geometric mean asymmetry, in the case of
equal statistic populations in each direction and target polarization, the
statistical uncertainty for the two-target geometric mean also reduces to
$\frac{1}{|S_T|}\frac{1}{\sqrt{N}}$, where $N$ is the sum of all counts.


\subsection{Systematic Studies} \label{sec::systematics}
Several tests were performed to estimate the systematic uncertainty of the
left-right asymmetry.  The systematic uncertainties are determined by adding all
non-zero systematic uncertainties in quadrature.  The impact from each source of
systematic error is summarized in Table.~\ref{tab::sysError}.

\subsubsection{Period Compatibility (Time Dependence)}\label{sec::sysPulls}
The asymmetries calculated for each time period in each kinematic bin are shown
in Fig.~\ref{fig::allPhysBinned4Targ}.

\begin{figure}[h!t]
  \begin{center}
    \includegraphics[width=\textwidth,trim=0cm 5cm 0cm
      5cm,clip]{allPhysBinned4Targ}
    \caption{$A_{lr}$ determined for each period}
    \label{fig::allPhysBinned4Targ}
  \end{center}
\end{figure}

\noindent
By eye the asymmetry fluctuations appear to be statistically compatible.  To
quantify the compatibility of the asymmetries between the periods, a pull
distribution is formed where the pull value is defined as

\begin{equation}
  \label{eq::pull}
  \Delta\mathrm{A}_i =
  \frac{
    \mathrm{A}_i - \langle \mathrm{A} \rangle
  }{
    \sqrt{
      \sigma^2_{\mathrm{A}_i} - \sigma^2_{\langle \mathrm{A} \rangle}
    }
  },
\end{equation}

\noindent
and is determined for each period and kinematic bin.  There are therefore 3
(number of bins) x 5 (number of kinematics) x 9 (number of periods) = 135
entries in the pull distribution. The pull distribution is shown in
Fig.~\ref{fig::pull4Targ} along with a Gaussian fit to determine the
distributions width and average.  If the asymmetries all come from the same
parent distribution then, due to the central limit theorem, the pull
distribution will be a Gaussian distribution with zero mean and unit variance.
The discrepancy of the pull distribution from a standard Gaussian distribution
is used to determine a systematic error as

\begin{equation}
  \label{equ::sysErrorPull}
  \frac{\sigma_{\mathrm{systematic}}}{\sigma_{\mathrm{statistical}}} =
  \sqrt{|\sigma^2_{\mathrm{pull}} - 1|} + \frac{\mu_{\mathrm{pull}}}{2}.
\end{equation}

\begin{figure}[h!t]
  \begin{center}
    \includegraphics[width=0.6\textwidth, trim=2cm 2cm 2cm 2cm,clip]{pull4Targ}
    \caption{Pull distribution from the two-target geometric mean}
    \label{fig::pull4Targ}
  \end{center}
\end{figure}

\noindent
As the asymmetries in different kinematic bins are formed using the same data
set, the asymmetries between kinematic binnings are correlated.  For this reason
an uncorrelated pull distribution is also formed for each kinematic bin and also
compared with a standard Gaussian distribution.  These distributions are shown
in Fig.~\ref{fig::allPhysPulls4Targ_fit} along with the results of their
respective Gaussian fits.  For these uncorrelated pull distributions there are
now only 3 (number of bins) x 9(number of periods) = 27 entries in each
kinetically binned pull distributions and only 9 (number of periods) bins in the
integrated pull distribution.

\begin{figure}[h!t]
  \begin{center}
    \includegraphics[width=\textwidth, trim=2.5cm 2.5cm 2.5cm 2.5cm,
      clip]{allPhysPulls4Targ_fit}
    \caption{Uncorrelated pull distributions and Results of the Gaussian fit for
      the pull distributions}
    \label{fig::allPhysPulls4Targ_fit}
  \end{center}
\end{figure}

Even though the Gaussian fits did not give exactly a standard Gaussian, the fit
parameters are well compatible with a standard Gaussian within the errors of the
fit.  Therefore no systematic error was assigned due to incompatibility of the
periods.

\subsubsection{Left/Right Event Migration} \label{sec::syslrEventMigration}
The spectrometer has finite resolution for any measured quantity.  For this
reason events measured as left outgoing could really be events that are right
outgoing and vise versa.  This left-right misidentification has the result of
diluting spin-dependent effects by effectively having a sample from an
unpolarized target along with the sample from the polarized target.  Therefore
the asymmetry $A_{lr}$ reduces from left-right misidentification and this
effect is included as a systematic effect. 

A Monte-Carlo data set was analyzed to determine the left-right
misidentification percentage.  Four Monte-Carlo processes were generated
corresponding to three background processes and a spin-independent signal
process.  The generator used was PYTHIA8 and the data was generated and
reconstructed on NCSA's Blue Waters at Urbana-Champaign.  The background
processes simulated were JPsi production, Psi' production and open charm (OC)
production.  Each of these backgrounds can decay into two final state muons
which results in a background contamination to the Drell-Yan signal.
Table~\ref{tab::MCproduction} gives the parameters used for the Monte-Carlo
generated.

\begin{table}[h!t]
  \centering
  \caption{Monte-Carlo settings produced on Blue Waters}
  \label{tab::MCproduction}
  \begin{tabular}{ |c|c| }
    \hline
    \textbf{Description}& \textbf{Monte-Carlo Setting Setting} \\ \hline
    \hline
    Event generator& PYTHIA8\\
    \hline

    Pion PDF& GRVPI1\\
    \hline

    Proton PDF& NNPDF23\\
    \hline
    
    Proton/Neutron mixing ratio& 1.96\\
    \hline

    Initial state radiation& on\\
    \hline
    
    Final state radiation& on\\
    \hline
    
    Multiple parton interactions& on\\
    \hline

    GEANT4 detector simulation& TGEANT \\
    \hline

    Simulated detector efficiency distributions& uniform\\
    \hline
    
  \end{tabular}
\end{table}

The generated Monte-Carlo data corresponds to a 4$\pi$ spectrometer acceptance.
The COMPASS spectrometer on the other hand, does not have a full 4$\pi$
acceptance.  Therefore to produce simulated data that corresponds to the actual
data taking conditions, a GEANT4~\cite{AGOSTINELLI2003250} based detector
simulation, called TGEANT~\cite{TGEANTthesis}, simulated the COMPASS
spectrometer response to the generated data.  The data from TGEANT was then
reconstructed with the same reconstruction software as real data.

Misidentification was estimated by comparing the data input to TGEANT with the
output reconstructed data.  The same analysis and cuts were performed on the
simulated and then reconstructed data as were performed on the real data set.
Fig.~\ref{fig::lrMigration} shows the rate of events identified correctly and
incorrectly as a function of $\phi_S$.  Fig.~\ref{fig::lrMigration} is made by
comparing which outgoing direction the generated events emerged with the
outgoing direction the reconstructed events emerged.

\begin{figure}[h!t]
  \centering
  \includegraphics[width=0.6\textwidth,trim=3cm 4cm 3cm 6cm, clip]{lrMigration}
  \caption{The rate of identified correctly and incorrectly left-right events as
    a function of $\phi_{S}$.  This is determined by comparing the generated
    outgoing direction with the reconstructed outgoing direction.  The
    left-right boundary is clearing visible at $\phi_{S}$ = 0~rad and $\phi_{S}$
    = -$\pi$~rad and $\phi_{S}$ = $\pi$~rad}.
  \label{fig::lrMigration}
\end{figure}

\noindent
As is clearly visible in Fig.~\ref{fig::lrMigration}, there is a band of higher
misidentification rate at the border between left and right.  For this reason a
cut on the $\phi_{S}$ variable symmetric about the left-right border was tested
to determine the percent of misidentification as a function of the amount of
$\phi_{S}$ cut.  These results are shown in Fig.~\ref{fig::percentLRmiss}.

\begin{figure}[h!t]
  \centering \includegraphics[width=0.5\textwidth, trim=4cm 10cm 4cm 10cm,
    clip]{percentLRmiss}
  \caption{Percent left-right migration as a function of the amount of
    $\phi_{S}$ cut.}
    \label{fig::percentLRmiss}
\end{figure}

The systematic error for left-right migration is calculated as

\begin{equation}
  \delta A_{lr,systematic} = \gamma *A_{lr} + \gamma *\delta A_{lr},
\end{equation}

\noindent
where this expression is derived in Appendix~\ref{app::sysLRmiss}.  No cut on
$\phi_{S}$ was ultimately used for the asymmetry determination.  This is to
avoid loss of statistics and due to the fact that the systematic error is
already small with no cut in $\phi_{S}$.  The integrated systematic error due to
left-right event migration was determined to be 9\% of the statistical error.


\subsubsection{False Asymmetries}
\subsubsection{Acceptance From False Asymmetries}
As was pointed out in Sec.~\ref{sec::GeoMean} and
Sec.~\ref{sec::TwoTargGeoMean}, the asymmetry measurement assumes the acceptance
does not change with time and therefore the acceptance ratio, $\kappa$ is
unitary.  Any deviation from a unitary acceptance ratio is estimated with a
false asymmetry and is taken as a systematic error.  To determine if acceptance
does change with time, a false asymmetry is calculated where the only way the
false asymmetry could be non-zero is if acceptance changes with time.  This
false asymmetry for the two-target geometric mean is

\begin{align}
  \label{equ::falseAcc}
  A_{lr,False} &= 
    \frac{1}{|S_T|}
    \frac{
      \sqrt[4]{
        N_{1,r}^\uparrow N_{1, l}^\downarrow
        N_{2,l}^\uparrow N_{2, r}^\downarrow
      } 
      -\sqrt[4]{
        N_{1,l}^\uparrow N_{1, r}^\downarrow
        N_{2,r}^\uparrow N_{2, l}^\downarrow
      }
    }{
      \sqrt[4]{
        N_{1,r}^\uparrow N_{1, l}^\downarrow
        N_{2,l}^\uparrow N_{2, r}^\downarrow
      } +
      \sqrt[4]{
        N_{1,l}^\uparrow N_{1, r}^\downarrow
        N_{2,r}^\uparrow N_{2, l}^\downarrow
      }
    } \\ \nonumber
    & =
    \frac{1}{|S_T|}
    \frac{
      \alpha_{2Targ} \sqrt[4]{\sigma_r\sigma_l\sigma_l\sigma_r} -
      \sqrt[4]{\sigma_l\sigma_r\sigma_r\sigma_l}
    }{
      \alpha_{2Targ} \sqrt[4]{\sigma_r\sigma_l\sigma_l\sigma_r} +
      \sqrt[4]{\sigma_l\sigma_r\sigma_r\sigma_l}
    }
    = \frac{1}{|S_T|}
    \frac{
      \alpha_{2Targ} - 1     
    }{
      \alpha_{2Targ} + 1
    },
\end{align}

\noindent
where $\alpha_{2Targ}$ is an acceptance ratio and is defined as

\begin{equation} \label{equ::alphaAcc}
  \alpha_{2Targ} =
  \frac{ \sqrt[4]{ a^\uparrow_{1,S}a^\downarrow_{1,S}
      a^\uparrow_{2,J}a^\downarrow_{2,J}}
  }{
    \sqrt[4]{
      a^\uparrow_{1,J} a^\downarrow_{1,J}
      a^\uparrow_{2,S} a^\downarrow_{2,S}
    }
  }.
\end{equation}

\noindent
The false asymmetry, Eq.~\ref{equ::falseAcc}, can be simplified as

\begin{equation}
  A_{lr,False} = 
  \frac{1}{|S_T|}
  \frac{
    \sqrt[4]{ N_{1, S} N_{2, J} }
    - \sqrt[4]{ N_{1, J} N_{2, S} }
  }{
    \sqrt[4]{ N_{1, S} N_{2, J} }
    + \sqrt[4]{ N_{1, J} N_{2, S} }
  }.
\end{equation}

\noindent
That is $A_{lr,False}$ is the normalized difference of counts from each
target cell assuming the upstream target is always polarized down and the
downstream target is always polarized up.  Given that the polarization flips for
both upstream and downstream target cells, $A_{lr,False}$ is an
asymmetry where physical effects cancel out.  The kinematic dependencies of the
false asymmetry are shown in Fig.~\ref{fig::falseAacc} and the kinematic
dependencies of the acceptance ratio, $\alpha_{2Targ}$, are shown in
Fig.~\ref{fig::alpha}.

\begin{figure}[h!t]
  \begin{center}
    \includegraphics[width=\textwidth, trim=0cm 6.5cm 0cm 6.5cm,
      clip]{falseAacc}
    \caption{False asymmetry, $A_{lr,False}$, to estimate fluctuations in
      acceptance in time}
    \label{fig::falseAacc}
  \end{center}
\end{figure}

\begin{figure}[h!t]
  \begin{center}
    \includegraphics[width=\textwidth, trim=0cm 6cm 0cm 6cm,
      clip]{alpha4Targ}
    \caption{Acceptance ratio $alpha_{2Targ}$, Eq.~\ref{equ::alphaAcc}, used to
      determine the systematic effects from acceptance changes in time}
    \label{fig::alpha}
  \end{center}
\end{figure}

While $\alpha_{2Targ}$ is an acceptance ratio it is not the same as,
$\kappa_{2Targ}$ the acceptance ratio in the true asymmetry.  However
$\alpha_{2Targ}$ is similar to $\kappa_{2Targ}$ in that $\alpha_{2Targ}$ will
only be different from unity as a result of time changes in the spectrometer.
Therefore it is assumed $\alpha_{2Targ}$ can be used as a good estimate of the
true acceptance ratio fluctuations.  The systematic error due to acceptance
fluctuations is determined as

\begin{equation}
  \delta A_{lr,systematic} =
  \frac{1}{|S_T|}
  \Big(\frac{|\alpha_{2Targ}-1|}{2}
  + \delta_{\frac{|\alpha_{2Targ}-1|}{2}} \Big),
\end{equation}

\noindent
where this expression is derived in Appendix~\ref{app::sysAcc}.  The kinematic
dependence of the systematic error normalized to the statistical error is shown
in Fig.~\ref{fig::accSysStat}.  The binned average systematic error due to
acceptance is 20\% of the statistical error.

\begin{figure}[h!t]
  \begin{center}
    \includegraphics[width=\textwidth, trim=0cm 5cm 0cm 5cm,
      clip]{accSysStat}
    \caption{Systematic uncertainty due to acceptance effects normalized to the
      statistical error.}
    \label{fig::accSysStat}
  \end{center}
\end{figure}

\subsubsection{Further False Asymmetry Effects}
Although the list of systematic effects specifically studied is quite exhaustive
there is always the potential for other systematic effects not considered.
Studies of the changes in time from additional false asymmetries were performed
in an attempt to take into account all other systematic effects.  All false
asymmetries considered must be constructed in such a way that the physical
process of interest cancels out.  A false asymmetry could therefore only be
non-zero from acceptance effects, luminosity or some other reason not
considered.  The additional false asymmetries are constructed in a way that
luminosity effects cancel out and acceptance effects are approximately constant.
With these assumptions, the pull values from Eq.~\ref{eq::pull} are expected to
be distributed as a standard Gaussian distribution.  Any deviation from a
standard Gaussian is conservatively taken as a systematic effect from some
unknown cause.  The additional studied false asymmetries are summarized in the
following enumerated list.

\begin{enumerate}
  \label{tab::additionalFA}

\item A false asymmetry similar to Eq.~\ref{equ::falseAcc} but with the upstream
  left and right counts flipped defined as
  
  \begin{equation}
    \label{equ::additionalfalseAsym}
    A_{lr, F1} = \frac{1}{|S_T|}
    \frac{
      \sqrt[4]{
        N_{1,l}^\uparrow N_{1,r}^\downarrow N_{2,l}^\uparrow N_{2,r}^\downarrow
      }
      - \sqrt[4]{ N_{1,r}^\uparrow N_{1,l}^\downarrow
        N_{2,r}^\uparrow N_{2, l}^\downarrow }
      }{
      \sqrt[4]{
        N_{1,l}^\uparrow N_{1,r}^\downarrow
        N_{2,l}^\uparrow N_{2, r}^\downarrow
      } + \sqrt[4]{ N_{1,r}^\uparrow N_{1,l}^\downarrow
        N_{2,r}^\uparrow N_{2, l}^\downarrow
      }
    }.
  \end{equation}
  This false asymmetry can be thought of as measuring the normalized counts on
  the Jura side minus the Saleve side.  The period weighted average results of
  this false asymmetry are shown in Fig.~\ref{fig::fa2TargJuraSaleve}.  As
  Fig.~\ref{fig::fa2TargJuraSaleve} shows, the asymmetry is systematically less
  than zero by more than a standard deviation resulting from acceptance effects.
  The uncorrelated pull distributions from this false asymmetry are shown in
  Fig.~\ref{fig::fa2TargJSPulls} along with the corresponding Gaussian fit
  results.  Due to the fact that there are less entries in these pull
  distributions the Gaussian fit results are not necessarily that good.  In an
  attempt to correct for this and to take into account the fit errors, a
  weighted average of the mean and standard deviation are made, as in
  Eq.~\ref{equ::wAvg}, using weights as the inverse fit variances.  The
  resulting systematic error is again determined as in
  Eq.~\ref{equ::sysErrorPull} using the weighted mean and weighted standard
  deviation.

  \begin{figure}[h!t]
    \centering \includegraphics[width=0.9\textwidth, trim=1.5cm 4.5cm 1.5cm
      4.5cm, clip]{fa2TargJuraSaleve}
    \caption{Two-target geometric mean false asymmetry.  This is non-zero due to
      acceptance effects}
    \label{fig::fa2TargJuraSaleve}
  \end{figure}
  
  \begin{figure}[h!t]
    \centering \includegraphics[width=0.9\textwidth, trim=1.5cm 2.5cm 1.5cm
      2.5cm, clip]{fa2TargJSPulls}
    \caption{Uncorrelated pulls of the two-target geometric mean false asymmetry
      and Gaussian fit results}
    \label{fig::fa2TargJSPulls}
  \end{figure}

\item A false asymmetries using only the information from the upstream or the
  downstream target defined as

  \begin{equation}
    \label{equ::falseANgeomean}
    A_{lr, F2} =
    \frac{1}{|S_T|}
    \frac{\sqrt{N_l^\uparrow N_r^\downarrow}
      - \sqrt{N_r^\uparrow N_l^\downarrow}
    }{
      \sqrt{N_l^\uparrow N_r^\downarrow}
      + \sqrt{N_r^\uparrow N_l^\downarrow}
    }.
  \end{equation}
  This false asymmetry can also be thought of as measuring the normalized counts
  on the Jura side minus the Saleve side but for each target individually.  Both
  this false asymmetry and the previous false asymmetry,
  Eq.~\ref{equ::additionalfalseAsym}, can be written as
  \begin{equation}
    A_{lr,F1/2} =
    \frac{1}{|S_T|}
    \frac{\alpha - 1}{\alpha + 1},
  \end{equation}
  where $\alpha$ will be an acceptance ratio of Jura/Saleve.  As the Jura/Saleve
  acceptance ratio is expected to be the same for the upstream and downstream
  targets, any difference between the two false asymmetries must be due to other
  reasons.  A by-period comparison between the upstream and downstream target is
  shown in Fig.~\ref{fig::alphaAsymPeriod} and as can be seen there are
  difference by-period between the upstream and downstream asymmetries.  A
  combined pull distribution is made using the information from both upstream
  and downstream asymmetries and is shown in Fig.~\ref{fig::alphaAsymPull}.  As
  with the previous false asymmetry, lack of data leads to the same problems
  with the fit and therefore the same weighting method is used to determine a
  systematic error.

  \begin{figure}[h!t]
    \centering \includegraphics[width=\textwidth, trim=2cm 4cm 2cm 4cm,
      clip]{alphaAsymPeriod}
    \caption{One target false asymmetries for the upstream target (red) and the
      downstream target (blue), as a function of x$_{\mathrm{N}}$.  Each graph
      is from a different period in time.}
    \label{fig::alphaAsymPeriod}
  \end{figure}

  \begin{figure}[h!t]
    \centering
    \includegraphics[width=\textwidth]{alphaAsymPull}
    \caption{Pull values from one-target geometric mean false asymmetries.  Both
      upstream and downstream values are used to make this pull}
    \label{fig::alphaAsymPull}
  \end{figure}

\item Finally the same false asymmetry used to determine the acceptance
  fluctuations, Eq.~\ref{equ::falseAcc}, is also checked for compatibility and a
  systematic error is determined in the same way as the previous false
  asymmetries.  The pulls are shown in Fig.~\ref{fig::fa2TargPulls} along with
  the corresponding fit parameters and errors.

  \begin{figure}[h!t]
    \centering \includegraphics[width=\textwidth, trim=2.5cm 6cm 2.5cm 6cm,
      clip]{fa2TargPulls}
    \caption{Pull distribution for a nearly acceptance free two-target false
      geometric mean asymmetry}
    \label{fig::fa2TargPulls}
  \end{figure}
  
\end{enumerate}

A summary of the systematic error from each false asymmetry is shown in
Table.~\ref{tab::faSys}.

\begin{table}[h!t]
  \centering
  \begin{tabular}{|c|c|}
    \hline Systematic error& \multirow{2}{9em}{$\langle
      \sigma_{\mathrm{systematic}}/\sigma_{\mathrm{statistical}}
      \rangle$}\\ & \\ \hline
    
    Two target Jura-Saleve& 0.26\\ \hline

    Combined one target& 0.5\\ \hline

    Two target acceptance estimation& 0.29\\ \hline
    
  \end{tabular}
  \caption{Summary of systematic error impacts from false asymmetries.  The
    maximum systematic error is chosen as the systematic error.}
  \label{tab::faSys}
\end{table}

\subsubsection{Total Systematics}
The total systematic uncertainty is determined by adding all non-zero systematic
effects in quadrature as

\begin{equation}
  \Big \langle \frac{
    \sigma_{\mathrm{systematic}}}{\sigma_{\mathrm{statistical}}} \Big \rangle =
  \sqrt{ \sum_i^{\mathrm{all \; systematic}} \Big \langle
    \frac{\sigma^2_{\mathrm{systematic, i}}}{\sigma^2_{\mathrm{statistical}}}
    \Big \rangle } \;,
\end{equation}
where all the systematic effects considered are summarized in
Table.~\ref{tab::sysError}.  For reference the integrated left-right asymmetry
is $\langle A_{lr} \rangle = 0.030$ and the integrated statistical error is
$\langle \sigma_{\mathrm{statistical}} \rangle$ = 0.039.

\begin{table}[h!t]
  \centering
  \begin{tabular}{|c|c|c|}
    \hline
    \multirow{2}{*}{Systematic Uncertainty}&
    \multirow{2}{*}{
      $\langle \sigma_{\mathrm{systematic}}/\sigma_{\mathrm{statistical}}
      \rangle$} &
    \multirow{2}{*}{$\langle \sigma_{\mathrm{systematic}} \rangle$}\\
    & & \\ \hline \hline

    Period compatibility& 0.0 & 0.0\\ \hline

    Left-Right migration& 0.09 & 0.004\\ \hline

    Target Polarization& 0.05& 0.003\\ \hline

    Dilution Factor& 0.05 & 0.003\\ \hline

    Acceptance fluctuation& 0.2 & 0.008\\ \hline

    False asymmetry& 0.5 & 0.020\\ \hline \hline
    \textbf{Total}& \textbf{0.55} & \textbf{0.022}\\\hline
    
  \end{tabular}
  \caption{Summary of systematic uncertainty impacts to the integrated
    asymmetry}
  \label{tab::sysError}
\end{table}

\subsection{Results} \label{sec::lr_results}
The left-right asymmetry is extracted per period, corrected for the target
polarization and ultimately combined as a weighed average, Eq.~\ref{equ::wAvg},
to get an overall result as in Sec~\ref{sec::doubleratio_results}.

The results for the geometric mean are shown in Fig.~\ref{fig::ANgeom} and the
results for the two-target geometric mean are shown in
Fig.~\ref{fig::AN4TargGeom}.  The numerical values for the two-target geometric
mean systematic uncertainty are summarized in Table~\ref{tab::sysError}.  To
verify the results in this section are correct, an independent cross-check was
performed by a member from the University of Turin and no discrepancies were
found.  The results of the cross-check are provided in
Appendix~\ref{app::xcheckAlr}.

\begin{figure}[h!t]
  \begin{center}
    \includegraphics[width=\textwidth, trim=1cm 5.5cm 1cm 5.5cm,
      clip]{ANgeom}
    \caption{$A_{lr}$ determined from the geometric mean method for the upstream
      target cell (red) and the downstream target cell (blue) for all kinematic
      binnings}
    \label{fig::ANgeom}
  \end{center}
\end{figure}

\begin{figure}[h!t]
  \begin{center}
    \includegraphics[width=\textwidth, trim=2cm 5cm 2cm 5cm,
      clip]{AN4TargGeom}
    \caption{$A_{lr}$ determined by the two-target geometric mean method
      for all kinematic binnings}
    \label{fig::AN4TargGeom}
  \end{center}
\end{figure}

It was shown in Sec~\ref{sec::lr_theory} that the left-right asymmetry is
related to the Sivers amplitude as

\begin{equation}
  \label{equ::all_an_convert}
  A_{lr} = \frac{2A^{\sin(\phi_S)}}{\pi} = \frac{2A_N}{\pi},
\end{equation}

\noindent
where $A_N$ is the analyzing power.  The Sivers amplitude was measured to be
approximately 1 sigma above zero from the unbinned maximum likelihood method,
Fig.~\ref{fig::DY_Siv_signFlip}, the double ratio method,
Fig.~\ref{fig::dr_final_results}, and the left-right asymmetry.  Adjusting the
left-right asymmetry, as in Eq.~\ref{equ::all_an_convert}, shows the amplitude
determined from the left-right asymmetry is statistically consistent with the
Sivers amplitude determined from the double ratio method,
Fig.~\ref{fig::AN_DR_comp}.  Therefore all methods to determine the Siver
amplitude in this chapter are consistent with the sign flip hypothesis between
the Drell-Yan and SIDIS processes.  On the other hand, the statistical error
bars are too large to definitively conclude the sign flip assumption holds.
There are also no clear trends in the kinematic variable binning due to the
large statistical error bars.

It is interesting to note that Eq.~\ref{equ::all_an_convert} was derived with
the assumption that the leading order Drell-Yan cross-section,
Eq.~\ref{equ::DY_MostusefulXsect}, is sufficient.  It is not theoretically ruled
out however, that the left-right asymmetry results from higher order amplitudes
in addition to the Sivers amplitude.  As Fig.~\ref{fig::AN_DR_comp} shows, the
left-right asymmetry is slightly less significant above zero than the Sivers
amplitude determined from the double ratio method.  This could indicate the need
to include higher order terms in the Drell-Yan cross-section.

\begin{figure}[h!t]
  \centering \includegraphics[width=0.2\textwidth, trim=7.5cm 7.5cm 7.5cm
    7.5cm, clip]{AN_DR_comp}
  \caption{The left-right asymmetry adjusted (analyzing power, blue) to be
    compared with Sivers amplitude determined from the double ratio method
    (red).}
  \label{fig::AN_DR_comp}
\end{figure}

In regards to the sign flip, Fig.~\ref{fig::AN_sign_flip} shows that the Sivers
amplitude determined from the left-right asymmetry is compatible with the
results published from 2015 COMPASS data and is compatible with the sign flip
between Drell-Yan and SIDIS.

\begin{figure}[h!t]
  \centering
  \includegraphics[width=0.6\textwidth,trim=1cm 7cm 0cm 7cm,clip]{AN_sign_flip}
  \caption{Including the left-right asymmetry to show it's compatibility with
    the sign flip.}
  \label{fig::AN_sign_flip}
\end{figure}

\chapter{J/$\Psi$ Transverse Spin Dependent Phenomena}
\label{ch::jpsi}
\ifpdf
\graphicspath{{Chapters/JPsi/Figs/}}
\fi

This chapter describes the analysis performed in the intermediate invariant mass
range at COMPASS.  The biggest advantage about the intermediate mass range is an
increase in statistics.  The intermediate mass range for this thesis is defined
as 2.5-4.3~{\gvcw}.  The intermediate mass range contains approximately 1.6
times more events than the high mass range above 4.3~{\gvcw}.  On the other
hand, the intermediate mass range results from several reactions.  Nevertheless,
{\jp} production is by far the most dominant reaction and therefore the this
chapter assumes the results are from the {\jp} reaction.  A theoretical
introduction related to transverse {\jp} spin asymmetries is provided in
Sec~\ref{sec::theory_jpsi}.  As of yet, the exact mechanism for J/$\Psi$
production is unknown and therefore the exact {\jp} vertex coupling is unknown.
Nevertheless, the analysis techniques in this chapter contribute to the
transverse spin knowledge of J/$\Psi$ production.  The reaction of
interest is
\begin{equation}
  \pi^-(P_a) + P(P_b, S_T) \rightarrow J/\Psi + X \rightarrow \mu^-(\ell) +
  \mu^+(\ell') + X,
\end{equation}
\noindent
where the proton target, $P$, is transversely polarized with spin $S_T$.

The dimuon final state from {\jp} production is indistinguishable from the
Drell-Yan dimuon final state described the previous analyses in
chapter~\ref{ch::hmanalysis}.  For this reason, similar event selection and data
quality are used to study the dimuon production resulting from {\jp} decays.  On
the other hand, the {\jp} production has a higher background percentage from
Drell-Yan and other components to be taken into account.  The results presented
in this chapter are determined from the left-right asymmetry analysis as in
Sec~\ref{sec::leftrightasym}, where again the left-right asymmetry is defined as

\begin{equation}
  A_{lr} = \frac{1}{|S_T|}
  \frac{\sigma_l - \sigma_r}{\sigma_l +
    \sigma_r}.
\end{equation}

\subsection{Data Collection and Event Selection}
The data collection is described in Sec~\ref{sec::datacollection} and the data
stability tests are described in Sec~\ref{sec::stability}.  Both the Drell-Yan
analysis and the {\jp} production analysis study dimuon final states so the
spectrometer data taking conditions are the same.  In particular the measurement
in this chapter results from a 190~{\gvc} $\pi^-$ beam impinged on a
transversely polarized NH$_3$ target from the 2015 COMPASS spectrometer data
taking conditions.

The event selection in this chapter is similar to the event selection in the
previous chapter, Sec~\ref{sec::dy_eventselection}.  The cuts are chosen to
ensure two oppositely polarized muons are detected with a vertex in the
transversely polarized NH$_3$ target.  Sec~\ref{sec::dy_eventselection}
describes the cut selection and the reason for each cut.  The only event
selection difference, from Sec~\ref{sec::dy_eventselection}, is the selected
invariant mass.  The nominal {\jp} invariant mass and width are 3.096~{\gvcw}
and 92.9$\times10^{-6}$~{\gvcw} respectively~\cite{Tanabashi:2018oca}.
Therefore to ensure the events for this analysis result from {\jp} production,
the analysis invariant mass range should be where the {\jp} signal to background
is highest.  This is in contrast to the Drell-Yan analyses which required the
invariant mass to be above 4.3~{\gvcw}.

\subsubsection{{\jp} Invariant Mass Range}\label{sec::jpMassRange}
The COMPASS spectrometer has a finite mass resolution which therefore means the
events resulting from {\jp} production have an invariant mass spread larger than
the nominal {\jp} width.  For this reason the cut on invariant mass should be a
range much larger than the nominal {\jp} width.
Fig.~\ref{fig::DY_InvariantMassJPsi} shows the 2015 dimuon invariant mass
distribution and the other production components around the {\jp} invariant
mass.

\begin{figure}[h!t]
  \centering \includegraphics[width=0.45\textwidth,trim=6.5cm 6cm 6.5cm 5.5cm,
    clip]{DY_InvariantMassJPsi}
  \caption{The 2015 COMPASS invariant dimuon mass distribution and a fit to this
    data.  The data fit is from Monte-Carlo and combinatorial background
    analysis and is provided to show the background processes.  The shaded blue
    region shows the analysis mass range for the analysis in this chapter.  This
    image is taken from~\cite{compassDYpaper}.}
  \label{fig::DY_InvariantMassJPsi}
\end{figure}

As can be seen in Fig.~\ref{fig::DY_InvariantMassJPsi}, regardless of the
analysis mass range chosen, there are still events resulting from the other
background processes.  The normalized systematic variance resulting from
background processes is derived in Appendix~\ref{app::sysEventContam} as

\begin{equation}
  \frac{\sigma^2_{systematic}}{\sigma^2_{statistical}} = \frac{(1-p)^2}{p^2},
\end{equation}
\noindent
and the total variance from statistical fluctuations and background
contributions is
\begin{equation}
  \delta^2 A_{lr,J/\Psi} = \frac{(1-p)^2 + 1}{p^2}\sigma^2_{statistical},
  \label{equ::JPerrorTot}
\end{equation}
where $p$ is the {\jp} purity.  Therefore the analysis invariant mass range
should have a {\jp} purity as high as possible to reduce the systematic error
while still including as much data as possible to reduce the statistical error.
The total error however, is dominated by statistical error for any purity larger
than 50\%.  That being the case, Eq.~\ref{equ::JPerrorTot} is derived assuming
$(1-p)$ is small and therefore a desired purity of 90\% or greater was chosen to
safely ensure Eq.~\ref{equ::JPerrorTot} is valid.

The {\jp} purity as a function of mass range is determined from a Monte-Carlo
data set.  The same Monte-Carlo described in Table~\ref{tab::MCproduction} was
used to calculate the {\jp} purity.  In particular the processes included are
Drell-Yan production, open charm production, $\Psi$' production and {\jp}
production.  To determine the purity, the real data is fit using the Monte-Carlo
data which determines the counts from each process.  The Monte-Carlo fit is
accomplished by normalizing the invariant mass distribution from each
Monte-Carlo sample and then fitting using the sum of the four normalized
distributions to fit the real data.  The fit function is defined as
\begin{align}
  F_{J/\Psi \;Fit}(x) = &N_{J/\Psi}h_{J/\Psi}(x) + N_{Drell-Yan}h_{Drell-Yan}(x)
  \\ \nonumber
  &+ N_{Open\;Charm}h_{Open\;Charm}(x)+N_{\Psi'}h_{\Psi'}(x),
\end{align}
\noindent
where $h(x)$ represents the number of counts from the normalized histogram
distribution and the $N$'s are the fit parameters.  Fig.~\ref{fig::MC_NormInvM}
shows the four normalized Monte-Carlo invariant mass distributions and
Fig.~\ref{fig::MC_fitQt} shows the Monte-Carlo fit to the real data in one $q_T$
bin.

\begin{figure}[h!t]
  \centering \includegraphics[width=0.8\textwidth, trim=0.5cm 2cm 0.5cm 2cm,
    clip]{MC_NormInvM}
  \caption{The normalized invariant mass distributions from the four simulated
    Monte-Carlo processes.  These distributions are used to fit the real data.}
  \label{fig::MC_NormInvM}
\end{figure}

\begin{figure}[h!t]
  \centering \includegraphics[width=0.95\textwidth, trim=2.5cm 6.5cm 2.5cm
    6.5cm, clip]{MC_fitQt}
  \caption{An example of the Monte-Carlo fit to real data in one $q_T$ bin of
    the real data.  This fit is used to determine the distribution of {\jp}
    purity.}
  \label{fig::MC_fitQt}
\end{figure}

Once the fits are performed, the {\jp} purity in each invariant bin is
determined as

\begin{equation}
  p(x) = \frac{N_{J/\Psi}h_{J/\Psi}(x)}{N_{J/\Psi}h_{J/\Psi}(x) +
    N_{Drell-Yan}h_{Drell-Yan}(x)
    N_{Open\;Charm}h_{Open\;Charm}(x)+N_{\Psi'}h_{\Psi'}(x)}.
\end{equation}
\noindent
Table~\ref{tab::JPsiPurity} summarizes the {\jp} purity as a function of the
mass range.  For this analysis an invariant mass range of 2.87-3.38~{\gvcw} was
chosen to safely have a {\jp} purity of 90\% or greater.
Tables~\ref{tab::JPsiStats1}-~\ref{tab::JPsiStats2} summarizes the number of
events remaining after each cut.

\begin{table}
  \centering
  \begin{tabular}{ |c|c| }
    \hline
    \textbf{Mass range GeV/c$^2$}& \textbf{{\jp} Purity \%}
    \\ \hline \hline
    2.5-4.3& 79.7 $\pm$ 2.9 \\ \hline
    2.78-3.46& 88.9 $\pm$ 2.0 \\ \hline
    \textbf{2.87-3.38}& \textbf{91.3 $\pm$ 1.5} \\ \hline
    2.95-3.29& 92.9 $\pm$ 1.0 \\ \hline
    3.08-3.17& 94.0 $\pm$ 1.0 \\ \hline
    
  \end{tabular}
  \caption{{\jp} purity as a function of the invariant mass range}
  \label{tab::JPsiPurity}
\end{table}

\begin{table}[h!t]
  \begin{tabular}{ |c|c|c|c|c|c| }
    \hline \textbf{Cuts}& \textbf{W07}& \textbf{W08}& \textbf{W09}&
    \textbf{W10}& \textbf{W11} \\ \hline

    \multirow{3}{9em}{$\mu^-\mu^+$ 2-8.5 GeV/c$^2$ with a common best primary
      vertex}& & & & & \\
    & 1,573,372& 1,572,255& 1,620,593& 1,683,263& 2,598,485\\
    & & & & & \\ \hline
    
    Good Spills& 1,298,306& 1,223,877& 1,333,335& 1,374,620& 1,901,071
    \\ \hline
    
    0$<$ x$_{\pi}$ x$_N$ $<$1, -1$<$ x$_F$ $<$1& 1,298,278& 1,223,851&
    1,333,307& 1,374,599& 1,901,033 \\ \hline

    0.4$<$ q$_T$ $<$5(GeV/c)& 1,121,908& 1,056,835& 1,151,253& 1,187,125&
    1,641,463\\ \hline

    Z Vertex within NH$_3$& 314,965& 298,531& 324,413& 335,659& 465,172
    \\ \hline

    Vertex Radius $<$ 1.9cm& 308,278& 292,114& 317,985& 328,658& 455,580
    \\ \hline

    2.87$< M_{\mu\mu} <$3.38 (GeV/c$^2$)&
    170,041& 160,450& 174,696& 180,795& 250,921
    \\ \hline
    
  \end{tabular}
  \caption{Selected dimuon events for the first five data periods from the
    intermediate mass range analysis of 2015 COMPASS data}
  \label{tab::JPsiStats1}
\end{table}

\begin{table}[h!t]
  \begin{tabular}{ |c|c|c|c|c|c|c| }
    \hline \textbf{Cuts}& \textbf{W12}& \textbf{W13}& \textbf{W14}&
    \textbf{W15} & \textbf{WAll} & \textbf{Remaining} \\ \hline

    \multirow{3}{9em}{$\mu^-\mu^+$ 2-8.5 GeV/c$^2$ with a common best primary
      vertex}& & & & & & \\
    &1,932,425& 1,680,706& 1,094,525& 640,095& 14,395,719&
    100.00 \% \\ & & & & & & \\ \hline

    Good Spills& 1,659,030& 1,314,489& 982,131& 616,734& 11,703,593&
    81.3 \% \\ \hline

    0$<$ x$_{\pi}$ x$_N$ $<$1, -1$<$ x$_F$ $<$1&
    1,658,996& 1,314,470& 982,125& 616,720& 11,703,379&	81.3 \% \\ \hline

    0.4$<$ q$_T$ $<$5(GeV/c)&
    1,432,115& 1,134,223& 846,897& 532,045& 10,103,864&	70.2 \% \\ \hline

    Z Vertex within NH$_3$&
    406,975& 322,964& 241,673& 151,937&	2,862,289& 19.9  \% \\ \hline

    Vertex Radius $<$ 1.9cm&
    398,610& 316,149& 236,019& 148,834& 2,802,227& 19.5  \% \\ \hline

    2.87$< M_{\mu\mu} <$3.38 (GeV/c$^2$)&
    219,110& 173,701& 129,346& 81,808& 1,540,868& 10.7  \% \\ \hline
    
  \end{tabular}
  \caption{Selected dimuon events for the last four data periods and the total
    number of events from the intermediate mass range analysis of 2015 COMPASS
    data}
  \label{tab::JPsiStats2}
\end{table}


\subsection{Binning}

The analysis is determined as a function of the variables $x_N$, $x_{\pi}$,
$x_F$, $q_T$ and $M_{\mu\mu}$.  These are the same variables used to bin the
high mass Drell-Yan analysis.  The left-right asymmetry analysis is binned in
each of the kinematic variables by requiring an equal amount of data per
kinematic bin.  The bin limits are also required to have a width of at least
three times the resolution per variable.  For this analysis there are enough
events and the resolution per variable is good enough to have four kinematic
bins.  The bin limits are provided in Table~\ref{tab::JPsi_binning} and the
spectrometer resolutions are provided in Table~\ref{tab::JPsiRes}.

The spectrometer resolution is determined from the Monte-Carlo data.  The
resolution is determined from the difference between the Monte-Carlo generated
value and the reconstruction Monte-Carlo value.  An example of this distribution
is shown in Fig.~\ref{fig::JPsiXPiRes} for the $x_\pi$ variable.  The
distribution has a longer tail than a Gaussian distribution and for this reason
a two Gaussian fit function is used to determine the distribution's width.  The
resolution is then determined as the width of the Gaussian with the larger
amplitude, so-called leading order Gaussian.  The actual spectrometer resolution
is between the width of the leading order Gaussian and the RMS of the
distribution.  However the leading order Gaussian width is closer to the true
resolution and is therefore used as the estimate for the spectrometer
resolution.

\begin{figure}[h!t]
  \centering
  \includegraphics[width=0.6\textwidth, trim=3cm 9cm 3cm 9cm,clip]{JPsiXPiRes}
  \caption{The distribution of generated $x_N$ minus reconstructed $x_N$.  The
    leading Gaussian width (red) is used to determine the resolution.}
  \label{fig::JPsiXPiRes}
\end{figure}

\begin{table}[h!t]
  \centering
  \begin{tabular}{ |c|c|c| }
    \hline \textbf{Variable}& \textbf{RMS}& \textbf{Leading Gaussian $\sigma$}
    \\ \hline \hline
    
    $x_N$& 0.01& 0.006 \\ \hline

    $x_\pi$& 0.0157& 0.009 \\ \hline

    $x_F$& 0.016& 0.011 \\ \hline

    $q_T$& 0.143& 0.022 \\ \hline
  \end{tabular}
  \caption{COMPASS spectrometer resolutions in the intermediate mass range
    2.5-4.3{\gvcw}}
  \label{tab::JPsiRes}
\end{table}

\begin{table}[h!t]
  \begin{adjustwidth}{-1.4cm}{}
  \centering
  \begin{tabular}{ |c|c|c|c|c|c| }
    \hline \textbf{Variable}& \textbf{Lowest limit}& \textbf{Upper limit bin
      1}& \textbf{Upper limit bin 2}& \textbf{Upper limit bin 3}&
    \textbf{Upper limit bin 4}\\ \hline
    
    $x_N$& 0.0& 0.062& 0.083& 0.11& 1.0\\ \hline
    
    $x_{\pi}$& 0.0& 0.21& 0.28& 0.38& 1.0\\ \hline
    
    $x_F$& -1.0& 0.10& 0.20& 0.32& 1.0\\ \hline
    
    $q_T$ (GeV/c)& 0.4& 0.72& 1.04& 1.47& 5.0\\ \hline
    
    $M_{\mu\mu}$ (GeV/c$^2$)& 2.87& 3.03& 3.13& 3.22& 3.38 \\ \hline
    
  \end{tabular}
  \caption{The {\jp} analysis bin limits for the four analysis bins}
  \label{tab::JPsi_binning}
  \end{adjustwidth}
\end{table}

The distributions for the binning variables are shown in
Fig.~\ref{fig::JPsi_Kinematics}.  This analysis is performed in the valence
region for both the beam pion and target proton as the average $x_\pi$ is about
0.09 and the average $x_N$ is about 0.3.  This means the dominant contribution
to {\jp} production is from quark-quark interactions.  The average $q_T$ value
is below the minimum mass range of 2.87~{\gvcw} for this analysis and therefore
the interpretation of this analysis assumes the TMD regime is valid.

\begin{figure}[h!t]
  \centering \includegraphics[width=0.95\textwidth,trim=1cm 3.5cm 1cm
    3.5cm,clip]{JPsi_Kinematics}
  \caption{The binning variable distributions. The longitudinal momentum
    fractions $x_N$ (top left) and $x_\pi$ (top right) and the $x_F$ (bottom
    left) and the virtual photon transverse momentum $q_T$ (bottom right).
    These distributions are plotted from in the mass range 2.87-3.38~{\gvcw}.}
  \label{fig::JPsi_Kinematics}
\end{figure}


\subsection{Asymmetry Extraction}
The asymmetry extraction method used is the two-target geometric mean.  This
asymmetry method is described in detail in
Secs~\ref{sec::leftrightasym}-~\ref{sec::TwoTargGeoMean}.  The asymmetry is
again defined as

\begin{equation}
  \label{equ::AN4TargGeomeanJPsi}
  A_{lr,2Targ} =
  \frac{1}{|S_T|}
  \frac{ \sqrt[4]{ N_{1,l}^\uparrow N_{1, l}^\downarrow
      N_{2,l}^\uparrow N_{2, l}^\downarrow }
    - \sqrt[4]{ N_{1,r}^\uparrow N_{1,r}^\downarrow
      N_{2,r}^\uparrow N_{2,r}^\downarrow }
  }{
    \sqrt[4]{ N_{1,l}^\uparrow N_{1, l}^\downarrow
      N_{2,l}^\uparrow N_{2, l}^\downarrow }
    + \sqrt[4]{ N_{1,r}^\uparrow N_{1,r}^\downarrow
      N_{2,r}^\uparrow N_{2,r}^\downarrow } },
\end{equation}
\noindent
where $N$ represents the counts, $l(r)$ denotes left(right), 1(2) denotes the
target cell and $\uparrow(\downarrow)$ denotes the transverse polarization
direction.  The definitions of left and right are defined relative to the target
spin as
\begin{equation}
  \begin{aligned}
    \text{Left} &: \hat{q}_T \cdot (\hat{S}_T \times \hat{P}_{\pi}) > 0 \\
    \text{Right} &: \hat{q}_T \cdot (\hat{S}_T \times \hat{P}_{\pi}) < 0, 
  \end{aligned}
\end{equation}
\noindent
where $\hat{q}_T$, $\hat{S}_T$ and $\hat{P}_{\pi}$ are unit vectors in the
target reference frame for the virtual photon transverse momentum, the target
spin and the beam pion momentum respectively.  The advantage of this asymmetry
method is that the acceptance from the upstream and downstream target cells
cancel as was shown in Eq.~\ref{equ::AN2targAcceptCancel}.  The statistical
uncertainty for this asymmetry method can be written as

\begin{equation}
  \delta A_{lr,2Targ} = \frac{1}{|S_T|}
  \frac{LR}{\Big( L+R \Big)^2}
  \sqrt{
    \sum_{c,p}
    \Big(
    \frac{1}{N_{c,l}^{p}}
    + \frac{1}{N_{c,r}^p}
    \Big)
  } \quad,
\end{equation}
where $L =\sqrt[4]{N_{1,l}^\uparrow N_{1,l}^\downarrow N_{2,l}^\uparrow
  N_{2,l}^\downarrow}$ and $R =\sqrt[4]{N_{1,r}^\uparrow N_{1,r}^\downarrow
  N_{2,r}^\uparrow N_{2,r}^\downarrow}$.  In the approximate case of equal
statistical populations in each left-right direction and each target cell, the
statistical uncertainty for the two-target geometric mean reduces to
$\frac{1}{|S_T|}\frac{1}{\sqrt{N}}$, where $N$ is the sum of all counts.

\subsection{Systematic Studies}
Similar tests as were performed for the high mass left-right asymmetry to
determine the systematic error are also performed for the intermediate mass
left-right asymmetry.  For the full details on the previous tests see
Sec~\ref{sec::systematics}.  This section will give the results for systematic
errors in the intermediate mass range and describe in detail the tests specific
to this analysis.  The overall systematic errors are determined by adding all
non-zero systematic uncertainties in quadrature.  The impact from each source of
systematic error is summarized in Table.~\ref{tab::sysErrorJPsi}.

\subsubsection{Period Compatibility (Time Dependence)}
It is expected that the asymmetry calculation will vary in time due to
statistical fluctuations.  Fig.~\ref{fig::JPsi_AlrPeriod} shows the left-right
asymmetry calculated for each period in time and Fig.~\ref{fig::JPsi_AlrXn}
shows the left-right asymmetry time fluctuations for each bin in $x_N$.  To
quantify if the time fluctuations are greater than what is expected from random
statistical fluctuations, the pull distribution is checked for a larger width
than one or a non-zero mean.  More details on the pull distribution are given in
Sec~\ref{sec::sysPulls}.

\begin{figure}[h!t]
  \centering \includegraphics[width=0.65\textwidth,trim=0.5cm 6cm 0.5cm
    6cm,clip]{JPsi_AlrPeriod}
  \caption{The left-right asymmetry in the {\jp} mass region as a function of
    time.  The red line is a constant fit and therefore shows the weighted
    averaged of the periods.}
  \label{fig::JPsi_AlrPeriod}
\end{figure}

\begin{figure}[h!t]
  \centering
  \includegraphics[width=0.65\textwidth,trim=1cm 8cm 1cm 8cm,clip]{JPsi_AlrXn}
  \caption{The left-right asymmetry time fluctuations in each bin of $x_N$.}
  \label{fig::JPsi_AlrXn}
\end{figure}

The pull value is defined as
\begin{equation}
  \label{eq::pullJPsi}
  \Delta\mathrm{A}_i =
  \frac{
    \mathrm{A}_i - \langle \mathrm{A} \rangle
  }{
    \sqrt{
      \sigma^2_{\mathrm{A}_i} - \sigma^2_{\langle \mathrm{A} \rangle}
    }
  },
\end{equation}
\noindent
where $\langle \mathrm{A} \rangle$ is the average asymmetry amplitude for the
data set.  The pull distribution is formed for each kinematic variable in
Fig.~\ref{fig::JPsi_Pulls}.  For this analysis there are therefore 4 (number of
bins) x 9 (number of periods) = 36 entries per pull distribution.  The
systematic error from period incompatibility is determined as

\begin{equation}
  \label{equ::sysErrorPullJPsi}
  \frac{\sigma_{systematic}}{\sigma_{statistical}} =
  \sqrt{|\sigma^2_{pull} - 1|} + \frac{\mu_{pull}}{2},
\end{equation}
\noindent
where in this analysis the $\sigma_{pull}^2$ and $\mu_{pull}$ are determined as
a weighted average of the mean and variance respectively from the pull
distribution for each kinematic variable including the parameters from the
integrated pull distribution.  The fit values from each pull distribution give
somewhat different estimates due to the errors associated with the fit.  This is
the reason the weighted average is performed to give the best estimate for the
pull mean and standard deviation and therefore the most accurate systematic
error calculation.  The systematic error due to time incompatibility is
determined to be 16\% of the statistical error.

\begin{figure}[h!t]
  \centering \includegraphics[width=\textwidth,trim=2.2cm 1.7cm 2.2cm
    1.7cm,clip]{JPsi_Pulls}
  \caption{The uncorrelated pull distributions for each of the kinematical
    variables and a pull distribution from the integrated asymmetry from each
    time period.}
  \label{fig::JPsi_Pulls}
\end{figure}


\subsubsection{Left/Right Event Migration}
The left-right event migration systematic error is calculated the same as in
Sec~\ref{sec::syslrEventMigration}.  The Monte-Carlo data used to determine the
left-right event migration is described in Table~\ref{tab::MCproduction}.  The
effect of misidentified events as left when the event should be counted as a
right event and vise-versa, dilutes the left-right asymmetry.  It is as if
there is an additional unpolarized contribution that dilutes the event sample.

The systematic error for left-right migration is derived in
Appendix~\ref{app::sysLRmiss} as

\begin{equation}
  \delta A_{lr,systematic} = \gamma *A_{lr} + \gamma *\delta A_{lr},
\end{equation}
\noindent
where $\gamma$ is the fraction of misidentified left and right events.  

As is clearly visible in Fig.~\ref{fig::JPsilrMigration}, there is a band of
higher misidentification rate at the border between left and right.  For the
{\jp} mass region only about 3\% of events were misidentified resulting in a
systematic error of 4.4\% of the statistical error.

\begin{figure}[h!t]
  \centering \includegraphics[width=0.4\textwidth,trim=2cm 7cm 2cm
    7cm,clip]{JPsilrMigration}
  \caption{The left-right migration as a function of the generated $\phi_S$
    angle in the mass range 2.87-3.38.}
  \label{fig::JPsilrMigration}
\end{figure}

\subsubsection{J/$\Psi$ Purity}

The systematic error due to a {\jp} purity less than unity was discussed in
Sec~\ref{sec::jpMassRange}.  The invariant mass range was chosen specifically
such that the {\jp} purity is 90\% or higher to reduce the systematic error
associated with impurities.  The systematic error is derived in
Appendix~\ref{app::sysEventContam} as

\begin{equation}
  \frac{\sigma_{systematic}}{\sigma_{statistical}} = \frac{(1-p)}{p}.
\end{equation}
\noindent
Table~\ref{tab::JPsiPurity} summarized the impurity as a function of the
analysis mass range and the impurity in this analysis mass range is 91.3\%.
This corresponds to a systematic error of 9.5\% of the statistical error.

\subsubsection{False Asymmetries}
\subsubsection{Acceptance From False Asymmetries}
The acceptance fluctuations are determined from the false asymmetry defined as
\begin{equation}
  A_{lr,False} = 
    \frac{1}{|S_T|}
    \frac{
      \sqrt[4]{
        N_{1,r}^\uparrow N_{1, l}^\downarrow
        N_{2,l}^\uparrow N_{2, r}^\downarrow
      } 
      -\sqrt[4]{
        N_{1,l}^\uparrow N_{1, r}^\downarrow
        N_{2,r}^\uparrow N_{2, l}^\downarrow
      }
    }{
      \sqrt[4]{
        N_{1,r}^\uparrow N_{1, l}^\downarrow
        N_{2,l}^\uparrow N_{2, r}^\downarrow
      } +
      \sqrt[4]{
        N_{1,l}^\uparrow N_{1, r}^\downarrow
        N_{2,r}^\uparrow N_{2, l}^\downarrow
      }
    } 
    = \frac{1}{|S_T|}
    \frac{
      \alpha_{2Targ} - 1     
    }{
      \alpha_{2Targ} + 1
    }.
    \label{equ::falseAccJPsi}
\end{equation}
\noindent
More details for acceptance fluctuations are discussed in
Sec.~\ref{sec::GeoMean} and Sec.~\ref{sec::TwoTargGeoMean}.  The kinematic
dependencies of the acceptance ratio, $\alpha_{2Targ}$, are shown in
Fig.~\ref{fig::alphaJPsi}.  As Fig.~\ref{fig::alphaJPsi} shows the acceptance is
only slightly greater than unity even though it can be above 1 by more than a
sigma.  The systematic error associated with acceptance fluctuations is defined
as

\begin{equation}
  \delta A_{lr,systematic} =
  \frac{1}{|S_T|}
  \Big(\frac{|\alpha_{2Targ}-1|}{2}
  + \delta_{\frac{|\alpha_{2Targ}-1|}{2}} \Big),
\end{equation}

\noindent
where this expression is derived in Appendix~\ref{app::sysAcc}.  The normalized
kinematic dependence of the systematic error to the statistical error are shown
in Fig.~\ref{fig::accSysStatJPsi}.  The average systematic error due to
acceptance is 23\% of the statistical error.

\begin{figure}[h!t]
  \begin{center}
    \includegraphics[width=\textwidth, trim=0cm 5cm 0cm 5cm,
      clip]{alphaJPsi}
    \caption{Acceptance fluctuations in each bin of the kinematic variables.}
    \label{fig::alphaJPsi}
  \end{center}
\end{figure}

\begin{figure}[h!t]
  \begin{center}
    \includegraphics[width=\textwidth, trim=0.5cm 5cm 0.5cm 5cm,
      clip]{accSysStatJPsi}
    \caption{Systematic error divided by statistical error due to acceptance}
    \label{fig::accSysStatJPsi}
  \end{center}
\end{figure}

\subsubsection{Further False Asymmetry Effects}
Additional false asymmetries are analyzed to account for systematic errors which
were not addressed directly.  The false asymmetries are constructed in such a
way that the cross-section and luminosity cancel out in the numerator and the
denominator.  Therefore these false asymmetries can only change in time due to
acceptance effects or for some unknown reason.  The false asymmetries
constructed to study the intermediate mass analysis are described in the
following enumerated list.

\begin{enumerate}
  \label{tab::additionalFAJPsi}

\item The false asymmetry used to determine the acceptance fluctuations,
  Eq.~\ref{equ::falseAccJPsi}, is checked for compatibility and the
  uncorrelated pulls are shown in Fig.~\ref{fig::fa2TargPullsJPsi} along with
  the corresponding fit parameters and errors.

  \begin{figure}[h!t]
    \centering \includegraphics[width=\textwidth, trim=0cm 3cm 0cm 3cm,
      clip]{fa2TargPullsJPsi}
    \caption{Pull distribution for a nearly acceptance free two-target false
      geometric mean asymmetry}
    \label{fig::fa2TargPullsJPsi}
  \end{figure}

  \item A false asymmetries using only the information from the upstream or the
  downstream target cell defined as

  \begin{equation}
    \label{equ::falseANgeomeanJPsi}
    A_{lr, FA} =
    \frac{1}{|S_T|}
    \frac{\sqrt{N_l^\uparrow N_r^\downarrow}
      - \sqrt{N_r^\uparrow N_l^\downarrow}
    }{
      \sqrt{N_l^\uparrow N_r^\downarrow}
      + \sqrt{N_r^\uparrow N_l^\downarrow}
    }.
  \end{equation}
  The pulls for the upstream target cell are shown in
  Fig.~\ref{fig::faPullsUpSJPsi} and the pulls for the downstream target cell
  are shown in Fig.~\ref{fig::faPullsDownSJPsi}.

  \begin{figure}[h!t]
  \centering \includegraphics[width=\textwidth, trim=0cm 3cm 0cm 3cm,
    clip]{faPullsUpSJPsi}
  \caption{Pull distributions from the false asymmetry in the upstream target
    cell}
  \label{fig::faPullsUpSJPsi}
  \end{figure}

\begin{figure}[h!t]
  \centering \includegraphics[width=\textwidth, trim=0cm 3cm 0cm 3cm,
    clip]{faPullsDownSJPsi}
  \caption{Pull distributions from the false asymmetry in the downstream target
    cell}
  \label{fig::faPullsDownSJPsi}
\end{figure}

\end{enumerate}

The systematic error from each false asymmetry is determined using the
Eq.~\ref{equ::sysErrorPullJPsi}.  The uncorrelated pulls have only 4 (number of
bins) $\times$ 9 (number of periods) = 36 entries which results in large errors
on the Gaussian fit results.  In an attempt to correct for this and to take into
account the fit errors, a weighted average of the mean and standard deviation is
made using the fit parameters and fit errors as weights from the uncrrelated
pull distributions.  This is the same technique as was used to determine the
systematic error from fluctuations in time.  The resulting weighted mean and
weighted standard deviation are then used to calculate the systematic error.  A
summary of the systematic error from each false asymmetry is shown in
Table~\ref{tab::faSysJPsi}.  The systematic error due to additional factors is
chosen as the largest systematic error from Table~\ref{tab::faSysJPsi}.

\begin{table}[h!t]
  \centering
  \begin{tabular}{|c|c|}
    \hline Systematic error& \multirow{2}{9em}{$\langle
      \sigma_{\mathrm{systematic}}/\sigma_{\mathrm{statistical}}
      \rangle$}\\ & \\ \hline

    Two target acceptance estimation& 0.19\\ \hline
    
    Target Cell 1& 1.20\\ \hline

    Target Cell 2& 1.11\\ \hline
    
  \end{tabular}
  \caption{Summary of systematic error impacts from false asymmetries changes in
    time.  The maximum systematic error is chosen as the systematic error.}
  \label{tab::faSysJPsi}
\end{table}

\subsubsection{Total Systematics}
The total systematic error is determined by adding all systematic errors in
quadrature as

\begin{equation}
  \Big \langle \frac{
    \sigma_{\mathrm{systematic}}}{\sigma_{\mathrm{statistical}}} \Big \rangle =
  \sqrt{ \sum_i^{\mathrm{all \; systematic}} \Big \langle
    \frac{\sigma^2_{\mathrm{systematic, i}}}{\sigma^2_{\mathrm{statistical}}}
    \Big \rangle } \;,
\end{equation}
where all the systematic effects considered are summarized in
Table.~\ref{tab::sysErrorJPsi}.  For reference the integrated left-right
asymmetry is $\langle A_{lr} \rangle = 0.0062$ and the integrated statistical
error is $\langle \sigma_{\mathrm{statistical}} \rangle$ = 0.0065.

\begin{table}[h!t]
  \centering
  \begin{tabular}{|c|c|c|}
    \hline
    \multirow{2}{*}{Systematic error}&
    \multirow{2}{*}{
      $\langle \sigma_{\mathrm{systematic}}/\sigma_{\mathrm{statistical}}
      \rangle$} &
    \multirow{2}{*}{$\langle \sigma_{\mathrm{systematic}} \rangle$}\\
    & & \\ \hline \hline

    Period compatibility& 0.16& 0.001\\ \hline

    Left-Right migration& 0.044& 0.0003\\ \hline

    J/$\Psi$ purity& 0.095& 0.0006\\ \hline

    Target Polarization& 0.05& 0.0003\\ \hline

    Dilution Factor& 0.05& 0.0003\\ \hline

    Acceptance fluctuation& 0.23 & 0.001\\ \hline

    False asymmetry& 1.2 & 0.008\\ \hline \hline
    \textbf{Total}& \textbf{1.24} & \textbf{0.008}\\\hline
    
  \end{tabular}
  \caption{Summary of systematic error impacts to the integrated asymmetry}
  \label{tab::sysErrorJPsi}
\end{table}

The integrated left-right asymmetry result and systematic error band is shown in
Fig.~\ref{fig::Alr_JPsi} and the kinematic dependencies are shown in
Fig.~\ref{fig::AlrBinned_JPsi}.  Similarly to the left-right asymmetry for the
high mass Drell-Yan analysis, the integrated left-right asymmetry is 1 sigma
above zero.  The asymmetry shows a weak inverse dependence on $x_F$ indicating
the asymmetry could be related to quark distributions in the proton target.
This can also be seen in the $x_N$ dependence which is most significant in the
highest $x_N$ bin.  Although the left-right asymmetry is model independent, it
is was discussed that the Sivers function could be the cause of for a non-zero
left-right asymmetry.  A positive left-right asymmetry would be consistent with
the sign change hypothesis.

\begin{figure}[h!t]
  \centering
  \includegraphics[width=0.4\textwidth,trim=6cm 6cm 6cm 6cm,clip]{Alr_JPsi}
  \caption{The integrated left-right asymmetry from the mass range
    2.87-3.38~{\gvcw}.  The systematic error bands are shown in red at the
    bottom of the plot.}
  \label{fig::Alr_JPsi}
\end{figure}

\begin{figure}[h!t]
  \centering \includegraphics[width=\textwidth,trim=1cm 4cm 1cm
    4cm,clip]{AlrBinned_JPsi}
  \caption{The kinematic dependencies of the left-right asymmetry from the mass
    range 2.87-3.38~{\gvcw}.}
  \label{fig::AlrBinned_JPsi}
\end{figure}

The Anselmino group derived an expression for the $A_N$ asymmetry from {\jp}
production as~\cite{Anselmino:2016fhz}

\begin{equation}
  A^{J\Psi}_{N} \propto f_{\bar{q}/H_a}(x_a, k_{aT})
  \frac{k_{bT}}{M_b} \otimes f_{1T}^{\perp q}(x_b, k_{bT}),
\end{equation}
\noindent
and a made predictions for the $A_N$ asymmetry as a function of $x_N$ and $q_T$
shown in Fig.~\ref{fig::AN_JPsiPredict}.  As derived in
Sec~\ref{sec::lr_theory}, $A_{N} = \frac{\pi A_{lr}}{2}$.  More details are
given in Sec~\ref{sec::theory_jpsi}, but their calculation assumed the dominant
contribution to {\jp} production is from quark-quark annihilation.
Fig.~\ref{fig::AN_JPsi} shows the results determined at COMPASS for the $A_N$
asymmetry determined by modifying the left-right results in
Fig.~\ref{fig::AlrBinned_JPsi}.  The maximum $A_N$ asymmetry, determined at
COMPASS, is less than 0.05 which when comparing to the prediction in
Fig.~\ref{fig::AN_JPsiPredict} is 2 sigma less than the prediction.

\begin{figure}[h!t]
  \centering \includegraphics[width=0.95\textwidth,trim=2cm 7cm 2cm
    7cm,clip]{AN_JPsiPredict}
  \caption{Predictions for the analyzing power, $A_N$, from {\jp} production at
    COMPASS as a function of $x_N$ and $q_T$.  This image was taken
    from~\cite{Anselmino:2016fhz}.}
  \label{fig::AN_JPsiPredict}
\end{figure}

\begin{figure}[h!t]
  \centering \includegraphics[width=0.55\textwidth,trim=3cm 5.5cm 3cm
    5.5cm,clip]{AN_JPsi}
  \caption{The analyzing power at COMPASS determined by modifying the left-right
    asymmetry as a function of $x_N$ and $q_T$ to be compared with the
    predictions in Fig.~\ref{fig::AN_JPsiPredict}. }
  \label{fig::AN_JPsi}
\end{figure}

The incompatibility of the theory prediction, Fig.~\ref{fig::AN_JPsiPredict},
and the results determined at COMPASS, Fig.~\ref{fig::AN_JPsi}, can be due to
either a Siver function much lower than expected or gluon-gluon fusion
contamination.  At the present moment, the Siver's function from gluon-gluon
fusion is not well known and therefore could be small or negative.
Fig.~\ref{fig::ggQQbarXsection} shows results for {\jp} production from
quark-quark annihilation and gluon-gluon fusion using the color evaporation
model~\cite{VOGT1999197}.  To determine the expected {\jp} contribution from
gluon-gluon fusion and quark-quark annihilation at COMPASS the results in
Fig.~\ref{fig::ggQQbarXsection} are weighted with the COMPASS $x_F$ distribution
in the intermediate mass range.  The determined ratio of {\jp} production from
quark-quark annihilation to gluon-gluon fusion at COMPASS is 0.8.  It is
therefore not ruled out that the results in Fig.~\ref{fig::AN_JPsi} are reduced
compared to the predictions in Fig.~\ref{fig::AN_JPsiPredict} due to gluon-gluon
fusion.

\begin{figure}[h!t]
  \centering \includegraphics[width=0.65\textwidth,trim=2.5cm 9.5cm 2.5cm
    9.5cm,clip]{ggQQbarXsection}
  \caption{{\jp} production cross-section from gluon-gluon fusion (blue) and
    quark-quark annihilation (red) and the sum (black) as a function of $x_F$.
    This plot is made assuming the color evaporation model~\cite{VOGT1999197}. }
  \label{fig::ggQQbarXsection}
\end{figure}

\chapter{Conclusion}
\label{ch::conclusion}

Mapping out the three dimensional momentum structure of the nucleon is a recent
and exciting field.  Both theoretical work and experiments have been
contributing to the transverse momentum dependent parton distribution functions.
This dissertation presents results from dimuon events which originate from a
190~{\gvc} negatively charged pion beam on a transversely polarized proton
target.  The measurements from this thesis expand the knowledge of parton
distributions.  Specifically the measurements in this thesis add knowledge to
the transverse momentum dependent parton distribution functions.  The
measurements in the high invariant mass region, above 4.3~{\gvcw}, constitues
the world's first ever transversely polarized Drell-Yan data.  The COMPASS
spectrometer is unique in that it can perform spin-dependent measurements for
Drell-Yan and semi-inclusive deep inelastic scattering in the same kinematic
regions.

The analysis techniques in this thesis are constructed to measure azimuthal
asymmetry amplitudes where the spectrometer acceptance cancels and therefore
does not effect the measurement.  Several analysis techniques are used and all
find the Sivers asymmetry amplitude to be approximately 1 sigma above zero.
This result is consistent with the sign change hypothesis between semi-inclusive
deep inelastic scattering and Drell-Yan production.  At the same time, the
statistical error bars are too large to distinguish between the major models of
the Sivers function.  For this reason the COMPASS collaboration performed
another Drell-Yan data taking campaign in 2018 with similar data taking
conditions.  The results from the 2018 measurement are expected soon.


% ********************************** Appendices ********************************
\begin{appendices}
\chapter{Systematic Error Derivations}
\ifpdf
\graphicspath{Chapters/Appendix/Figs/PDF/}
\fi

\section{Systematic Error From Acceptance} \label{app::sysAcc}

For an asymmetry defined as
\begin{equation}
  A_{\alpha} =
  \frac{1}{|S_T|}
  \frac{\alpha\sigma_l - \sigma_r}{\alpha\sigma_l + \sigma_r} 
\end{equation}

\noindent
where $\alpha$ is an acceptance ratio.  $\alpha$ is assumed to be close to unity
therefore let

\begin{equation}
  \alpha = 1 \pm 2\epsilon,
\end{equation}

\noindent
where $\epsilon$ is a small positive number.  The asymmetry can
therefore be written

\begin{equation}
  A_\alpha = \frac{1}{|S_T|}
  \frac{(1\pm2\epsilon)\sigma_l -
    \sigma_r}{(1\pm2\epsilon)\sigma_l + \sigma_r} = \frac{1}{|S_T|} \frac{\sigma_l
    - \sigma_r \pm 2\epsilon  \sigma_l}{ (\sigma_l +
    \sigma_r)(1\pm\frac{2\epsilon \sigma_l}{\sigma_l + \sigma_r}) }.
\end{equation}

\noindent
From there Taylor expand the denominator to get
\begin{align}
  A_{\alpha} &\approx
  \frac{1}{|S_T|}
  \frac{\sigma_l - \sigma_r \pm
    2\epsilon  \sigma_l}{ (\sigma_l + \sigma_r)} (1\mp\frac{2\epsilon
    \sigma_l}{\sigma_l + \sigma_r})
  \\ \nonumber
  &= 
  A_{lr}
  \pm \frac{1}{|S_T|} \frac{2\epsilon \sigma_l}{\sigma_l + \sigma_r}
  \mp A_{lr} \frac{2\epsilon \sigma_l}{\sigma_l + \sigma_r}
  \mp \frac{1}{|S_T|}
  \Big( \frac{2\epsilon \sigma_l}{\sigma_l + \sigma_r} \Big )^2.
\end{align}

\noindent
Assuming $A_{lr}$ is small and $\sigma_l \approx \sigma_r$

\begin{equation}
A_{\alpha}\approx 
A_{lr} \pm \frac{\epsilon}{|S_T|}.
\end{equation}

\noindent
The true asymmetry can now be written

\begin{equation}
  A_{lr,systematic} \approx
  A_{\alpha} \mp \frac{\epsilon}{|S_T|}.
\end{equation}

\noindent
Including the $\frac{\epsilon}{|S_T|}$ term as an additive error and using
standard error propagation the systematic error can be approximated as

\begin{equation}
  \delta A_{lr,systematic} = \frac{ \mid\alpha - 1
    \mid}{2}\frac{1}{|S_T|} + \frac{\delta_{\frac{\mid \alpha -1
        \mid}{2}}}{|S_T|}.
\end{equation}


\section{Systematic Error From Left-Right Event Migration}
\label{app::sysLRmiss}

Assuming the fraction of events miss-identified is $\gamma$ and that the amount
of miss-identified events reconstructed left equals the amount of outgoing
events reconstructed right

\begin{equation}
  A_{lr,measure} =
  \frac{1}{|S_T|} \frac{(l+ \frac{\gamma}{2}
    N_{total}) - (r + \frac{\gamma}{2}
    N_{total})} {(l+ \frac{\gamma}{2}
    N_{total})+(r+ \frac{\gamma}{2}
    N_{total})}
  = \frac{1}{|S_T|} \frac{l - r}
         {(l+r)(1+ \gamma
           \frac{N_{total}}{l+r})},
\end{equation}

\noindent
where $N_{total}$ is the total events measure, $l$ is the true events
measured to the left that should be measured left and $r$ is the number of
events measure to the right that should be measured to the right.\par

Assuming $\gamma$ is a small percentage, the denominator can be Taylor expanded
to give

\begin{equation}
  A_{lr,measure} \approx
  A_{lr}
  \Big (1-\gamma\frac{N_{total}}{l+r}\Big).
\end{equation}

\noindent
Including $\gamma A_{lr,measure}$ as an additive error and using
standard error propagation the systematic error can be approximated as

\begin{equation}
  \delta A_{lr,systematic} =
  \gamma A_{lr,measure} +
  \gamma \delta A_{lr,measure}.
\end{equation}


\section{Systematic Error From Event Contamination} \label{app::sysEventContam}
Often times the measured counts come from multiple sources where only a
measurement from a single source is of interest.  As long as the source of
interest is dominates the total counts, a left-right asymmetry can still be
determined for the source of interest.  In this derivation there will be an
assumed a signal source with counts $N_S$ and a background source with counts
$N_{bg}$, where the background takes into account all processes that are not of
interest.  Defining the purity of the signal as

\begin{equation}
  p = \frac{N_S}{N_S + N_{bg}},
\end{equation}
\noindent
then the left-right asymmetry can be determined as

\begin{align}
  A_{lr} &= \frac{1}{|S_T|}
  \frac{N_{l,S} + N_{l,bg} - \Big(N_{r,S} + N_{r,bg}\Big)
  }{N_{l,S} + N_{l,bg} + \Big(N_{r,S} + N_{r,bg} \Big)}
  \\ \nonumber
  &= \frac{1}{|S_T|}\Big\{
  \frac{N_{l,S} - N_{r,S}}{N_{l,S} + N_{r,S} + N_{l,bg} + N_{r,bg}} 
  + \frac{N_{l,bg} - N_{r,bg}}{N_{l,bg} + N_{r,bg} + N_{l,S} + N_{r,S} }
  \Big\}
  \\ \nonumber
  &= \frac{1}{|S_T|}\Big\{
  \frac{N_{l,S} - N_{r,S}}{
    \Big(N_{l,S} + N_{r,S}\Big)\Big(1+ \frac{N_{l,bg} + N_{r,bg}}{N_{l,S} + N_{r,S}}\Big)}
  + \frac{N_{l,bg} - N_{r,bg}}{\Big(N_{l,bg} + N_{r,bg}\Big)\Big(1+ \frac{N_{l,S} + N_{r,S}}{N_{l,bg} + N_{r,bg}}\Big) }
  \Big\}
  \\ \nonumber
  &= \frac{1}{|S_T|}\Big\{
  \frac{N_{l,S} - N_{r,S}}{
    \Big(N_{l,S} + N_{r,S}\Big)\Big(\frac{1}{p}\Big)}
  + \frac{N_{l,bg} - N_{r,bg}}{\Big(N_{l,bg} + N_{r,bg}\Big)\Big(\frac{p}{1-p}\Big) }
  \Big\}
  \\ \nonumber
  &= pA_{lr,S} + \frac{1-p}{p}A_{lr,bg}.
\end{align}
\noindent
This means that by measuring the purity, $p$, and the left-right asymmetry,
$A_{lr}$, the left-right asymmetry from the signal, $A_{lr,S}$, can be
determined as

\begin{equation}
  A_{lr,S} = \frac{1}{p} A_{lr} - \frac{1-p}{p^2} A_{lr,bg}.
\end{equation}
\noindent
Assuming a purity above 90\% and a background left-right asymmetry, $A_{lr,bg}$,
of 5\%
\begin{equation}
  A_{lr,S} = 1.11 A_{lr} - 0.123(0.05) = 1.11 A_{lr} - 0.006\approx 1.11 A_{lr}.
\end{equation}

The systematic error from a purity less than 1 can be determined as

\begin{align}
  A_{lr,S} &= \frac{1}{p}A_{lr} = A_{lr} + \frac{1-p}{p}A_{lr}
  \\ \nonumber
  \Rightarrow \sigma^2_{A_{lr,S}} &= \sigma^2_{A_{lr}} + \frac{(1-p)^2}{p^2} \sigma^2_{A_{lr}}
  \\ \nonumber
  \Rightarrow \sigma^2_{systematic}/\sigma^2_{statistic} &= \frac{(1-p)^2}{p^2}.
\end{align}
 
\end{appendices}

% ********************************** Bibliography ******************************
\renewcommand{\chaptername}{}
\renewcommand{\thechapter}{} 
%\bibliographystyle{plain}
\bibliographystyle{unsrtnat}
%\bibliographystyle{ieeetr}
\bibliography{References/bibliography}{}


\end{document}
