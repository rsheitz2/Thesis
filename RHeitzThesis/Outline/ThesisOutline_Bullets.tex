\documentclass{article}

\title{Transverse Nucleon Spin Structure determined through the Drell-Yan
  Process from Pions on a Transversely Polarized Proton Target} \date{}

\begin{document}
\maketitle

\section{Introduction}


\section{Theoretical Overview}
\begin{itemize}
\item Longitudinal Nucleon Structure
\item Transverse Nucleon Structure
  \begin{itemize}
  \item Transverse Momentum Dependent Parton Distribution Functions
  \item Sivers and Boer-Mulders Functions
  \item Semi-universality
  \end{itemize}
  
\item{Drell-Yan}
\item{SIDIS}
\end{itemize}


\section{COMPASS Experiment}
\begin{itemize}
\item Beam
\item Transversely Polarized Target
\item Trigger
\item Spectrometer
\item Drift Chamber 5
  \begin{itemize}
  \item Construction
  \item Performance (Resolutions, Efficiency)
  \end{itemize}

\end{itemize}


\section{COMPASS data}
\begin{itemize}
\item Data acquisition system
\item Data reconstruction
\item Alignment
  \begin{itemize}
  \item Procedure
  \item Quality Criteria
  \item Results
  \end{itemize}
  
\end{itemize}


\section{Transverse Spin Asymmetries from Drell-Yan}
\begin{itemize}
\item Event Selection
\item Data Quality
\item Asymmetry Methods
  \begin{itemize}
  \item Double Ratio Method
  \item Left Right Asymmetry
  \item qT weighting Amplitudes (x-check)
  \end{itemize}
  
\end{itemize}


\section{Conclusion}

\end{document}
