\documentclass{article}
%\usepackage{mss}
\title{Transverse Spin effects in the Drell-Yan Process from Pions on a
  Transversely Polarized Proton Target}
%\author{J.~TELLINGHUISEN}
%\author{\underline{X.~ZHENG}, F.~FEI, and M.~C.~HEAVEN}
\date{}
\begin{document}
\maketitle

\section{Introduction and Theory}	% Produces section heading.  Lower-level sections
			% are begun with similar \subsection and
			% \subsubsection commands; numbering is automatic!

\subsection{Transverse Momentum Dependent Parton Distribution Functions}
TMDs take into account the transverse momentum of partons to describe
how constituent quark and gluon momentum makes up the proton's
momentum.  At leading twist there are eight TMDs needed to describe
the proton.  One specific TMD is called the Sivers function.  The
Sivers function gives a correlation between the parton's transverse
momentum and the transverse spin of the hadron it makes up.  It is
believe that this function can be used to determine the angular
momentum of a constituent parton.

\subsection{Drell-Yan}
The amplitudes of the spin dependent Drell-Yan cross section give
convolutions of TMDs of the scrattering hadrons.  With a transversely
polarized target the amplituded related to the Sivers function can be
measured.

\section{COMPASS Experiment}
COMPASS took nine two week periods of Drell-Yan data in the summer of
2015.  Specifically COMPASS took Drell-Yan data from a 190 GeV pion
beam on a transversely polarized Ammonia target.  The COMPASS data can
be used to determine the amplitude of the Drell-Yan cross-section
related to the proton Sivers function.

\subsection{Drift Chamber 5}
Simulations showed that 95\% of Drell-Yan data will include a track in the large angle spectrometer.  With two aging detectors in this section of the spectrometer, it was shown that the reconstruction efficiency for Drell-Yan events could reduce to less than 30\%.

Drift chamber 5 was build at UIUC, Old Dominion University, and assembled at CERN.

\subsection{Alignment}
The alignment of the spectrometer is an important aspect for reconstruction.  There are over 200 detector planes which need to be aligned in order to have track resolutions approximately 100 $\mu$m.  This is accomplished by minimizing the $\chi^2$ function of all the track positions minus hit positions in each detector.  

\section{Data Analysis}
\subsection{Physics cuts}
A clean DY sample is ensured through various cuts.

\subsection{Amplitude Extraction}
Through comparing the number of hits from the two oppositely polarized targets, all the amplitudes of the polarized Drell-Yan cross section can be determined.  
\end{document}
