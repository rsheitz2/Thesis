\chapter{Conclusion}
\label{ch::conclusion}

Mapping out the three dimensional momentum structure of the nucleon is a recent
and exciting field.  Both theoretical work and experiments have been
contributing to the transverse momentum dependent parton distribution functions.
This dissertation presents results from dimuon events which originate from a
190~{\gvc} negatively charged pion beam on a transversely polarized proton
target.  The measurements from this thesis expand the knowledge of parton
distributions.  Specifically the measurements in this thesis add knowledge to
the transverse momentum dependent parton distribution functions.  The
measurements in the high invariant mass region, above 4.3~{\gvcw}, constitues
the world's first ever transversely polarized Drell-Yan data.  The COMPASS
spectrometer is unique in that it can perform spin-dependent measurements for
Drell-Yan and semi-inclusive deep inelastic scattering in the same kinematic
regions.

The analysis techniques in this thesis are constructed to measure azimuthal
asymmetry amplitudes where the spectrometer acceptance cancels and therefore
does not effect the measurement.  Several analysis techniques are used and all
find the Sivers asymmetry amplitude to be approximately 1 sigma above zero.
This result is consistent with the sign change hypothesis between semi-inclusive
deep inelastic scattering and Drell-Yan production.  At the same time, the
statistical error bars are too large to distinguish between the major models of
the Sivers function.  For this reason the COMPASS collaboration performed
another Drell-Yan data taking campaign in 2018 with similar data taking
conditions.  The results from the 2018 measurement are expected soon.
