%Chapter 1 
\chapter{Measurement of the Left Right Asymmetry in the Drell-Yan Process} 
\label{Chap:setup}
\ifpdf
\graphicspath{{Chapters/OneCh/Figs/Raster/}{Chapters/OneCh/Figs/PDF/}{Chapters/OneCh/Figs/}}
\else \graphicspath{{Chapters/OneCh/Figs/Vector/}{Chapters/OneCh/Figs/}} \fi

Introduction about L/R asym.  In this Chapter....  Define AN
\section{Event Selection}
The cuts in the event selection were chosen to ensure the final state consisted
of dimuons resulting from a pion collision in the transversely polarized target.
The event selection was initial filtered from miniDSTs to $\mu$DSTs using the
criteria of at least two muons in the final state.  The following event
selection is performed on these $\mu$DSTs where the events used come from the
slot1 production.  A summary of the number of events remaining after each cut is
shown in figure~\ref{fig::EventTable}.
\begin{itemize}
\item Two oppositely charged particles from a common best primary vertex.  The
  criteria for a primary vertex is any vertex with an associated beam particle.
  In case of multiple common vertices the best primary vertex was determined by
  CORAL tagging the vertex as best primary (PHAST method
  PaVertex::IsBestPrimary()).  If CORAL did not tag any of the common vertices
  as the best primary the vertex with the smallest spatial chi$^2$ value was
  used as the best primary vertex.
\item A dimuon trigger fired.  A dimuon trigger firing means there are at least
  two particles in coincidence in this event. The dimuon triggers used were a
  coincidence between two particles in the large angle spectrometer, LAS-LAS
  trigger, or a particle in the large angle spectrometer and a particle in the
  Outer hodoscope in the small angle spectrometer, LAS-Outer trigger.  The
  LAS-Middle trigger was used a veto on beam decay muons.  This is because the
  LAS-Middle trigger was found to have many events resulting from a beam pion
  decaying to a muon.
\item Both particles are muons.  A muon was defined as having crossed 30
  radiation lengths of material between the particles first and last measured
  points.  This criteria has been previously determined to be effective at
  distinguishing between muons and hadrons.  In the final production no
  detectors were used from upstream of the hadron absorber so the absorber is
  not included in the determination of material crossed.
\item The first measured point for both particles is before 300 cm and the last
  measured point is after 1500 cm.  This cut ensures both particles have
  positions upstream of the first spectrometer magnet and downstream of the
  first muon filter.
\item The timing of both muons is defined.  This checks that the time relative
  to the trigger time is determined for both muons so further timing cuts can be
  performed.
\item Both muons are in time within 5 nanoseconds.  This cut helps rejected
  uncorrelated muons.
\item The muon tracks reduced chi$^2$ are individually less than 10.  This cut
  ensure track quality.
\item A validation that each muon crossed the trigger it was associated as
  having triggered.  This trigger validation cut was performed by extrapolating
  (PHAST Method PaTrack::Extrapolate()) each muon track back to the hodoscopes
  it fired and determining if the muon crossed the geometric acceptance of both
  hodoscopes.
\item The event does not occur in the bad spill or run list.
\item The Drell-Yan kinematics are physical.  That is the beam and target
  x-Bjorken are between 0 and 1 and x-Feynman is between -1 and 1.
\item The transverse momentum of the virtual photon is between 0.4 and 5.0
  GeV/c.  The lower limit ensures azimuthal angular resolution is sufficient and
  the upper cut is minimal and ensure physical kinematics.
\item The vertex originated within the z-positions of the transversely polarized
  targets defined by the target group (-294.5$<$ Z$_{\mathrm{vertex}}$ $<$-239.3
  or -219.5$<$ Z$_{\mathrm{vertex}}$ $<$-164.3 cm).
\item The vertex is within the radius of the target defined as 1.9 cm.
\end{itemize}

\begin{figure}
  \begin{center}
    \includegraphics[width=\textwidth]{EventTable}
    \caption{Events}
    \label{fig::EventTable}%
  \end{center}
\end{figure}

\section{Extraction of Asymmetries} 

\subsection{Geometric Mean}
The number of physics counts, N, detected from any particular target with any
polarization can be written as
\begin{equation}
\mathrm{N} = \mathrm{L} * \sigma * \mathrm{a},
\end{equation}

\noindent
where L is the luminosity, $\sigma$ is the cross-section to produce such an
event and a is the acceptance.  In simple words, the number of counts detected
is the number of chances for an event to occur times the probability for an
event to occur and that the event will be detected.  To get spin-dependent
counts for the left, right asymmetry, the target, polarization and left or right
direction relative to the spin should be included in the counts formula.
Generically this can be written
\begin{equation}
  \label{eqn:indexedCount}
\mathrm{N}^{\uparrow(\downarrow)}_{\mathrm{target},\mathrm{Left(Right)}} =
\mathrm{a}^{\uparrow(\downarrow)}_{\mathrm{target},\mathrm{spectrometer \;
    direction}} * \mathrm{L}^{\uparrow(\downarrow)}_{\mathrm{target}} *
\sigma_{\mathrm{Left(Right)}},
\end{equation}

\noindent
where $^{\uparrow(\downarrow)}$ denotes the target polarization,
$_{\mathrm{target}}$ is either the upstream or downstream target
$_{\mathrm{Left(Right)}}$ is left or right of the spin direction and
${_\mathrm{spectrometer \; direction}}$ denotes which side of the spectrometer
the event was detected on.

The previous definitions of the detected counts all depend on the spectrometer
acceptance.  This is a problem because the spectrometer acceptance can change
with time and space and therefore can be dependent on the physical kinematics
which produced the event.  Such dependences can cause unphysical false
asymmetries in the measurement of A$_{\mathrm{N}}$ and must therefore be removed
or must included as systematic effects.

The geometric mean asymmetry method is a way to determine the left, right
asymmetry without acceptance effects from the spectrometer.  It is defined as
\begin{equation}
  \label{eqn:ANgeomean}
\frac{1}{\mathrm{P}}\frac{\sqrt{N_{\mathrm{target,
        Left}}^{\uparrow}N_{\mathrm{target, Left}}^{\downarrow}} -
  \sqrt{N_{\mathrm{target, Right}}^{\uparrow}N_{\mathrm{target,
        Right}}^{\downarrow}} }{\sqrt{N_{\mathrm{target,
        Left}}^{\uparrow}N_{\mathrm{target, Left}}^{\downarrow}} +
  \sqrt{N_{\mathrm{target, Right}}^{\uparrow}N_{\mathrm{target,
        Right}}^{\downarrow}} },
\end{equation}

\noindent
where P represents the fraction of polarized partons. Using
Eq. \ref{eqn:indexedCount} for the definition of counts, the geometric mean
asymmetry is
\begin{equation}
\frac{1}{\mathrm{P}}\frac{\kappa \sqrt{\sigma_{Left}\sigma_{Left}} -
  \sqrt{\sigma_{Right}\sigma_{Right}}}{\kappa \sqrt{\sigma_{Left}\sigma_{Left}}
  + \sqrt{\sigma_{Right}\sigma_{Right}}},
\end{equation}

\noindent
where $\kappa$ is a ratio of acceptances defined as
\begin{equation}
  \label{eqn:ANgeomean_expand}
\frac{\sqrt{\mathrm{a}^{\uparrow}_{\mathrm{target,Jura}}
    \mathrm{a}^{\downarrow}_{\mathrm{target,Saleve}}}}
     {\sqrt{\mathrm{a}^{\uparrow}_{\mathrm{target,Saleve}}
         \mathrm{a}^{\downarrow}_{\mathrm{target,Jura}}}}.
\end{equation}

\noindent
Here the detection side of spectrometer is specified by looking down the beam
line as either Jura to mean left or Saleve to mean right.  These relations of
Jura is left and Saleve is right are only strictly true if in the target frame
the polarization is pointing straight up or straight down.  In particular if the
beam particle and the target polarization do not make a right angle in the
laboratory frame this relation will no longer be strictly true but is an
approximation for ease of notation.

Relation \ref{eqn:ANgeomean_expand} is equal to A$_{\mathrm{N}}$ if $\kappa$ is
equal to one.  However time effects can vary $\kappa$ from unity. These effects
are estimated through false asymmetry analysis and included in the systematics.
Equation \ref{eqn:ANgeomean} is therefore to a good approximation an acceptance
free method to determine A$_{\mathrm{N}}$.  It is also defined for the upstream
and downstream targets independently and therefore can used as a consistency
check between the two targets.

\section{Systematic Studies}

\section{Results} 
