%Chapter 1 
\chapter{Measurement of the Left Right Asymmetry in the Drell-Yan Process} 
\label{Chap:setup}
\ifpdf
\graphicspath{{Chapters/OneCh/Figs/Raster/}{Chapters/OneCh/Figs/PDF/}{Chapters/OneCh/Figs/}}
\else \graphicspath{{Chapters/OneCh/Figs/Vector/}{Chapters/OneCh/Figs/}} \fi

Introduction about L/R asym.  In this Chapter....  Define AN
\section{Event Selection}


\section{Extraction of Asymmetries} 

\subsection{Geometric Mean}
The number of physics counts, N, detected from any particular target with any
polarization can be written as
\begin{equation}
\mathrm{N} = \mathrm{L} * \sigma * \mathrm{a},
\end{equation}

\noindent
where L is the luminosity, $\sigma$ is the cross-section to produce such an
event and a is the acceptance.  In simple words, the number of counts detected
is the number of chances for an event to occur times the probability for an
event to occur and that the event will be detected.  To get spin-dependent
counts for the left, right asymmetry, the target, polarization and left or right
direction relative to the spin should be included in the counts formula.
Generically this can be written
\begin{equation}
  \label{eqn:indexedCount}
\mathrm{N}^{\uparrow(\downarrow)}_{\mathrm{target},\mathrm{Left(Right)}} =
\mathrm{a}^{\uparrow(\downarrow)}_{\mathrm{target},\mathrm{spectrometer \;
    direction}} * \mathrm{L}^{\uparrow(\downarrow)}_{\mathrm{target}} *
\sigma_{\mathrm{Left(Right)}},
\end{equation}

\noindent
where $^{\uparrow(\downarrow)}$ denotes the target polarization,
$_{\mathrm{target}}$ is either the upstream or downstream target
$_{\mathrm{Left(Right)}}$ is left or right of the spin direction and
${_\mathrm{spectrometer \; direction}}$ denotes which side of the spectrometer
the event was detected on.

The previous definitions of the detected counts all depend on the spectrometer
acceptance.  This is a problem because the spectrometer acceptance can change
with time and space and therefore can be dependent on the physical kinematics
which produced the event.  Such dependences can cause unphysical false
asymmetries in the measurement of A$_{\mathrm{N}}$ and must therefore be removed
or must included as systematic effects.

The geometric mean asymmetry method is a way to determine the left, right
asymmetry without acceptance effects from the spectrometer.  It is defined as
\begin{equation}
  \label{eqn:ANgeomean}
\frac{1}{\mathrm{P}}\frac{\sqrt{N_{\mathrm{target,
        Left}}^{\uparrow}N_{\mathrm{target, Left}}^{\downarrow}} -
  \sqrt{N_{\mathrm{target, Right}}^{\uparrow}N_{\mathrm{target,
        Right}}^{\downarrow}} }{\sqrt{N_{\mathrm{target,
        Left}}^{\uparrow}N_{\mathrm{target, Left}}^{\downarrow}} +
  \sqrt{N_{\mathrm{target, Right}}^{\uparrow}N_{\mathrm{target,
        Right}}^{\downarrow}} },
\end{equation}

\noindent
where P represents the fraction of polarized partons. Using
Eq. \ref{eqn:indexedCount} for the definition of counts, the geometric mean
asymmetry is
\begin{equation}
\frac{1}{\mathrm{P}}\frac{\kappa \sqrt{\sigma_{Left}\sigma_{Left}} -
  \sqrt{\sigma_{Right}\sigma_{Right}}}{\kappa \sqrt{\sigma_{Left}\sigma_{Left}}
  + \sqrt{\sigma_{Right}\sigma_{Right}}},
\end{equation}

\noindent
where $\kappa$ is a ratio of acceptances defined as
\begin{equation}
  \label{eqn:ANgeomean_expand}
\frac{\sqrt{\mathrm{a}^{\uparrow}_{\mathrm{target,Jura}}
    \mathrm{a}^{\downarrow}_{\mathrm{target,Saleve}}}}
     {\sqrt{\mathrm{a}^{\uparrow}_{\mathrm{target,Saleve}}
         \mathrm{a}^{\downarrow}_{\mathrm{target,Jura}}}}.
\end{equation}

\noindent
Here the detection side of spectrometer is specified by looking down the beam
line as either Jura to mean left or Saleve to mean right.  These relations of
Jura is left and Saleve is right are only strictly true if in the target frame
the polarization is pointing straight up or straight down.  In particular if the
beam particle and the target polarization do not make a right angle in the
laboratory frame this relation will no longer be strictly true but is an
approximation for ease of notation.

Relation \ref{eqn:ANgeomean_expand} is equal to A$_{\mathrm{N}}$ if $\kappa$ is
equal to one.  However time effects can vary $\kappa$ from unity. These effects
are estimated through false asymmetry analysis and included in the systematics.
Equation \ref{eqn:ANgeomean} is therefore to a good approximation an acceptance
free method to determine A$_{\mathrm{N}}$.  It is also defined for the upstream
and downstream targets independently and therefore can used as a consistency
check between the two targets.

\section{Systematic Studies}

\section{Results} 
