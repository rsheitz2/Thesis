\chapter{Introduction} \label{ch::intro}

More than 100 years ago, Rutherford's famous scattering experiment discovered
the atom was composed of a small but massive nucleus~\cite{Rutherford:1911zz}.
Rutherford scattered relativistic particles off a fixed target and by measuring
the angular products was able to propose the existence of a nucleus.  His
technique provided the blue prints to study the nucleus over the next century
and with higher energies and new probes the nucleus was as well found to have a
substructure.  The nucleus is now known to be composed of protons
and neutrons which are collectively called nucleons.

With still higher energy scattering experiments the nucleons were found to be
composite particles as well.  In 1964 Gell-Mann and Zweig proposed the quark
model to describe the substructure of
nucleons~\cite{GellMann:1964nj,Zweig:1964jf}.  In the quark model the proton is
composed of two u-quarks and a d-quark.  Quarks were defined as elementary
particles with fractional electric charge and a spin of 1/2.  Later Feynman
proposed the parton model to explain the results of deep inelastic scatter (DIS)
experiments~\cite{PhysRevLett.23.1415}.  In the parton model, DIS scattering
takes place between a distribution of partons inside a nucleon.  Later these two
theories were unified by the theory of quantum chromodynamics (QCD) which
described the nucleon as being held to together by force carrier gluons.  In the
QCD model, a nucleon is composed of a distribution of valence quarks surrounded
by sea quarks and gluons.  The valence quarks are responsible for the charge of
the nucleon as the sea quarks occur in quark-antiquark pairs.  QCD is now the
accepted theory for describing the dynamics inside a nucleon.

While there is a theory to describe the quark and gluon interactions, there is
currently no theory to describe the bound dynamics of quarks and gluons inside a
nucleon.  Instead the bound nuclear properties are input as parameters.  High
energy hadron scattering can be factorized as a hard scattering process
multiplied by a soft non-perturbative scattering process.  The hard scattering
can be calculated ab initio using perturbative QCD, while the soft scattering,
on the other hand, is parameterized as the hadron structure or a fragmentation
process.  Both the hadron structure and fragmentation processes are determined
experimentally as either parton distribution functions (PDF) or fragmentation
functions (FF) respectively.  The former describes how quarks and gluons are
bound in a hadron while the latter describes the probability for a quark to
fragment into a detectable hadron.

The quark and gluon parton distribution functions have been determined with
increasing precision from QCD analysis of DIS, Drell-Yan and semi-inclusive deep
inelastic scattering (SIDIS) high energy experiments.  When the scattered
nucleon has large momentum, the PDFs describe the parton distributions
in longitudinal momentum space along the direction of the nucleon's momentum.
The most recent theories of the nucleon attempt to give a three dimensional
tomographic image of the quark and gluon structure which extend beyond the
longitudinal picture to include transverse effects.  The extended distributions
include either transverse position or transverse momentum to the nucleon's
momentum.  The former are described by generalized parton distributions (GPD)
and the latter are described by transverse momentum dependent (TMD) PDFs.  This
thesis focuses on TMDs.

A unique way to probe TMDs is by studying transverse spin effects.  Polarizing
the nucleon transverse to it's momentum gives access to the internal nucleon
structure which cannot be accessed from spin averaged experiments.  As an
example the Sivers TMD PDF is a correlation between transverse spin of the
nucleon and transverse momentum of a constituent parton~\cite{Sivers}.  It
therefore makes sense that the only way to measure the Sivers function is
through transverse spin experiments.

The first TMDs were proposed to explain the results from large single-spin
asymmetries (SSA).  A SSA is defined as a normalized difference between spin
related counts from a given reaction.  One SSA, for example, is the normalized
difference between left and right counts from a transversely polarized beam.
This left-right asymmetry was first measured in 1976 and found to be non-zero in
proton-proton collisions~\cite{PhysRevLett.36.929}.  The Sivers function was
proposed to describe large SSAs which lead to a the theoretical frame work of
TMDs and is the subject of this thesis.

This thesis is organized into nine chapters.  Chapter~\ref{ch::theory_exp}
provides the theoretical and experimental background needed to describe and the
analysis results in this thesis.  Chapter~\ref{ch::compass} describes the data
taking setup, particularly by describing the experimental apparatus and the
beam.  Chapter~\ref{ch::dc05} gives details on the DC05 detector which was
needed for data taking and which the author of this thesis helped construct and
maintain.  Chapter~\ref{ch::alignment} provides details on the spectrometer
alignment which is a crucial prepossessing step for reconstructing data and
which the author of this thesis was responsible for in 2015.
Chapters~\ref{ch::tmd_analysis}-~\ref{ch::jpsi} go over the author's analysis
techniques, results and conclusion and chapter~\ref{ch::conclusion} provides
a final conclusion.
